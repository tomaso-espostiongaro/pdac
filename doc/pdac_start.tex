
\section{Getting Started}
\label{section:start}

\subsection{What is needed}

Before running \PDAC, explained in section \ref{section:run}, 
the following are be needed:
\begin{itemize}
\item The \PDAC executable pdac.x
\item The \PDAC\ parameter file pdac.dat .
\end{itemize}

\PDAC\ provides the \verb#pp.x# utility,
documented in Section~\ref{section:pp},
which is ....

\subsection{\PDAC\ parameter file pdac.dat}
\label{section:config}

\PDAC\ uses a file referred to as the {\it parameter file\/},
whose name should be pdac.dat .
This file specifies simulation options, dimensions, initial
and boundary conditions, and topography information that 
\PDAC\ should use, such as the number of timesteps to perform, 
initial temperature, etc.  
The options and values in this file control how 
the system will be simulated.  

The parameters and options in the pdac.dat file are specified 
in a set of F90 namelists followed by cards for initial and boundary
conditions.  
The options and values specified determine the exact behavior of
\PDAC, what features are active or inactive, how long the simulation
should continue, etc.  Section \ref{section:configsyntax} describes how
options are specified within a \PDAC\ configuration file.  Section
\ref{section:requiredparams} lists the parameters which are required
to run a basic simulation.  
Several sample \PDAC\ parameter files are shown
in section \ref{section:sample}.


\subsubsection{Parameter file syntax}
\label{section:configsyntax}
The pdac.dat file should contains the namelists:
\begin{verbatim}

&control
  control parameters
/

&mesh
  mesh parameters
/

&particles
  particles parameters
/

&numeric
  numeric parameters
/
\end{verbatim}

in exactly this order, followed by the cards

\begin{verbatim}

'ROUGHNESS'
roughness parameters

'MESH'
nonuniform mesh distances

'FIXED_FLOWS'
fixed flows parameters

'INITIAL_CONDITIONS
initial conditions parameters

\end{verbatim}

within a namelist, the parameters are specified as
\begin{verbatim}
keyword      =     value
\end{verbatim}
Blank lines in the namelists are ignored.  Comments are prefaced by
a \verb!\!! and may appear after the keyword assignment:
\begin{verbatim}
keyword  = value          !  This is a comment
\end{verbatim}
or may be at the beginning of a line:
\begin{verbatim}
! keyword  = value   This is a keyword commented out
\end{verbatim}
Some keywords require several values, that are specified as an array of values
separated by a comma:
\begin{verbatim}
keyword  = value_1, value_2, value_3, 
\end{verbatim}
....
\begin{verbatim}
......
\end{verbatim}

comments comments comments


\subsubsection{Required \PDAC\  parameters}
\label{section:requiredparams}

The following parameters are {\em required} for every
\PDAC\ simulation:

\begin{itemize}

\item
{\tt numsteps} (page \pageref{param:numsteps}),

\item
{\tt coordinates} (page \pageref{param:coordinates}),

\item
{\tt structure} (page \pageref{param:structure}),

\item
{\tt parameters} (page \pageref{param:parameters}),

\item
{\tt exclude} (page \pageref{param:exclude}), 

\item
{\tt outputname} (page \pageref{param:outputname}), 

\item
one of the following three:
\begin{itemize}
\item
{\tt temperature} (page \pageref{param:temperature}),

\item
{\tt velocities} (page \pageref{param:velocities}),

\item
{\tt binvelocities} (page \pageref{param:binvelocities}).
\end{itemize}

\end{itemize}

\noindent These required parameters specify the most basic properties of
the simulation.  %  that is to be performed.
