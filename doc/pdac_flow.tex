\section{\label{sect:solproc} Solution Procedure}
%
The solution of the model equations at a given time is obtained by time-marching
with a discrete time-step from the initial conditions. At each time-step the
coupled transport equations are solved by adopting an iterative solution 
procedure on the mesh.

\subsection{Main flow chart}

\begin{enumerate}
\item {\bf Startup}: the parallel environment is initialized
\item {\bf Input}: root processor reads input files and open I/O units
\item {\bf Broadcast}: input data are broadcasted among parallel processes
\item {\bf Domain Decomposition}: set the processors sub-domains, ghost cells, and communication routines
(after DD, all computation is carried out in parallel)
\item {\bf Setc}: set constants and evaluate physical parameters
\item {\bf Setup}: set initial conditions (standard atmosphere, topography, vent conditions)
\item {\bf Time Marching}: solve the model equation to advance the solution 
from the prescribed initial conditions until the final time
\item {\bf Hangup}: finalize the parallel environment
\end{enumerate}

\subsection{Time marching}
\begin{enumerate}
\item {\bf Boundary Conditions} are computed
\item {\bf EOS} Compute gas density from the thermal equation of state
\item {\bf Store} all fields at time $n\delta t$ and
compute temperature dependent viscosity and thermal conductivity of gas
\item Start Runge-Kutta cycle
\item {\bf Tilde} terms and interphase coefficients are computed
\item {\bf Iter}: Solve iteratively the coupled system equations to update velocity fields,
pressure, and volumetric fractions.
\item {\bf Gas components} mass fraction updated
\item {\bf Enthalpy} transport equations solved explicitly for all phases
\item Loop over the Runge-Kutta cycle or increase the time and restart the time-marching procedure
\item Write {\bf Output} file and restart file when prescribed
\item {\bf RETURN} if maximum time is reached
\end{enumerate}
\clearpage

%**************************************************************************
% ... user-defined objects ...
\setlength{\unitlength}{12pt}
\scriptsize
\newsavebox{\IO}
\savebox{\IO}(0,0)
{
\put(1,2){\line(1,0){5}}
\put(0,0){\line(1,0){5}}
\put(0,0){\line(1,2){1}}
\put(5,0){\line(1,2){1}}
}
\newsavebox{\IF}
\savebox{\IF}(0,0)
{
\put(0,-1){\line(2,1){2}}
\put(0,-1){\line(-2,1){2}}
\put(0,1){\line(2,-1){2}}
\put(0,1){\line(-2,-1){2}}
}
%..................................MAIN...................................
\begin{figure}
\begin{minipage}{7.5cm}
\begin{center}
{\normalsize \bf MAIN}\\[1cm]
\begin{picture}(6,25)
\multiput(3,23)(0,-3){6}{\vector(0,-1){1}}
\put(0,23){\makebox(6,2){\bf Startup}}
\put(3,24){\oval(6,2)}
\put(0,20){\makebox(6,2){\bf Input}}
\put(0,21){\usebox{\IO}}
\put(0,17){\makebox(6,2){\bf Broadcast}}
\put(3,18){\oval(6,2)}
\put(0,14){\framebox(6,2){\parbox{30mm}{\begin{center} 
\bf Domain \\ \bf decomp. \end{center}}}}
\put(0,11){\framebox(6,2){\bf Set Const.}}
\put(0,8){\framebox(6,2){\bf Setup}}
\put(0,3){\framebox(6,4){\parbox{30mm}{\begin{center} 
\bf Time \\ \bf marching \end{center} }}}
\put(6,4.5){\vector(1,0){4.5}}
\put(3,3){\vector(0,-1){1}}
\put(0,0){\makebox(6,2){\bf Hangup}}
\put(3,1){\oval(6,2)}
\end{picture}
\end{center}
\end{minipage}
%..................................TIMESWEEP..............................
\begin{minipage}{7.5cm}
\begin{center}
{\normalsize \bf TIME MARCHING}\\[1cm]
\begin{picture}(17,33)
%
\put(-2,-7){\framebox(18,41)}
\put(3,32.5){\vector(0,-1){0.5}}
\put(3,32.5){\line(1,0){11}}
\put(0,30)
  {\framebox(6,2)
    {\parbox{30mm}
      {\begin{center} \bf Boundary \\ \bf Conditions \end{center}}
    }
  }
\put(3,30){\vector(0,-1){1}}
\put(0,27){\framebox(6,2){\bf EOS}}
\put(3,27){\vector(0,-1){1}}
%
\put(0,21)
  {\framebox(6,5)
    {\parbox{25mm}{\begin{center} 
      \bf Store\\
      \bf Fields at $\bf n\delta t$ \\
      \bf Viscosity\\
      \bf Conductivity \end{center} }
    }
  }
\put(6.5,23.5){\bf $dt_0=dt$}
\put(3,21){\vector(0,-1){2}}
\put(0,17)
  {\framebox(6,2)
    {\parbox{30mm}
      {\begin{center} \bf Tilde \end{center}}
    }
  }
\put(0,16){\framebox(6,1){\bf Drag}}
\put(0,15){\framebox(6,1){\bf Gravity}}
\put(3,15){\vector(0,-1){1}}
\put(0,9)
  {\framebox(6,5)
    {\parbox{25mm}
      {\begin{center} 
         \bf Iter \\
         \bf solver for \\
         \bf implicit\\
         \bf coupling
       \end{center}
      }
    }
  }
\put(9,13)
  {\makebox(4,2)
    {\parbox{20mm}
      {\begin{center} 
         \bf Runge-\\
         \bf Kutta \\
         \bf Cycle
       \end{center}
      }
    }
  }
\put(9.2,9){\tiny {\bf $dt=\frac{dt}{N+1-k}$}}
\put(3,9){\vector(0,-1){1}}
\put(0,6)
  {\framebox(6,2)
    {\parbox{25mm}
      {\begin{center} 
         \bf Gas \\
         \bf Components
       \end{center}
      }
    }
  }
\put(3,6){\vector(0,-1){1}}
\put(0,4){\framebox(6,1){\bf Enthalpy}}
\put(3,4){\line(0,-1){2.0}}
%
\put(3,3){\vector(1,0){6}}
\put(9,3){\vector(0,1){17}}
\put(9,20){\vector(-1,0){6}}
%
\put(0,0){\makebox(6,2){\bf Output}}
\put(0,1){\usebox{\IO}}
\put(3,0){\vector(0,-1){1}}
\put(3,-2){\usebox{\IF}}
\put(1,-3){\makebox(4,2){$t<t_{stop}$}}
\put(5,-2){\vector(1,0){1}}
\put(5,-3.0){yes}
\put(3,-3){\vector(0,-1){1}}
\put(1.5,-3.5){no}
\put(3,-5){\oval(6,2)}
\put(1,-6){\makebox(4,2){\bf Return}}
%
\put(6,-2.0){\vector(1,0){8}}
\put(5,-1.5){\bf $dt=dt_0; \quad t=t+dt$}
\put(14,-2.0){\vector(0,1){18.0}}
\put(14,16){\vector(0,1){16.5}}
%
\end{picture}
\begin{picture}(16,4.5)
\end{picture}
\end{center}
\end{minipage}
\vspace{1cm}
\caption{\label{fig:prog_chart} Flow chart of the PDAC {\em main} program and 
time-advancement procedure. The whole time-advancement block is repeated at 
each time-step, from the initial condition to the prescribed final time.} 
\end{figure}
\clearpage
%...................................ITER..............................
\normalsize
%
\begin{figure}
\scriptsize
\setlength{\unitlength}{10pt}
\begin{center}
{\normalsize \bf ITER: Iterative Solver for Pressure Coupling}\\[1cm]
\begin{picture}(19,53)
\put(-2,-2){\framebox(24,57)}
\put(6,52){\framebox(6,2){\bf $C$, $\frac{dD_g}{dP}$}}
\put(9,52){\vector(0,-1){1}}
\put(6,49){\framebox(6,2){\parbox{30mm}
{\begin{center} 
\bf GUESS \\
\bf velocities\\
\end{center}}}}
\put(9,49){\vector(0,-1){1}}
\put(6,46){\framebox(6,2){\parbox{30mm}
{\begin{center} 
\bf Gas \\
\bf Mass Fluxes\\
\end{center}}}}
\put(9,46){\vector(0,-1){1}}
\put(9,44){\oval(6,2)}
\put(6,43){\makebox(6,2){\parbox{30mm}
{\begin{center} 
\bf START\\
\bf Mesh Sweep\\
\end{center}}}}
\put(9,43){\vector(0,-1){2}}
\put(6,39){\framebox(6,2){\parbox{30mm}
{\begin{center} 
\bf Gas Mass\\
\bf Residual $D_g$\\
\end{center}}}}
\put(9,39){\vector(0,-1){1}}
\put(9,37){\usebox{\IF}}
\put(7,35.9){\makebox(4,2){\bf $D_g< C$}}
\put(11,37){\line(1,0){7}}
\put(18,37){\vector(0,-1){2}}
\put(15,34){\framebox(6,1){$ \bf \epsilon_s$}}
\put(15,33){\framebox(6,1){$ \bf \epsilon_g = 1-\sum\epsilon_s$}}
\put(15,32){\framebox(6,1){$ \bf \rho_g'=\epsilon_g\rho_g$}}
\put(18,32){\line(0,-1){22}}
\put(18,24){\vector(0,-1){1}}
\put(18,10){\vector(-1,0){9}}
\put(9,36){\vector(0,-1){2}}
\put(6,32){\framebox(6,2){\bf Correct $P_{ij}$}}
\put(6,30){\framebox(6,2){\bf EOS $\rho_g(P,T)$}}
\put(9,30){\vector(0,-1){1}}
\put(6,27){\framebox(6,2){\parbox{30mm}
{\begin{center} 
\bf GUESS \\
\bf velocities\\
\end{center}}}}
\put(9,27){\vector(0,-1){1}}
\put(6,24){\framebox(6,2){\parbox{30mm}
{\begin{center} 
\bf Particles \\
\bf Mass Fluxes\\
\end{center}}}}
\put(9,24){\vector(0,-1){1}}
\put(6,22){\framebox(6,1){$ \bf \epsilon_s$}}
\put(6,21){\framebox(6,1){$ \bf \epsilon_g = 1-\sum\epsilon_s$}}
\put(6,20){\framebox(6,1){$ \bf \rho_g'=\epsilon_g\rho_g$}}
\put(9,20){\vector(0,-1){1}}
\put(6,17){\framebox(6,2){\parbox{30mm}
{\begin{center} 
\bf Gas \\
\bf Mass Fluxes\\
\end{center}}}}
\put(9,17){\vector(0,-1){1}}
\put(6,14){\framebox(6,2){\parbox{30mm}
{\begin{center} 
\bf Gas Mass\\
\bf Residual $D_g$\\
\end{center}}}}
\put(9,14){\vector(0,-1){1}}
\put(9,12){\usebox{\IF}}
\put(7,10.9){\makebox(4,2){\bf $D_g< C$}}
\put(9,11){\line(0,-1){2}}
\put(9,8){\oval(6,2)}
\put(6,7){\makebox(6,2){\bf NEXT cell}}
\put(9,7){\line(0,-1){1}}
\put(9,6){\vector(-1,0){7}}
\put(2,5){\bf Mesh LOOP}
\put(7,12){\vector(-1,0){4}}
\put(3,11){\bf Cell LOOP}
\put(2,6){\line(0,1){36}}
\put(2,23){\vector(0,1){1}}
\put(3,12){\line(0,1){23}}
\put(3,23){\vector(0,1){1}}
\put(2,42){\vector(1,0){7}}
\put(3,35){\vector(1,0){6}}
\put(9,6){\vector(0,-1){1}}
\put(9,4){\usebox{\IF}}
\put(7,2.9){\makebox(4,2){\bf conv?}}
\put(7,4){\vector(-1,0){7}}
\put(1,3){\small \bf SOR}
\put(0,4){\line(0,1){41.5}}
\put(0,23){\vector(0,1){1}}
\put(0,45.5){\vector(1,0){9}}
\put(9,3){\vector(0,-1){2}}
\put(9,0){\oval(6,2)}
\put(6,-1){\makebox(6,2){\bf RETURN}}
\end{picture}
\end{center}
\caption{\label{fig:iter_chart} Flow chart of the algorithm in the iterative 
solver: at each time-step, converegence is reached by iteratively correcting
the pressure field in each cell and re-calculating the velocity fields. 
Corrections are propagated to neighbour cells by using Successive 
Over-Relaxation (SOR) technique until the gas mass residual is less than a 
prescribed value in the whole computational domain.}
\end{figure}
\normalsize
%**************************************************************************
\clearpage

\subsection{Iterative solver}
The core of the numerical algorithm is the iterative procedure to compute
the simultaneous solution of the mass and momentum equations, that are
coupled through the pressure term and the velocities.\\

{\bf Initial guess}\\
The procedure starts with a pressure field that may be represented
by either the initial pressure distribution of standard atmosphere (initial
conditions) or the pressure distribution of the previous time-step.
By using such a pressure field, momentum equations are solved explicitly 
(i.e. the matrix of momenta is directly inverted) to guess the velocity 
of each phase in each cell. The velocity field thus obtained is
biassed by the ``wrong'' pressure field. The basic idea of the solver is
to iteratively correct the pressure in each cell by minimizing the residual
of the gas continuity equation until convergence is achieved.\\

%
{\bf Mesh sweep}\\
The residual of the gas mass balance equation is evaluated cell-by-cell
by using the old volumetric fraction and the velocity field just computed: 

\begin{eqnarray}
D_{i,j,k} =  (\epsilon_g \rho_g)_{i,j,k}^{*} - (\epsilon_g \rho_g)_{i,j,k}^{n}
-\frac {\delta t} {R_i \delta r_i} \langle R (\epsilon_g \rho_g) u_g \rangle_{i,j,k}^{*}
- \frac {\delta t} {\delta y_j } \langle (\epsilon_g \rho_g) v_g \rangle_{i,j,k}^{*}
- \frac {\delta t} {\delta z_k } \langle (\epsilon_g \rho_g) w_g \rangle_{i,j,k}^{*}
\nonumber
\end{eqnarray}
%
and compared with the convergence criterion. Starred quantities are evaluated at current
iteration step. At the first iteration $*=n$, whereas $*=n+1$ when convergence is
reached in the whole computational domain.
In the code the convergence is reached when $D_{i,j}$ is $\le 10^{-8} \times 
(\epsilon_g \rho_g)_{i,j}^{n}$. 

If the convergence criterion is satisfied in a cell, the continuity equations of the solid 
phases are solved to determine the updated particulate volumetric fraction of each phase.
The new gas volumetric fraction is then computed from:
%
\begin{eqnarray}
\epsilon_g  = 1 - \sum_{k=1}^{n} \epsilon_{k}
\nonumber
\end{eqnarray}
%
and the calculation proceeds to the next cell.

If the residual is greater than the prescribed limit the {\em in-cell} correction procedure
is started, proceedings throughout the following steps:
\begin{itemize}
\item Pressure is first corrected by using one of the iterative techniques illustrated in section 
\ref{sect:pressure}. 
\item The new thermodynamic gas density is computed according to the prescribed
equation of state (EOS in Fig.~\ref{fig:iter_chart}). 
\item New velocities are computed by solving the system of coupled momentum transport 
equations with the new pressure and density, and the old volumetric fractions. 
\item Particle volumetric fractions can be now computed from solid mass transport 
equations with the updated velocity fields. The gas volumetric fraction is also computed 
from the closure equation.
\item The new gas volumetric fraction and the new velocities are used in the gas 
continuity  equation to compute the new mass residual.
\item If the gas mass residual is too large the in-cell correction is repeated, otherwise
the entire procedure is repeated in the next cell.
\end{itemize}
The iteration begins at the left-bottom corner fluid cell. The computation proceeds 
from left to right and from bottom to top until the entire computational domain is covered.\\

{\bf Successive Over-Relaxation (SOR)}\\
When all cells have been updated the mesh sweep is repeated (by applying over-relaxation)
to propagate the corrections to the neighbour cells, until convergence is simultaneously reached
in all the cells. 
As few tens of iterations are needed to reach simultaneous convergence
in all cells, in-cell pressure corrections are limited to a max number of inner iteration,
even if correction is not usually completed. In fact, a further correction could be needed in the next 
mesh loop, and therefore a good trade-off must be selected between the number of pressure corrections
({\tt inmax}) and the number of mesh iterations ({\tt maxout}).\\
%
\subsection{\label{sect:pressure}Pressure correction algorithm}
%
The above reported convergence criterion is reached by adjusting 
pressure through a combination of different numerical methods.
\begin{enumerate}
\item
The initial adjustment is done by using Newton's method (Fig.~\ref{fig:newtonm}). 
If $P_g$ is the gas pressure and the superscripts $s$ indicates the $s$-th iteration:
%
\begin{eqnarray}
(P_g)_{i,j}^{s+1} =  (P_g)_{i,j}^{s} - \omega dP_g  \label{eq:correction}\\
dP_g = \frac {D_{i,j}^s} {\left(\partial D_{i,j} / \partial (P_g)_{i,j}\right)^s}\label{eq:dDdP}
\end{eqnarray}
%
where the index $s$ indicates the iteration level and $\omega$ is a relaxation parameter.
The derivative of the residue with respect to pressure has been computed by (see below for
a derivation of this expression):
%
\begin{eqnarray}
\frac{\partial D_{i,j}} {\partial (P_g)_{i,j}} & \approx & 
\frac{\epsilon_g}{C_{i,j}^{2}} +
\frac{1}{R_i}\Big(\frac {\delta t} {\delta r_i}\Big)^2 \big( 
R_{i+\frac{1}{2}} (\epsilon_g)_{i+\frac{1}{2},j,k} +
R_{i-\frac{1}{2}} (\epsilon_g)_{i-\frac{1}{2},j,k} \big)+\\
&&+\Big(\frac {\delta t} {\delta y_j}\Big)^2 \big( 
(\epsilon_g)_{i,j+\frac{1}{2},k} + (\epsilon_g)_{i,j-\frac{1}{2},k} \big)+
\Big(\frac {\delta t} {\delta z_k}\Big)^2 \big( 
(\epsilon_g)_{i,j,k+\frac{1}{2}} + (\epsilon_g)_{i,j,k-\frac{1}{2}} \big)
\nonumber
\end{eqnarray}
%
where the sound speed of the gas phase $C_{i,j}$ is determined from the equation of state. This expression
is evaluated once at the beginning ($\frac{dD_g^o}{dP_g}$ in Fig.~\ref{fig:newtonm}) 
and it is not recalculated during the iteration, since its variations do not affect strongly 
the convergence of the method.\\
Newton's method is used until the residual changes sign.
\begin{figure}[h]
\centerline{\psfig{figure=./newtonm.eps,height=8cm}}
\caption{\label{fig:newtonm} Sketch of the Newton's method used to find the 
zero of the function $D_g(P_g)$: the slope of the function is computed once 
and used to iteratively approach the zero.  If the value of the function 
$D_g$ changes sign (point 2) the iterative Newton's method is stopped and 
the secant method of Fig.~\ref{fig:secantm} is used.}
\end{figure}
%
\item A new approximation of the solution is reached in a further stage by using the 
secant method (Fig.~\ref{fig:secantm}), i.e. by evaluating Eq.~\ref{eq:dDdP} with:
%
\begin{eqnarray}
\partial (P_g)_{i,j}^{s} / \partial D_{i,j} = \bigg( \frac {(P_g)_{i,j}^{s-1} - (P_g)_{i,j}^{s}}
{D_{i,j}^{s-1} - D_{i,j}^{s}} \bigg)
\nonumber
\end{eqnarray}
%
$(P_g)_{i,j}^{s-1}$ and $(P_g)_{i,j}^{s}$ are the two last values obtained by the Newton's 
method, and their corresponding residuals have opposite sign. The secant method is used once 
to obtain a third pressure between these two values. 
%
\begin{figure}[h]
\centerline{\psfig{figure=./secantm.eps,height=8cm}}
\caption{\label{fig:secantm} Sketch of the secant method used when the 
iterative Newton's method changes sign: a third point between points P1 and P2 
is found by tracing the secant between 1 and 2.} 
\end{figure}
%
\item With three points surrounding the solution, the bi-secant method (Fig.~\ref{fig:bisecantm}) 
can be applied iteratively to fastly converge to the zero-residual pressure.
%
\begin{figure}[h]
\centerline{\psfig{figure=./bisecantm.eps,height=8cm}}
\caption{\label{fig:bisecantm} Sketch of the iterative bisecant method used with three points surrounding
the zero of the function $D_g(P_g)$. The two secants between 1-3 and 2-3 are traced to find points
A and B on the $D_g=0$ axis. Point 4 lays mid-way between A and B. In the case shown (where P4 is 
between P2 and P3) point 1 is discarded and the procedure is repeated with points 2,3,4.}
\end{figure}
%
\end{enumerate}
%

The total number of iterations is usually of the order of 3-4 for the pressure adjustment
in each cell, and of few tens for the outer iterations over the entire domain.\\[5mm]
%
\clearpage
\subsubsection{Derivative of gas mass residual}
%
A few manipulation and approximation of the momentum transport equations are necessary
to obtain the derivative of the gas mass-residual with respect to pressure, used in 
Eq.~\ref{eq:dDdP}.
The procedure is presented in 1D and cartesian coordinates although its extension to 2D 
and 3D is straighforward.

The residual is obtained from the mass transport equation as:
\begin{eqnarray}
\label{eq:resid}
(D_g)_i & = &  (\epsilon_g \rho_g)_i^{*} - (\epsilon_g \rho_g)_i^{n}
+ \frac {\delta t} {\delta x_i } \langle (\epsilon_g \rho_g) u_g \rangle_i^{*} \\ \nonumber
& = & (\epsilon_g \rho_g)_i^{*} - (\epsilon_g \rho_g)_i^{n}
+ \frac {\delta t} {\delta x_i } 
\left[ 
(\epsilon_g \rho_g)^n_{i+\frac{1}{2}}(u_g)^*_{i+\frac{1}{2}} -
(\epsilon_g \rho_g)^n_{i-\frac{1}{2}}(u_g)^*_{i-\frac{1}{2}}
\right]
\end{eqnarray}
In this equation the term $(\epsilon_g \rho_g)_i^{n}$ is explicitly given,
the terms $(\epsilon_g \rho_g)^n_{i+\frac{1}{2}}$ and $(\epsilon_g \rho_g)^n_{i-\frac{1}{2}}$ are
modeled by using some upwind scheme (such as the First Order Upwind), whereas starred terms
are guessed from some approximated equation. The first term $(\epsilon_g \rho_g)_i^{*}$ is
obtained from the previous iteration, or is equal to $(\epsilon_g \rho_g)_i^{n}$ in the
first iteration. The starred terms in the square bracket are evaluated by using the momentum
equation. As we cannot solve the momentum equation explicitly, we retain only the terms
that are relied to pressure variation (as we want to calculate the derivative with respect
to pressure variations!) so that the momentum equation can be written:
\begin{eqnarray}
(\epsilon_g \rho_g)^n_{i+\frac{1}{2}}(u_g)^*_{i+\frac{1}{2}} \approx
(\epsilon_g \rho_g)^n_{i+\frac{1}{2}}(u_g)^n_{i+\frac{1}{2}} -
\frac {\delta t} {\delta x_{i+\frac{1}{2}}}
(\epsilon_g)^n_{i+\frac{1}{2}}(P_{i+1}-P_i)^*
\nonumber
\end{eqnarray}
Now, if this equation is substituted into \ref{eq:resid}, we obtain:
\begin{eqnarray}
(D_g)_i & \approx &  (\epsilon_g \rho_g)_i^{*} - (\epsilon_g \rho_g)_i^{n} \\
& + & \frac{\delta t}{\delta x_i } 
\left[ 
(\epsilon_g \rho_g u_g)^n_{i+\frac{1}{2}} - (\epsilon_g \rho_g u_g)^n_{i-\frac{1}{2}} 
- \frac{\delta t}{\delta x_{i+\frac{1}{2}}} (\epsilon_g)^n_{i+\frac{1}{2}}(P_{i+1}-P_i)^*
+ \frac{\delta t}{\delta x_{i-\frac{1}{2}}} (\epsilon_g)^n_{i-\frac{1}{2}}(P_i-P_{i-1})^*
\right]
\nonumber
\end{eqnarray}
Deriving this expression with respect to $P_i$ we obtain the final expression, where
the first term was approximated as:
\begin{displaymath}
\frac{\partial (\epsilon_g \rho_g)_i^{*} } {\partial P_i } \approx 
(\epsilon_g)_i^n \frac{\partial (\rho_g)_i^{*} } {\partial P_i } = 
\frac{(\epsilon_g)_i^n}{C^2}
\end{displaymath}
where $C$ is the adiabatic speed of sound of the gas.\\
\begin{eqnarray}
\frac{\partial D_{i}} {\partial (P_g)_{i}} \approx
\frac{\epsilon_g}{C_{i}^{2}} +
\frac {\delta t} {\delta x_i} \left(
\frac{\delta t}{\delta x_{i+\frac{1}{2}}}(\epsilon_g)^n_{i+\frac{1}{2}} + 
\frac{\delta t}{\delta x_{i-\frac{1}{2}}}(\epsilon_g)^n_{i-\frac{1}{2}} \right)
\nonumber
\end{eqnarray}
%
