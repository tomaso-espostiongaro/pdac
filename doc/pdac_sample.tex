\section{Sample configuration files}
\label{section:input_sample}
This section describes some of the \PDAC\ configuration files to be used
as templates. The files are included in the \DIR{examples} directory.
To run an example, copy the corresponding input file into the 
\FIL{pdac.dat} file.
\prettypar
\subsection{2D axisymmetric Plinian column}
Input file: \FIL{Plinian\_2D.dat}
\begin{itemize}
\item one particle phase (50 $\mu m$ diameter, $1500 Kg/m^3$ density)
\item no topographic profile specified
\item Smagorinsky turbulence model with $C_S=0.33$
\item high-order (Ultraquick scheme) discretization of convective fluxes
\item non-uniform mesh with minimum cell size of 30m and maximum cell size
      of 200 m in both radial direction. Uniform vertical discretization of
      30m.
\item vent conditions specified in cells (2,1) and (3,1): gas and particle
      vertical velocity $w = 110 m/s$, Pressure balanced ($1atm$), $0.1\%$
      particle fraction, equal gas and particle temperature ($T=900 K$).
      Water vapour is injected at the vent.
\item initial atmospheric conditions modified with respect to the standard 
      atmosphere
\end{itemize}

\newpage
\subsection{Classical Sod problem}
The Sod problem is the numerical analogous to the laboratory shock-tube 
experiment, and it is used to check numerical discretization schemes on 
the standard 1D Riemann problem.
\begin{itemize}
\item Pseudo-one-dimensional (transverse direction has one computational cell and two boundary cells)
\item Cartesian regular mesh
\item One inviscid gas species
\item Particle phase is dummy (volumetric fraction is set to zero). Please note that
      it is not allowed to set {\tt nsolid = 0}, otherwise the program stops.
\item Gravity force is neglected
\item Density is specified as initial condition instead of temperature
\end{itemize}

\newpage
\subsection{Simulation of pyroclastic flow at Vesuvius}
Input file: \FIL{L\_w2\_t950\_B.dat}
\begin{itemize}
\item 2D axisymmetric simulation
\item standard multiphase model
\item volcano topography is introduced by specifying ``blocking cells''.
These are described by topographic blocks in the {\tt FIXED FLOW} card.
\item roughness at ground is specified
\end{itemize}
%
\newpage
\subsection{Simulation of pyroclastic flow at Vesuvius (new!)}
Input file: \FIL{L\_w2\_t950\_B\_new.dat}
\begin{itemize}
\item The same input conditions as above 
\item automatic grid generation
\item new specification of the topography and
      immersed boundaries technique switched on
\item Input file topoB.dat contains the Northern profile of Mt.Vesuvius 
\end{itemize}

%
\newpage
\subsection{3D simulation of a volcanic column}
\begin{itemize}
\item Simulation A6 in Neri and Macedonio, {J. Geophys. Res.} 1996
\item fully 3D simulation on a non-uniform cartesian mesh
\item no topography specified
\end{itemize}
