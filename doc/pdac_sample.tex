\section{Sample configuration files}
\label{section:input_sample}
This section describes some of the \PDAC\ configuration files to be used
as templates. The files are included in the \DIR{examples} directory.
To run an example, copy the corresponding input file into the 
\FIL{pdac.dat} file.
\prettypar
%
\subsection{2D simulation of pyroclastic flows at Vesuvius}
Input file: \FIL{L\_w2\_t950\_B.dat}\\
Reference: Simulation L\_w2\_t950\_B in Todesco et al. (2002)\\
Main features:
\begin{itemize}
\item Duration: 800s
\item Includes the roughness model of ground turbulence
\item 2D axisymmetric simulation
\item two particle phases of 30 and 500 $\mu$m
\item first order discretization of convective fluxes
\item the mesh and the domain are defined in the \OBJ{'MESH'} card: the
      mesh size varies from 7.5 m to 100 m in radial direction and from 10 to
      1000 m along the vertical
\item inlet conditions are described imposed in one block (from i=2 to i=7)
      in the {\tt FIXED FLOW} card. The corrisponding vent radius is 57.5 m.
\item volcano topography is introduced by specifying ``blocking cells''
\end{itemize}
%
\newpage
\subsection{3D simulation of a volcanic column}
Input file: \FIL{Vesuvio\_SIM3.dat}\\
Reference: Simulation SIM3 in Esposti Ongaro et al. (2008)\\
Main features:
\begin{itemize}
\item Simulation time: 800s
\item Domain of $13\times11\times8$ km$^3$.
      Non-uniform mesh with minimum cell size of 20 m and maximum cell size
      of 100 m in both horizontal direction. Non-uniform vertical 
      discretization from 20 to 100 m.
\item topographic profile is specified by the DEM file Vesuvio\_10m.dat
      (also included in the directory). The immersed boundary technique is used.
      The topography is flattened to impose inlet conditions at
      the exit of the crater, after flow decompression.
\item the inlet is located on the mesh center. The density is modified on 
      the vent perimeter to adjust the mass flow rate to the prescribed
      value of $5\times10^7$ kg/s. The velocity is modified from the
      'top-hat' profile, with a coefficient 1.1. 
      Pressure is balanced to atmospheric value at the same quota.
      Velocity and density correspond to the values reached after decompression.
\item standard initial atmospheric conditions 
\item two particle phases of 30 and 500 $\mu$m.
\item second-order (with minmod limiter) discretization of convective fluxes
\end{itemize}
%
\newpage
\subsection{3D simulation of the Boxing-Day blast at Montserrat}
Input file: \FIL{Plinian\_2D.dat}\\
Reference: Simulation B-3D in Esposti Ongaro et al. (2007b)\\
Main features:
\begin{itemize}
\item Simulation time: 90s
\item Domain of $9\times9\times6$ km$^3$.
      Non-uniform mesh with minimum cell size of 20 m and maximum cell size
      of 100 m in both horizontal direction. Non-uniform vertical 
      discretization from 20 to 200 m.
\item topographic profile is specified by the DEM file SHV\_DEM\_20m\_wide.dat
      (also included in the directory). The immersed boundary technique is used.
\item the dome of volume $12\times 10^6$ m$^3$ is located on the mesh center, 
      above the topography 
      (the topography is not modified by the dome). Pressure distribution
      is set by assuming a gas flow through the porous dome with permeability
      $k=10^{-12}$. Total particle fraction is 0.9 (i.e., porosity is 0.1)
\item standard initial atmospheric conditions 
\item three particle phases of 5000, 500 and 50 $\mu$m.
\item second-order (with minmod limiter) discretization of convective fluxes
\end{itemize}
%
\newpage
\subsection{Classical Sod problem}
Input file: \FIL{shock\_tube\_1.dat}\\
Reference: Esposti Ongaro et al. (2007a)
Main features:
\begin{itemize}
\item The Sod problem is the numerical analogous to the laboratory shock-tube 
experiment, and it is used to check numerical discretization schemes on 
the standard 1D Riemann problem.
\item Pseudo-one-dimensional (transverse direction has one computational cell and two boundary cells)
\item Cartesian regular mesh
\item One inviscid gas species
\item Particle phase is dummy (volumetric fraction is set to zero). Please note that
      it is not allowed to set {\tt nsolid = 0}, otherwise the program stops.
\item Gravity force is neglected
\item Density is specified as initial condition instead of temperature
\end{itemize}

