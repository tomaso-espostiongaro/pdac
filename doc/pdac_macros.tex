%
% generally useful macros
%
\newcommand{\REFAND} {\&}
\newcommand{\ETALNP}{\mbox{\it et al}}
\newcommand{\ETAL}{\mbox{\ETALNP{\it.}}}
\newcommand{\eqnref}[1] {\mbox{eq (\ref{#1})}}
\newcommand{\mycite}[2] {\cite{#2}}

\newcommand{\PDAC} {PDAC}
\newcommand{\PDACDATE} {\today}
\newcommand{\PDACAUTHORS} {
 A.~Neri, G.~Macedonio, D.~Gidaspow, T.~Esposti Ongaro, C.~Cavazzoni }
\newcommand{\PDACVERSION} {1.0}

%
% macros for style conventions when describing the program.
%

% name of class or object in program
\newcommand{\OBJ}[1] {{\bf\tt#1}}

% function arguments
\newcommand{\FA}[2] {{\rm{\bf#1}\ {\it#2}}}

% global function name
\newcommand{\FN}[3] {{\rm\bf#1}\ {\tt #2(}#3{\tt)}}

% class member function name
\newcommand{\FNO}[4] {{\rm\bf#2}\ \OBJ{#1::}{\tt#3(}#4{\tt)}}

% list item, for optional components, parameters, etc.
\newcommand{\TTLISTITEM}[1] {\item {\tt #1} \\}
\newcommand{\RMLISTITEM}[1] {\item {\rm #1} \\}
\newcommand{\BOLDLISTITEM}[1] {\item {\bf #1} \\}
\newcommand{\EMLISTITEM}[1] {\item {\em #1} \\}
\newcommand{\LISTITEM}[1] {\RMLISTITEM{#1}}

%
% other generally useful macros
%

% Other program names, formatted nicely
\newcommand{\FEP} {FEP}
\newcommand{\VMD} {VMD}
\newcommand{\MDCOMM} {MDComm}
\newcommand{\MDSCOPE} {MDScope}
\newcommand{\CESB} {MDScope}
\newcommand{\ALLNAMES} {MDScope}
\newcommand{\SMALLMDSCOPE} {mdscope}
\newcommand{\SMALLCESB} {mdscope}
\newcommand{\SMALLALLNAMES} {mdscope}

% full name for MDScope, i.e., what it stands for
\newcommand{\MDSCOPENAME} {Molecular Dynamics computational environment}

% title of MDScope paper
\newcommand{\MDSCOPEPAPER} {MDScope: A Visual Computing Environment for
Structural Biology}


\newcommand{\DOCTITLE} {User's Guide}
\newcommand{\DOCDESC} {
The \PDAC\ {\em\DOCTITLE\/} describes how to run and use the 
various features of the Piroclastic Density Analysis Code \PDAC.  
This guide includes the capabilities of the program, how 
to use these capabilities, the necessary input files and 
formats, and how to run the program both on uniprocessor 
machines and in parallel.}

\newcommand{\PG}{\PDAC\ {\it Programmer's Guide\/}}
\newcommand{\UG}{\PDAC\ {\it User's Guide\/}}
\newcommand{\prettypar}{

\smallskip

}

\newcommand{\eg}{{\it e.g.\/}}
\newcommand{\ie}{{\it i.e.\/}}

\newcommand{\KEY}[1]{{\tt #1}}
\newcommand{\IKEY}[1]{{\tt #1\index{#1 psfgen command}}}
\newcommand{\OKEY}[1]{$[${\tt #1}$]$}
\newcommand{\ARG}[1]{$<${\em #1}$>$}
\newcommand{\OARG}[1]{$[${\em #1}$]$}
\newcommand{\ARGDEF}[2]{$<${\em #1}$>$: #2}
\newcommand{\KEYDEF}[2]{{\tt #1}: #2}
\newcommand{\COMMAND}[4]{%
  #1 \\ {\bf Purpose:} #2 \\ {\bf Arguments:} #3 \\ {\bf Context:} #4 }

\newcommand{\icommand}[1]{#1\index{#1 command}}

\newcommand{\NAMDCONF}[4]{%
%  \addcontentsline{toc}{subparagraph}{#1}%
  {\bf \tt #1 } $<$ #2 $>$ \\%
  \index{#1 parameter}
  {\bf Acceptable Values: } #3 \\%
  {\bf Description: } #4%
}

\newcommand{\NAMDCONFWDEF}[5]{%
%  \addcontentsline{toc}{subparagraph}{#1}%
  {\bf \tt #1 } $<$ #2 $>$ \\%
  \index{#1 parameter}
  {\bf Acceptable Values: } #3 \\%
  {\bf Default Value: } #4 \\%
  {\bf Description: } #5%
}

\newcommand{\XNCOMP}[3]{%
  {\bf \NAMD\ Parameter: \tt #1 } \\%
  {\bf X-PLOR Parameter: \tt #2 } \\%
  #3%
}

