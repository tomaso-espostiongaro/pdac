\section{Introduction}
\label{section:intro}

\PDAC\ is a numerical code that has been developed to solve the multiphase
flow equations for the simulation of pyroclastic atmospheric dispersal dynamics.
It is specifically designed for high performance computing on UNIX platforms.
This document describes how to use 
\PDAC, its features, and the platforms on which it runs.
The document is divided into six sections:
\begin{description}
\item[Section \ref{section:intro}] gives an overview of \PDAC.
\item[Section \ref{section:files}] describes \PDAC\ file formats.
\item[Section \ref{section:input_par}] lists basic simulation options.
\item[Section \ref{section:input_sample}] provides sample configuration files.
\item[Section \ref{section:run}] gives details on running \PDAC.
\item[Section \ref{section:avail}] gives details on installing \PDAC.
\end{description}

We have tried to make this document complete and easy to understand as well 
as \PDAC\ itself easy to install and run.
We welcome any suggestion for improving the documentation and code itself
at \PDACADDRESS\

\subsection{The \PDAC\ main features}

The \PDAC\ code solves the multiphase flow transport equations for
density, momentum and energy of a gas-pyroclast mixture in a steady
standard atmosphere.
The gas phase can be composed of several chemical components leaving
the crater - such as water vapor, carbon dioxide, etc. - and
atmospheric air (considered as a single chemical component).
The pyroclasts are described by $N$ phases of solid particles,
each one characterized by a diameter, density, specific
heat, thermal conductivity, and viscosity, and considered
representative of a granulometric class commonly present in
the eruptive mixture. Momentum and energy exchange between the gas
and the different particulate phases are expressed through semi-empirical
correlation equations.

\prettypar
\PDAC\ has several important features that make it a useful tool for the
simulation of volcanic columns and pyroclastic flows: 

\begin{itemize}

\item{\bf Robust over a wide range of flow regimes}
\begin{itemize}
\item from weak to strong gas-particle coupling;
\item from laminar to turbulent regimes;
\item from quasi-isothermal to highly buoyant flows;
\item from subsonic (incompressible) to supersonic (compressible) regimes.
\end{itemize}

\item{\bf Unlimited number of particles classes}\\
The number of particle classes used to approximate an observed distribution
of grain sizes is limited only by the computational resources available.
Up to 6 particulate phases have been used successfully. The size and density
of particles is nevertheless constrained by the multiphase flow assumptions,
that limit the diameter of grain particles from about $10 \mu m$ to a few
millimeters.

\item{\bf Up to 7 gas species}\\
Up to 7 gas species can be considered as chemical components of the gas phase. The 
molecular composition allowed are: $O_2,N_2,CO_2,H_2,H_2O,Air,SO_2$.
Atmospheric air is considered as a single component with averaged properties.
Gas species are treated as tracers and no chemical reactions are allowed up to
now. Future versions of \PDAC\ will hopefully consider interphase mass transfer
(such as water vapour condensation) and chemical reactions.

\item{\bf Atmospheric stratification}\\
A standard stratified atmosphere can be assigned as initial condition.
The temperature gradient within the different atmospheric layers are
the Earth averaged values, whereas pressure is computed by assuming hydrostatic
equilibrium. No vertical stratification of humidity is assumed at this time. 
Cross wind can be imposed in a Cartesian reference frame.

\item{\bf Control of simulation options}\\
The simple textual input file allows \PDAC\ to be started from prescribed
initial conditions or from a binary restart file, representing the state
of the main flow fields when the code stops for any reason. When the code
is started from scratch, vent inlet flow conditions the volcano topography, 
and the atmospheric conditions need to be  specified as input conditions. 
When the code is restarted, the final state saved in the restart file is used 
as initial condition. The computational mesh and all numerical parameters 
are easily specified in the input file.

\end{itemize}

\subsection{New features in version \PASTPDACVERSION}

The numerical code \PDAC\ is based on previous numerical codes developed
by Dobran et al. (1993), Neri and Macedonio (1996) and Neri et al. (2003)
for the multiphase flow simulation of collapsing volcanic columns. We based
the release \PASTPDACVERSION\ on the Neri et al. (2003) model (considered 
here as version 1.0). If you are interested in knowing more about
the history of \PDAC\, consult the \PDAC\ web site at 
\PDACURL\ or contact the development team at 
\PDACADDRESS\

\subsubsection*{New modular structure}
The release \PASTPDACVERSION\ of \PDAC\ has been fully rewritten in 
Fortran90 by adopting a modular approach. Dynamic allocation of memory
allows the use of a precompiled \PDAC\ code. In addition the more readable 
source code will help the programmers to mantain and develop \PDAC\ more easily.
If you wish to implement a new feature in \PDAC\ you are encouraged to contact 
the developers team at \PDACADDRESS\ for guidance.

\subsubsection*{Parallel implementation and improved serial performance}
The parallel implementation of \PDAC\ includes load balancing sheme 
and ad-hoc communication routines that allow \PDAC\ to scale efficiently 
on many processors, thus reducing the execution time almost linearly.
The data distribution method can be chosen from input to improve the 
load-balancing and thus the scalability. All routines have been 
optimized in order to improve the efficiency of the memory access.
Optimized subroutines have been implemented in 3D for three phases and
first-order simulations, with considerable performance gain.

\subsubsection*{Portable on most architectures}
Any machine with a message passing library compatible with the MPI 
standard should be able to run \PDAC\ in parallel.
In recent years, distributed memory parallel computers have been offering
cost-effective computational power.  \PDAC\ was designed to run efficiently
on such parallel machines using large number of cells. 
\PDAC\ is particularly well suited also for the increasingly popular Beowulf-class PC 
clusters, which are quite similar to the workstation clusters for which is was 
originally designed.
Future versions of \PDAC\ will also make efficient use of clusters of 
multi-processor workstations or PCs.

\subsubsection*{Second-order spatial and third-order temporal discretizations}
Second-order spatial discretization schemes have been introduced in order
to improve the code accuracy and to reduce the so-called numerical 
dissipation which represents a limit in the simulation of turbulent flows.
High-order schemes are implemented as corrections of the first-order
scheme previously used. Few parameters have to be specified in the input
file to control numerical accuracy. Up to third-order low-storage 
Runge-Kutta temporal discretizations can be selected by the input file.

\subsubsection*{Dynamic Smagorinsky model}
Dynamic Smagorinsky sub-grid scale turbulence model has been implemented. 
This procedure allows to compute the Smagorinsky constant from the simulated
fluid flow (see Reference Manual for details).
An input flag allows the selection of the standard or dynamic model.

\subsubsection*{Improved control on convergence and errors}
More controls on the convergence and timing are added to reduce and
easily identify the source of convergence errors.

\subsubsection*{Possible fully implicit solution of model equations}
Fully implicit solution or implicit pressure coupling of the energy
equation can be selected by setting appropriate input parameters. This can
be useful in the evaluation of the accuracy of the numerical solution
but it comes out to be computationally very expensive.

\subsubsection*{Easy model settings}
Running inviscid simulation, or testing different turbulence models or closure 
relations can be easily controlled from the input files.

\subsubsection*{More friendly I/O}
A slef-explanatory Input file is now available. Pre-processing of OUTPUT files
is also introduced. Binary or ascii formatted output files can be selected.

\subsection{New features in version \PDACVERSION}

\subsubsection*{Extension to 3D}
Model equations have been rewritten in 3D Cartesian formulation. The
finite-volumes discretization schemes and the skeleton of the solution
algorithm have been mantained and the solution procedures have been extended
to 3D rectilinear, non-uniform, cartesian mesh.
New grid-decomposition schemes suited to 3D meshes have been devised, while
mantaining the low-level parallelization strategy used in 2D.
The old 2D Cartesian and cylindrical code has been nested into the 3D routines.

\subsubsection*{Parallel optimization}
Detailed benchmarks have been carried out on several (High Performance 
Computing (HPC) platforms to asses the efficiency of different 
domain-decomposition schemes with different mesh sizes. A low-level
profiling on the code and sub-routines performances has been carried out
to evaluate the bottle-necks in the run execution.
Optimized routines have been implemented for different parameter configurations.
The different levels of optimization can be selected from input.

\subsection*{Automatic grid generation}
New procedures for the definition of a non-uniform rectilinear mesh
have been implemented. Mesh parameters (cell size increase rate, domain
size, resolution, etc.) are set from input.

\subsubsection*{Integration of a 3D topography and UTM geo-referencing}
New routines for the integration of a real volcano topography from a 
standard ASCII Digital Elevation Model (DEM) have been integrated in the
setup procedures. These include interpolation, filtering, averaging
functions. The georeferencing of the computational mesh based on the 
UTM coordinates of the volcanic vent has been introduced.

\subsubsection*{New initial conditions at the vent}
Vent conditions can now be set from scratch with a simple procedure.
The vent is automatically located on the topography without the need
to specify the inlet cells.

\subsubsection*{New boundary conditions}
New boundary conditions for the free in-outflow boundaries
and for the topography have been implemented. 
Non-reflecting atmospheric boundary conditions have been set.
The topographic (no-slip) boundary
conditions have been introduced by adopting an extension to multiphase
compressible flows of the {\em immersed boundaries} technique.

\subsection{Added features in release \RELEASEVERSION}

Release \RELEASEVERSION fixed some minor bugs and re-organized
the module hierarchy. Some new boundary conditions have been
also implemented.

\subsubsection*{New initial conditions for dome explosions}
An automatic procedure to impose the initial conditions for the
explosion of a pressurized, confined volume of particles has been
implemented. This conditions can be representative of the initial
state of an exploding lava dome. The grain size distribution, 
the pressurization model, the dome porosity and permeability can
be easily set.

\subsubsection*{New automatic routines for the bluff-body problem}
An automatic procedure to tag the blunt blocks and to compute the
drag and lift forces have been implemented. This allow to compute,
for example, the action exerted by a pyroclastic flow on a building.

\subsection{User feedback}

If you have problems installing or running \PDAC\ after
reading this document, please send a
complete description of the problem by email to \PDACADDRESS.
If you discover and fix a problem not described in this manual we would
appreciate if you would tell us about this as well, so that we can alert
other users and incorporate the correction into the public distribution.
\prettypar
We are interested in making \PDAC\ more useful to the volcanological
community.  Your suggestions are welcome at \PDACADDRESS\ 
We also appreciate hearing from you about how you are using \PDAC\ 
in your work.

\subsection{Collaborations}
The formulation of the high-order numerical schemes, for both spatial
and temporal discretizations, has been developed in collaboration with
Maria Vittoria Salvetti (Dip.to di Ingegneria Aerospaziale, Universit\`a 
di Pisa) and Fran\c{c}ois Beux (Scuola Normale Superiore, Pisa). The 
immersed boundaries technique has been implemented in collaboration with
Mattia De'Michieli Vitturi (Dip.to di Matematica, Universit\`a di Pisa, now 
at INGV, Pisa). 
We also credit Maria Vittoria Salvetti for the formulation of the turbulence 
Large Eddy Simulation (LES) model and Sara Barsotti (INGV - Pisa) 
for the implementation.

\subsection{Acknowledgments}

The developement of the \PDAC\ release \PDACVERSION\ has been supported 
by European project EXPLORIS (EVR1-CT2002-40026), Gruppo Nazionale per
la Vulcanologia (Italy) and Ministero dell'Istruzione, dell'Universit\`a 
e della Ricerca (Italy).
\prettypar
The authors would also like to thank the High Performance Computing
staff of CINECA-Italy, for providing the computational resources and expertises
needed for developing and running \PDAC.
\prettypar
