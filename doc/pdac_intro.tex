\section{Introduction}
\label{section:intro}

\PDAC\ is a numerical code that has been developed to solve the multiphase
flow equations for the simulation of pyroclastic atmospheric dispersal dynamics.
It is designed for high performance computing on UNIX platforms.
This document describes how to use 
\PDAC, its features, and the platforms on which it runs.
The document is divided into several sections:
\begin{description}
\item[Section \ref{section:intro}] gives an overview of \PDAC.
\item[Section \ref{section:files}] describes \PDAC\ file formats.
\item[Section \ref{section:input_par}] lists basic simulation options.
\item[Section \ref{section:input_sample}] provides sample configuration files.
\item[Section \ref{section:run}] gives details on running \PDAC.
\item[Section \ref{section:avail}] gives details on installing \PDAC.
\end{description}

We have attempted to make this document complete and easy to understand and 
to make \PDAC\ itself easy to install and run.
We welcome your suggestions for improving the documentation or code
at \PDACADDRESS\

\subsection{\PDAC\ and pyroclastic flows simulations}

The \PDAC\ code solves the multiphase flow transport equations for
density, momentum and energy of a gas-pyroclast mixture in a steady
standard atmosphere.
The gas phase can be composed of several chemical components leaving
the crater - such as water vapor, carbon dioxide, etc. - and
atmospheric air (considered as a single chemical component).
The pyroclasts are described by $N$ phases of solid particles,
each one characterized by a diameter, density, specific
heat, thermal conductivity, and viscosity, and considered
representative of a granulometric class commonly present in
the eruptive mixture. Momentum and energy exchange between the gas
and the different particulate phases are expressed through semi-empirical
correlation equations.

\prettypar
\PDAC\ has several important features that make it a useful tool for the
simulation of volcanic columns and pyroclastic flows: 

\begin{itemize}

\item{\bf Robust over a wide range of flow regimes}
\begin{itemize}
\item from weak to strong gas-particle coupling;
\item from laminar to turbulent regimes;
\item from quasi-isothermal to highly buoyant flows;
\item from subsonic (incompressible) to supersonic (compressible) regimes.
\end{itemize}

\item{\bf Unlimited number of particles classes}\\
The number of particle classes used to approximate an observed distribution
of grain sizes is limited only by the computational resources available.
Up to 6 particulate phases have been used successfully. The size and density
of particles is nevertheless constrained by the multiphase flow assumptions,
that limit the size of solids up to few millimeters and down to about ten 
microns.

\item{\bf Up to 7 gas species}\\
Up to 7 gas species can be considered as components of the gas phase. The 
molecular composition allowed are: $O_2,N_2,CO_2,H_2,H_2O,Air,SO_2$.
Atmospheric air is considered as a single component with averaged properties.
Gas species are treated as tracers and no chemical reactions are allowed up to
now. Future versions of \PDAC\ will hopefully consider gas phase transitions
(such as water vapour condensation) or chemical reactions.

\item{\bf Atmospheric stratification}\\
A standard stratified atmosphere can be assigned as initial condition.
The temperature gradient within the different atmospheric layers are
the Earth averaged values, whereas pressure is computed by assuming hydrostatic
equilibrium. No vertical stratification of humidity is assumed. Cross wind
can be imposed in cartesian reference frames.

\item{\bf Control of simulation options}\\
The simple textual input file allow \PDAC\ to be started from prescribed
initial conditions or from binary restart file, representing the state
of the main flow fields when the code stops for any reason. When code
is started from scratch, vent inlet flow conditions the volcanic topography 
and atmospheric conditions are specified as input conditions. When the code
is restarted, the final state saved in restart file is used as initial 
conditions. The computational mesh and all numerical parameters are easily
specified in the input file.

\end{itemize}

\subsection{New features in version \PDACVERSION}

The numerical code \PDAC\ is based on previous numerical codes developed
by Dobran et al. (1993), Neri and Macedonio (1996) and Neri et al. (2003)
for the multiphase flow simulation of collapsing volcanic columns. We based
the present release \PDACVERSION\ on the most recent of these models 
(considered as version 1.0). If you are interested in knowing more about
the history of \PDAC\, consult the \PDAC\ web site at 
\PDACURL\ or contact the development team at 
\PDACADDRESS\

\subsubsection*{New modular structure}
The present release \PDACVERSION\ of \PDAC\ has been fully rewritten in 
Fortran90 by adopting a modular approach. Dynamic allocation of memory
allows the use of precompiled \PDAC\ code. The more readable source code
will help the programmers to mantain and develop \PDAC\ more easily.
If you are contemplating a particular modification to \PDAC\
you are encouraged to contact the developers team at \PDACADDRESS\
for guidance.

\subsubsection*{Parallel implementation and improved serial performance}
The parallel implementation of \PDAC\ includes load balancing sheme 
and ad-hoc communication routines that allow \PDAC\ to scale efficiently 
on many processors, thus reducing the execution time almost linearly.
Data distribution method can be chosen from input to improve the 
load-balancing and thus the scalability. All routines have been 
optimized in order to improve the efficiency of the memory access.
Optimized subroutines have been implemented for three phase, 3D and
first order simulation, with considerable performance gain.

\subsubsection*{Portable on most architectures}
Any machine with a message passing library compatible with the MPI 
standard should be able to run \PDAC\ in parallel.
In recent years, distributed memory parallel computers have been offering
cost-effective computational power.  \PDAC\ was designed to run efficiently
on such parallel machines for simulating large number of cells. 
\PDAC\ is particularly well suited to the increasingly popular Beowulf-class PC 
clusters, which are quite similar to the workstation clusters for which is was 
originally designed.
Future versions of \PDAC\ will also make efficient use of clusters of 
multi-processor workstations or PCs.

\subsubsection*{Second order spatial and third order temporal discretizations}
Second order spatial discretization schemes have been introduced in order
to improve the code accuracy and to reduct the so-called numerical 
dissipation, that represent a limit in the simulation of turbulent flows.
High order schemes are implemented as corrections to the first order
scheme previously used: few parameters must be specified in the input
file to control numerical accuracy. Up to third order low-storage 
Runge-Kutta temporal discretization can be selected from input.

\subsubsection*{Improved control on convergence and errors}
More controls on the convergence and timing are added to reduce or
easily identify the source of convergence errors.

\subsubsection*{Possible fully implicit solution of model equations}
Fully implicit solution or implicit pressure coupling of the energy
equation can be selected by setting few input parameters. This can
be useful in the evaluation of the accuracy of the numerical solution
but comes out to be very expensive computationally.

\subsubsection*{Easy model settings}
Running inviscid simulation, or testing different turbulence models or closure 
relations can be easily controlled from input files.

\subsubsection*{More friendly I/O}
More descriptive Input file is now available. Preprocessing of OUTPUT files
is also introduced. Binary or ascii formatted output files can be selected.

\subsection{User feedback}

If you have problems installing or running \PDAC\ after
reading this document, please send a
complete description of the problem by email to \PDACADDRESS.
If you discover and fix a problem not described in this manual we would
appreciate if you would tell us about this as well, so we can alert
other users and incorporate the fix into the public distribution.
\prettypar
We are interested in making \PDAC\ more useful to the volcanological
community.  Your suggestions are welcome at \PDACADDRESS\ 
We also appreciate hearing about how you are using \PDAC\ in your work.


\subsection{Acknowledgments}

The development of the \PDAC\ release \PDACVERSION\ has been supported 
by European project EXPLORIS (EUR1-CT2002-40026)

The authors would also like to thank the High Performance Computing
staff of CINECA-Italy, for the continuous support in making suggestion, pushing
for new features, testing bug-ridden code, and providing the computational 
resources for developing and running \PDAC.
\prettypar
