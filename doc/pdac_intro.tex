

\section{Introduction}
\label{section:intro}

\PDAC\ is a pyroclastic density analysis program for UNIX 
platforms designed for high-performance 
simulations in pyroclastic flows.  This document describes how to use 
\PDAC, its features, and the platforms on which it runs.
The document is divided into several sections:
\begin{description}
\item[Section \ref{section:intro}] gives an overview of \PDAC.
\item[Section \ref{section:start}] lists the basics for getting started.
\item[Section \ref{section:files}] describes \PDAC\ file formats.
\item[Section \ref{section:basic}] lists basic simulation options.
\item[Section \ref{section:add}] lists additional simulation options.
\item[Section \ref{section:sample}] provides sample configuration files.
\item[Section \ref{section:run}] gives details on running \PDAC.
\item[Section \ref{section:avail}] gives details on installing \PDAC.
\end{description}

We have attempted to make this document 
complete and easy to understand and to make \PDAC\ itself
easy to install and run.
We welcome your suggestions for improving the documentation or code
at {\tt pdac@..... }.

\subsection{New features in version \PDACVERSION}

\subsubsection*{Improved Parallel Scaling and Serial Performance}

Load balancing sheme and communication improvements that allow
\PDAC to scale to ... or more processors on ... are
included in this release. Any machine with a message passing
library compatible with the MPI standard should be able to run
\PDAC in parallel.
Optimized subroutines have been implemented for three phase, 3D and
first order simulation, with considerable performance gain.

\subsubsection*{Post Processing}

A new post processing command pp.x allows ....


\subsection{\PDAC\ and pyroclastic flows simulations}

Pyroclastic simulations compute scalar and vector fields of ... by solving
equations of ...... numerically. 

..... BREVE DESCRIZIONE DEL FUNZIONAMENTO ...

One way to run fine grain 3D simulation is to utilize parallel computers. 
In recent 
years, distributed memory parallel computers have been offering
cost-effective computational power.  \PDAC\ was designed to run efficiently
on such parallel machines for simulating large number of cells. 
\PDAC\ is particularly well suited to the increasingly popular Beowulf-class PC clusters, which are quite similar to the workstation clusters for which is was originally designed.
Future versions of \PDAC\ will also make efficient use of clusters of multi-processor workstations or PCs.
\prettypar
\PDAC\ has several important features: 

\begin{itemize}

\item{\bf PRIMA CARATTERISTICA}\\
bla bla bla bla
bla bla bla bla
bla bla bla bla
bla bla bla bla
bla bla bla bla

\item{\bf SECONDA CARATTERISTICA}\\
bla bla bla bla
bla bla bla bla
bla bla bla bla
bla bla bla bla

\item{\bf TERZA CARATTERISTICA}\\
bla bla bla bla
bla bla bla bla
bla bla bla bla
bla bla bla bla

\item{\bf Simulation Options}\\
MD simulations may be carried out using several options, including
\begin{itemize}
  \item Constant energy dynamics,
  \item Constant temperature dynamics via
  \begin{itemize}
    \item Velocity rescaling,
    \item Velocity reassignment,
    \item Langevin dynamics,
  \end{itemize}
  \item Periodic boundary conditions,
  \item Constant pressure dynamics via
  \begin{itemize}
    \item Berendsen pressure coupling,
    \item Nos\'{e}-Hoover Langevin piston,
  \end{itemize}
  \item Energy minimization,
  \item Fixed atoms,
  \item Rigid waters,
  \item Rigid bonds to hydrogen,
  \item Harmonic restraints,
  \item Spherical or cylindrical boundary restraints.
\end{itemize}

\item{\bf Easy to Modify and Extend}\\
Another primary design objective for \PDAC\ is extensibility and 
maintainability. In order to achieve this, it is designed in an 
modular style with F90. 
\PDAC's modular design allows one to integrate and test new algorithms 
easily.  If you are contemplating a particular modification to \PDAC\
you are encouraged to contact the developers at {\tt PDAC@..... }
for guidance.

\item{\bf Load Balancing}\\
An important factor in parallel applications is the equal distribution
of computational load among the processors. 
....
\PDAC\ addresses this problem by using a simple uniform 
spatial decomposition where the entire model is split into regular domains of 
space called {\em blocks}. An initial load balancer assigns blocks
to processors such that the computational load is balanced as much as possible.

\end{itemize}

\subsection{User feedback}

If you have problems installing or running \PDAC\ after
reading this document, please send a
complete description of the problem by email to {\tt PDAC@.... }.  If
you discover and fix a problem not described in this manual we would
appreciate if you would tell us about this as well, so we can alert
other users and incorporate the fix into the public distribution.
\prettypar
We are interested in making \PDAC\ more useful to the .....
community.  Your suggestions are welcome at {\tt PDAC@.... }.
We also appreciate hearing about how you are using \PDAC\ in your work.

\subsection{Acknowledgments}

This work is supported by grants from the European Community
(.....) and ....
\prettypar
The authors would particularly like to thank the members of the
.... Group, past and present, who have helped
tremendously in making suggestions, pushing for new features, and
testing bug-ridden code.
