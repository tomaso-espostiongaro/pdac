
\section{\PDAC\ Availability and Installation}
\label{section:avail}

\PDAC\ is distributed freely for non-profit use.
\PDAC\ \PDACVERSION\ is based on the MPI message passing interface
({\tt http://www.mpi-forum.org/})
and a proprietary ad-hoc communication layer
which have been ported to a wide variety of parallel platforms.
This section describes how to obtain and install \PDAC\ \PDACVERSION.

\subsection{How to obtain \PDAC}

\PDAC\ may be downloaded from {\tt http://www..../PDAC/}.
You will be required to provide minimal registration information and
agree to a license before receiving access to the software.
Both source and binary distributions are available.

\subsection{Platforms on which \PDAC\ will currently run}

\PDAC\ should be portable to any serial platform with a
Fortran90 and C compilers and to any parallel platform with
the above compilers and the MPI library.
Precompiled \PDAC\ \PDACVERSION\ binaries are available for the following platforms:  
\begin{itemize}
\item Linux on Intel
\item Mac OS X (also called Darwin) on PowerPC processors
\item AIX on RS/6000 processors 
\item HP-UX on PA-RISC processors 
\item Solaris on Sparc processors (with and without MPI)
\item Tru64 Unix on Alpha processors (with and without MPI)
\item IBM RS/6000 SP (using MPI)
\item Compaq AlphaServer SC (using the Quadrics Elan library)
\item SGI Origin 2000 (with and without MPI) 
\end{itemize}

\subsection{Compiling \PDAC}

We provide binaries for all
platforms to which \PDAC has been ported, but it is 
advisable to recompile the code on your system, 
since hardware is changing rapidly and most probably 
your native Fortran90 compiler will optimize
the code further. 
On Linux parallel platform with MPI, recompiling 
is in practice mandatory, since there are too
many possible combinations of MPI libraries and networking
hardware for which the executables are incompatible.
Recompiling is also required if you wish  
to add or modify features in \PDAC.

\subsubsection{ directory structure}

The \PDAC source code is distributed with a 
directory structure with two lavel, a single
main directory named EXPLORIS (in the official distribution), 
and several sub directories.
In what follow the different directories are listed
and their content commented.

\begin{itemize}

\item directory EXPLORIS \
      this is the main directory and contains the Fortran90 source files
      of \PDAC\ main subroutines. This directory contains also the 
      Makefile and the Machine file (see below).

\item directory EXPLORIS/comm \
      this subdirectory contains the \PDAC\ communication layer,
      interfaced to MPI, and few wrappers for system dependent
      features. In general all low level subroutines that 
      need preprocessing are in this directory.

\item directory EXPLORIS/doc \
      this directory contains \PDAC documentation and this manual.

\item directory EXPLORIS/examples \
      this directory contains several input files examples.
      
\item directory EXPLORIS/utility \
      this directory contains few utility scripts and program
      to help compiling and porting the code.

\end{itemize}

\subsubsection{ Machine file and Makefile}

\PDAC\ source code comes with a makefile (named Makefile in the distribution)
that take care of the compilation of \PDAC\ itself.
The Makefile file contains general commands that should be quite general
and platform independent. There are other Makefile files in the 
subdirectories to compile library, utility and documentations.
All Makefile files include a configuration file named Machine,
which contains all the system dependent parameters for the compilation,
like compiler name and compilations flags.
In the main directory there are several Machine files already prepared
for a number of existing architectures, then, before compiling anything
you should copy the Machine file (named Machine.arch, where arch vary 
on the name of the architectures where \PDAC\ has been ported) that match your 
architecture onto the Machine file (without any extensions), see next section
for more details. The Machine.arch files distributed with the source code
are configured and tested on the architecture "arch", although your system
could have some particular feature which require some change in the 
Machine.arch file, in this case you should edit the file to make the
proper changes. If your architecture is not included among those covered
by the preconfigured Machine.arch files, you should take the Machine.arch
file most close to your architecture and edit it to make the proper changes.
The Machine file should be edited also to choose which kind of executable
(serial or parallel) you want to produce (see comments in the Machine files)
The main Makefile file include also a file ( .dependencies ) containing
all the dependency among the source Fortran90 files, infact \PDAC\
source code is written using Fortran90 modules and source file
should be compiled in the proper order determined by the dependencies
among modules. The file .dependency is generated by a script ( shdep )
distributed togheter with the code (see next section for more details ).

\subsubsection{ makeing dependencies and compiling }

In this section the single steps required to compile the code
are listed and commented.

\begin{itemize}

\item Machine file \
      The very first thing you should do to compile \PDAC\ is to choose
      the Machine file for your architecture among those distributed with
      the source code and named Machine.arch, where the "arch" suffix
      vary on the list of the supported architectures. In exaple if
      your machine is an IBM sp4, you should copy the file Machine.sp4
      onto the Machine file. If there isn't the Machien.arch for 
      your architecture, choose the most close among those present
      and edit it to set the compiling parameters to the proper values.
      Once you have find your Machine file you should copy it into the
      file named Machine with the following command:\
      cp Machine.arch Machine\
      Then you can edit Machine to change compiling parameters,
      like the compiler options or the executable type (serial or parallel).
      For a serial executable you should unset the -D\_\_MPI precompiler macro
      subsituting it with -D\_\_SERIAL, and substitute the parallel compilers
      with the scalar one, changing the variables FC, MPIFC and LINKER
      in the Machine file.

\item Build the .dependencies file\
      Before compiling \PDAC\ executable you shold build the 
      dependencies file (.dependencies), to do this, you should
      move to the directory  EXPLORIS/utility and type make all,
      then supposing you are in the EXPLORIS directory, 
      the exact sequence of command is:\
      cd utility\
      make all\
      cd ../\

\item Compiling pdac.x and pp.x\
      at this point everything is ready to compile the \PDAC\ executable
      ( pdac.x ) and the post processing executable ( pp.x ), from
      the EXPLORIS directory type:\
      make pdac.x\
      make pp.x\

\item Compiling this manual\
      to compile the manual you should have installed a tex packages
      with standard latex extensions support, then type the following commands:\
      cd doc\
      latex pdac\_ug.tex\
      latex pdac\_ug.tex\
      dvips -f < pdac\_ug.dvi > pdac\_ug.ps

\end{itemize}

\subsection{Documentation}

All available \PDAC\ documentation is available for download without
registration via the \PDAC\ web site
{\tt http://www...../PDAC/}.

