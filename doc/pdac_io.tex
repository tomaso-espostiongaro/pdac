\section{Input and Output Files}
\label{section:files}

This section describes all files that are red or written by \PDAC.
Release \PDACVERSION\ does not have a post-processing tool integrated,
so that all interaction you have with the program is through textual files.
The I/O files are always red and written in the running directory,
i.e. the directory from which \PDAC\ has been run.
The level of verbosity, controlling the amount of information written
by the program during the execution, can be set before starting a run.
Note that a too high level of verbosity for very long-lasting simulations
could stress the file-system causing the program to stop.

\subsection{\PDAC\ input file formats}
\label{section:input_files}

\PDAC\ has two possible input files from which it gets the initial
conditions to start, but only one of these is strictly required to run \PDAC.
A textual parameters file ``pdac.dat'' should be compiled by the user. 
It contains the namelists and cards with the parameters of the simulation
and the various option controlling the program execution.
\PDAC\ will stop if this file is not present in the directory where the 
executable is run.\\

The other Input file that can be required by the program is the {\em restart}
file ``pdac.res''. This is a binary file written by the code itself and
contains the dump of all independent fields that are saved when a
simulation run is interrupted for any reason. Therefore it cannot be edited by 
the user. Depending on the size of the problem envisaged, the restart file can 
be very large, so that you must ensure that your file-system can manage it.

In a parallel execution all input files are opened, red, and closed
by the root processor (the one with MPI id equal to 0), that
is charged of broadcasting data to the other computing units.

\subsubsection{pdac.dat file}
\label{section:padc_dat}

This file contains the parameters that define the physical system
you want to simulate with \PDAC. ``pdac.dat'' is not red from standard 
input but is connected explicitly to the fortran {\bf unit 5}. The file should
be present in the running directory of \PDAC\, and it is always red
at the beginning of the simulation run even if the run is a 
continuation/restart of a previous simulation run. 
The parameters and options in the ``pdac.dat'' file are specified
in a set of F90 {\em namelists} followed by {\em cards} for topography, 
initial and boundary conditions.
The options and values specified determine the exact behavior of
\PDAC, what features are active or inactive, how long the simulation
should last, etc.  
Section \ref{section:input_par} lists the parameters required to run a 
simulation, their default values and the range of variability.
Several sample \PDAC\ parameter files are shown in section 
\ref{section:input_sample}.\\

The layout of pdac.dat file is as follow:

\begin{verbatim}

&control
  control_parameters1 = control_value1,
  control_parameters2 = control_value2,
  ...
/

&model
  model_parameters1 = model_value1,
  model_parameters2 = model_value2,
  ...
/

&mesh
  mesh_parameters1 = mesh_value1,
  mesh_parameters2 = mesh_value2,
  ...
/

&particles
  particles_parameters1 = particles_value1,
  particles_parameters2 = particles_value2,
  ...
/

&numeric
  numeric_parameters1 = numeric_value1,
  numeric_parameters2 = numeric_value2,
  ...
/

'ROUGHNESS'
roughness_parameters

'MESH'
nonuniform_mesh_distances

'FIXED_FLOWS'
fixed_flows_parameters

'INITIAL_CONDITIONS'
initial_conditions_parameters

\end{verbatim}

Within a namelist, the parameters are specified as
\begin{verbatim}
keyword      =     value
\end{verbatim}
Blank lines in the namelists are ignored.  Comments are prefaced by
a {\tt !} and may appear after the keyword assignment:
\begin{verbatim}
keyword  = value          !  This is a comment
\end{verbatim}
or may be at the beginning of a line:
\begin{verbatim}
! keyword  = value   This is a keyword commented out
\end{verbatim}
Some keywords require several values, that are specified as an array of values
separated by a comma:
\begin{verbatim}
keyword  = value_1, value_2, value_3,
\end{verbatim}

Note that the namelists ( \&control, \&model, \&mesh, \&particles and \&numeric )
and the cards ( 'ROUGHNESS', 'MESH', 'FIXED\_FLOWS' and 'INITIAL\_CONDITIONS' )
should appear in exactly this order. 
Parameters lines in the namelists are optional, if a parameter is omitted
it will take a default values (see \ref{section:input_par} ).
On the contrary parameter lines in the cards are not optional and
should always be present.
Commet are allowed in the namelist using the "!" character to identify 
that wath follow is a comment. Commet are not allowed within the cards
parameter, but you could insert free text (not including cards and namelists names)
between different namelists and cards.
Parameters within cards are in general distributed on more lines depending
on the simulation parameters. The layout of the input parameters 
within each cards is described in details in section \ref{section:input_par}.

\subsubsection{pdac.res file}
\label{section:pdac_res}

\PDAC\ dumps all fields and parameters that are required to restart a 
simulation from exactly the same instant on the file ``pdac.res'',
as if no interruption occurred. The file is in double precision 
binary format to maintain the numerical accuracy of the internal 
representation of numbers and to save space. 
This file is not meant for post processing
and analysis of the fields, since its internal layout could
vary easily from one version of the code to the other.
The restart file is connected to the fortran {\bf unit 9}, it is
opened in {\it READ} mode at the beginning of a restart simulation run,
to read in the status of a previous simulation to be continued.
Then pdac.res file is opened in {\it WRITE} mode at regular interval
of simulated time. The rate at which pdac.res is overwritten
with a new dump of the simulation status is specified by
the user through the control parameter "tdump" 
(see \ref{section:input_par} ).
All the appropriate measures should be taken in order
not to lose this file. We suggest to make a copy of the file in 
a safe device (like tapes) before every run, since any error occurring
when the file is opened in write mode would make you losing the possibility
to restart the simulation.

\subsection{\PDAC\ output file formats}
\label{section:output_files}

\PDAC\ writes several output files, that can be subdivided into 
{\em information} and {\em data} files. 
Information files (``pdac.log'', ``pdac.tst'' and
``pdac.err'' ) contain logging, testing and debugging information
meant for the user to control the progress of the simulations.
Data files (``OUTPUT.nnnn'') contain the values of all the 
independent physical fields saved at regular interval of time, in different 
files distinguished by an incremental index (the ``nnnn'' in the file name).
Obviously the restart file described above can be considered also as an 
output data file.

Data output files are opened,
written and closed by the root processor (that having MPI rank 0), 
that gathers data from other computing units before writing. \\
All other information output files 
are opened, written and closed by all processors, and can be distinguished
by an extension indicating the rank of the process. 

\subsubsection{Log file}
\label{section:padc_log}

The file pdac.log is used by \PDAC\ to log information
concerning the advancing of the simulations. pdac.log is 
connected explicitly to the fortran {\bf unit 6}. 
pdac.log is a free text file with self-explanatory informations.
Note that \PDAC\ does not use standard output stream, but on many
system the unit 6 is associated by default to the standard output, therefore
on these systems pdac.log is actually a redirected standard output stream.
It is important also to underline that when a new pdac.log is opened 
\PDAC\ cancels any existing pdac.log file.

\subsubsection{Test file}
\label{section:padc_tst}

The file pdac.res, like pdac.log, is a free text file, and contains debugging
and testing informations, that are meant for developers only,
therefore many informations present there are of no use for the user.
The file is connected to the fortran {\bf unit 7}, and is cleared every new run.

\subsubsection{Error file}
\label{section:padc_err}

Like pdac.log and pdac.tst, pdac.err is a free textual file,
it is connected to the fortran {\bf unit 8}. In this file \PDAC\ writes
all messages related to severe errors that cause the program to stop.
Note that this file is not cleared at every run, but it is opened in 
{\it APPEND} mode.

\subsubsection{OUTPUT file}

All scalar and vector fields related to the evolution of the simulated
system are saved into the OUTPUT.nnnn files at regular interval of
simulated time. The user can choose the interval of time through
the control parameter "tpr" (see \ref{section:input_par} ).
The extension ``nnnn'' is a progressive number, ranging from 0000 to 9999, 
incremented each time the fields are written. 
The OUTPUT files are associated to the fortran {\bf unit 12}. 
The format of these file can be selected by the user using
The logical control parameter ``formatted\_output'' controls if the
output will be written in formatted textual format
or in binary (single precision) format. In general the OUTPUT files 
are meant for postprocessing analisys, it is therefore useful to know 
their layout. In both formatted and binary forms, the fields are saved
in this order:

\begin{verbatim}

time            ! Simulated time

p               ! Thermodynamic pressure
ug              ! Gas velocity along x
vg              ! Gas velocity along y (only for 3D simulations)
wg              ! Gas velocity along z
tg              ! Gas temperature

xgc             ! Molar fraction of gas components
...             ! (for all gas components)

eps             ! Volumetric fraction of particles
us              ! Particle velocity along x
vs              ! Particle velocity along y (only for 3D simulations)
ws              ! Particle velocity along z
ts              ! Particle temperature
...             ! (for all solid phases)

\end{verbatim}

where the first line is a scalar real value representing the 
simulated time at which the fields are written, the oter lines
represent the fields (of dimension {\tt nx*ny*nz}) that are written 
using the fortran format specifier {\tt FORMAT( 5(G14.6E3,1X) )}.
The components of the gas and solid velocities in the y dimension
({\tt vg} and {\tt vs}) are present in the OUTPUT file only if the simulation
type is 3D.

When the file is written in the binary unformatted format,
the time and each variable is written in a separate record.
In particular "time" is written in a record of lenght 1
while all other fields in records of lenght {\tt nx*ny*nz}.

\subsection{Post processing files}
\label{sect:pp}

\PDAC\ release \PDACVERSION\ provides the \verb#pp.x# utility,
which is intended to be a preprocessing tool for visualization of
output files produced by \PDAC. Even if this tool is still under 
development, its usage is shortly described in order to make it
useful for the user, especially when the OUTPUT of very large simulations
are too big to be processed directly by a visualization tool.
pp.x extracts from a series of OUTPUT.nnnn files the individual single fields,
with the possibility of {\it downsize}, interpolate and crop the data,
and store them in separate files that can be easily plotted with
varius graphical application.

\subsubsection{Post processing input files}

The preprocessing tool must read the input file pdac.dat, to which
an additional namelist with the parameters that control the post processing
is added. The name of the additional namelist is \&pp , and
should be typed in between the namelist \&numeric and the first card
'ROUGHNESS' . The layout of the pdac.dat file for the post processing
is then:

\begin{verbatim}
...
...

&numeric
  numeric_parameters1 = numeric_value1, 
  numeric_parameters2 = numeric_value2,
  ...
/

&pp
  pp_parameter1 = pp_value1,
  pp_parameter2 = pp_value2,
  ...
/

'ROUGHNESS'
roughness_parameters

...
...
\end{verbatim}

Note that you can leave the \&pp namelist in the pdac.dat file
for the usage of main program too, since it will be simply ignored.

\subsubsection{Post processing output files}

The post processing produces several output files, 
containing all individual fields and grid information useful
for visualization.
In particular post processing (pp.x) read a sequence
of OUTPUT.nnnn files (the numbers indicating the first, the last, and
the increment of ``nnnn'' are red in input)
and write out a sequence of files for each individual field.
The name of new files are of the form ``filter.FIELDNAME.nnnn''.
The format of these file is the same of OUTPUT, as specified above.
A list of all file produced by the post processing is following:

\begin{itemize}

\item pdac.xml  \\
      this file contains all the information contained in pdac.dat
      written using an XML compliant format

\item pdac.grx \\
      contains the list of cell centers (excluding border cells) 
      in x direction

\item pdac.gry \\
      contains the list of cell centers (excluding border cells) 
      in y direction

\item pdac.grz \\
      contains the list of cell centers (excluding border cells) 
      in z direction

\item filter.pgas.nnnn \\
      these files contain the gas pressure field 

\item filter.ug.nnnn \\
      these files contain the gas velocity field in x direction
      for the gas phase

\item filter.vg.nnnn \\
      these files contain the gas velocity field in y direction
      for the gas phase
      This file is not present in 2D simulations

\item filter.wg.nnnn \\
      these files contain the gas velocity field in z direction
for the gas phase

\item filter.tgas.nnnn \\
      these files contain the gas temperature field 

\item filter.xgGG.nnnn \\
      these files contain the molar fraction fields for all the gas 
      components ( GG runs from 01 to ngas ) 

\item filter.epSS.nnnn \\
      these files contain the gas ... field for all the gas 
      phases ( SS runs from 01 to nsolid ) 

\item filter.tsSS.nnnn \\
      these files contain the particles temperature field 
      for all solid phases ( SS run from 01 to nsolid )

\item filter.usSS.nnnn \\
      these files contain the particles velocity field in x direction
      for all solid phases ( SS run from 01 to nsolid )

\item filter.vsSS.nnnn \\
      these files contain the particles velocity field in y direction
      for all solid phases ( SS run from 01 to nsolid )
      These files are not present in 2D simulations

\item filter.wsSS.nnnn \\
      these files contain the particles velocity field in z direction
      for all solid phases ( SS run from 01 to nsolid )


\end{itemize}

Note that when running the post processing all 
OUTPUT.nnnn files in the selected range should 
be present in the running directory of "pp.x"

\subsection{General remark}

\noindent Caveat:

\noindent 1. ..... MAXIMUM DIMENSION

\noindent 2. ..... DISK SPACE

\noindent 3. ..... AVOIDING DATA LOSS
