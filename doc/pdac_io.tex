\section{Input and Output Files}
\label{section:files}

This section describes all the files that are red or written by \PDAC.
Release \PDACVERSION\ does not have a post-processing tool integrated,
so that all interaction you have with the program is through textual files.
The I/O files are always red and written in the running directory,
i.e. the directory from which \PDAC\ has been run.
In a parallel execution all input files are opened, red, and closed
by the root processor (the one with MPI id equal to 0), that
is charged of broadcasting data to the other computing units.
The level of verbosity, controlling the amount of information written
by the program during the execution, can be set before starting a run
(note that a too high level of verbosity for very long-lasting 
simulations could cause the program to stop for file-system problems).
Input and Output files, and the related Fortran units, are defined in the 
module io\_files, in the file \FIL{io\_files.f}.

\subsection{Input parameters file}
\subsubsection{The \FIL{pdac.dat} file}

The user can set the \PDAC\ simulation parameters
by editing the main input file \FIL{pdac.dat}.
The code stops if this file is not present in the 
directory where the executable is running, even if the run is a 
continuation/restart of a previous simulation run.
\FIL{pdac.dat} is not red from standard 
input but is connected explicitly to the Fortran unit 5.\\

The parameters and options in the \FIL{pdac.dat} file are specified
in a set of F90 namelists followed by ``cards'', that are added to
guarantee compatibility with old \PDAC\ input files. 
The layout of \FIL{pdac.dat} file is as follows:

\begin{verbatim}
&namelist1
  control_parameters1 = control_value1,
  control_parameters2 = control_value2,
  ...
/

&namelist2
  control_parameters1 = control_value1,
  control_parameters2 = control_value2,
  ...
/
  ...

'CARD1'
card1_parameters

'CARD2'
card2_parameters

  ...

\end{verbatim}

Within a namelist, the parameters are specified as
\begin{verbatim}
keyword      =     value
\end{verbatim}
Parameters lines in the namelists are optional, if a parameter is omitted
it will take its default values.
Blank lines in the namelists are ignored. Comments are prefaced by
a {\tt !} and may appear after the keyword assignment:
\begin{verbatim}
keyword  = value          !  This is a comment
\end{verbatim}
or may be at the beginning of a line:
\begin{verbatim}
! keyword  = value   This is a keyword commented out
\end{verbatim}
Some keywords require several values, that are specified as an array of values
separated by a comma:
\begin{verbatim}
keyword  = value_1, value_2, value_3,
\end{verbatim}

On the contrary, parameter lines in the cards are not optional and
should always be present. Comments are not allowed within the cards,
but you can insert free text (not including cards and namelists 
names) between namelists and cards.
The namelists and the cards should appear in exactly the order
specified. 

Sect. \ref{section:input_par} lists the parameters required to run a 
simulation, their default values and the range of variability.
Several sample \PDAC\ parameter files are shown in Sect. 
\ref{section:input_sample}.\\

\subsubsection{The topography file}
When the \OBJ{itp} parameter in the 
{\tt \&topography} namelist is set to 1, \PDAC\
seeks for a DEM file, whose name must be specified by
the parameter \OBJ{dem\_file} in the same namelist.
The DEM file must be formatted (ASCII) and must contains
a header and the dataset.\\
In 2D, the Digital Elevation Model (DEM) header consists of a single line
containing the integer number of points in {\it x} (or {\it r}) direction where
the profile is defined. The points can be not equally spaced.\\
The dataset contains the couples $(xtop,ztop)$, one per each line, in meters
(in double precision).\\
In 3D, the Digital Elevation Model (DEM) header consists of 6 lines containing,
the following items:
\begin{itemize}
\item the number of nodes in {\it x} direction
\item  the number of nodes in {\it y} direction
\item  the Cartesian x-coordinate (such as the UTM geographic coordinates) of the upper-left point of the matrix
\item  the Cartesian y-coordinate of the upper-left point of the matrix
\item  the resolution of the DEM in metres (the DEM must be equally spaced, with the same incremente in both directions)
\item an integer value defining missing data (usually -9999)
\end{itemize}
The dataset contains the topography elevations, one per each line, in
centimeters (without decimals), spanning the elevation matrix
from the upper-left to the lower-right corner, from the upper to the lower
row.

\subsubsection{The radial inlet profile file}
When the \OBJ{ipro} parameter in the 
{\tt \&inlet} namelist is set to 1, \PDAC\
seeks for a file, whose name must be specified by
the parameter \OBJ{rad\_file} in the same namelist.
This file must be formatted (ASCII) and must contains,
on the first line, the number of points defining the profile in the inlet.
On the following lines, the user must report the arrays, each on a new
lines, of the flow variables along the vent radial profile.
Array elements should be separated by comma or spaces, and follow this order:
\begin{itemize}
\item point locations
\item radial gas velocities
\item vertical gas velocities
\item gas temperatures
\item pressures
\item gas component mass fractions (one line per each species)
\item radial particle velocities
\item vertical particle velocities
\item particle temperatures
\item particle volume fractions
\end{itemize}
For every particle class, the last 4 lines must be repeated.

\subsection{Output log files}

\PDAC\ writes several log files where significant information
concerning the simulation conditions are saved.
Log files are opened,
written and closed by the root processor (the one having MPI rank equal to 0), 
that gathers data from the other computing units before writing.

\subsubsection{The \FIL{atmo.log} file}
This file is always written during the simulation setup.
It reports the result of the computation of the atmospheric
vertical temperature and pressure profile. In particular, 
for each layer, it reports the temperature gradient, the maximum vertical
extent, the pressure at the base and top, the temperature at the base and top.
Moreover, for every point of the vertical discretization, it lists the 
cell number, its quota, pressure and temperature.

\subsubsection{The \FIL{data.log} file}
The file \FIL{data.log} reports all input parameters, including
those not explicitly set through the input file, for which the
default value is set.

\subsubsection{The \FIL{dome.log} file}
This file is written when the \OBJ{idome} parameter in the 
{\tt \&dome} namelist is set to 1.
It reports the initial conditions for dome explosion. In details,
it contains the mesh indices and coordinates of the dome center and its radius,
 and the number, mesh indices, distance from the dome center, and angle of every
dome cell. Information on the radial pressure profile along the
vertical axis of the dome is also reported.
Finally, the total mass and energy stored in the dome is computed and 
written on this file.

\subsubsection{The \FIL{pdac.log} file}
This file is always written during the execution. It is the last file
written by the code before stopping.
The file \FIL{pdac.log} is used by \PDAC\ to log information
concerning the execution parameters and the progress of the simulation. 
In detail, it contains the starting date of the run, the number of
processors used for the simulation, the mesh size and the information
on the mesh decomposition for parallel execution. It then reports
the initial time (the restart time if this is the case) and the
number of iteration needed to reach convergence at every time step.
Finally, it reports the execution time for the main program blocks
and the total execution time.

\FIL{pdac.log} is connected explicitly to the Fortran unit 6. 
\FIL{pdac.log} is a free text file with self-explanatory information.
Note that \PDAC\ does not use standard output stream but on many
systems the unit 6 is associated by default to the standard output. 
Therefore, on these systems, \FIL{pdac.log} is actually a redirected 
standard output stream.
It is important also to underline that when a new \FIL{pdac.log} is opened 
\PDAC\ cancels any existing \FIL{pdac.log} file.

\subsubsection{The \FIL{topo.log} file}
This file is written when the \OBJ{itp} parameter in the 
{\tt \&topography} namelist is set to 1. It contains information
about the limits, resolution, and coordinates of the topographic
profile imported from a Digital Elevation Model (DEM). 

\subsubsection{The \FIL{vent.log} file}
This file is written when the \OBJ{ivent} parameter in the
{\tt \&inlet} namelist is set to 1. It contains information
about the mesh indices and coordinates of the vent center,
its radius, the radius of the topographic area modified to
insert the inlet conditions, the number of vent cells and,
for every vent cell, its mesh coordinates, the fraction of
the cell area occupied by the vent and the factor used to
set a radial profile.
Finally, the mass flow rate, gas density, solid density, 
mixture velocity and the equivalent radius are computed and 
reported in this file. 

\subsection{Output debug information}

\subsubsection{The \FIL{pdac.tst*} files}

The file \FIL{pdac.tst} is a free text file that contains debugging
and testing informations. This is meant for developers only,
therefore the information present in it is not particularly useful to the 
user.
The file is connected to the Fortran unit 7 and is cleared at every new run.
When a simulation is run on {\em nproc} processors on a parallel computer, 
each process writes its own test file, named \FIL{pdac.tstnnn}, where
the extension ``nnn'' is a progressive number, ranging from 1 to {\em nproc - 1}
indicating the rank of the process. 
Test files contain information concerning the communication buffers used
for interprocessor data exchanging and, for every time step, a report
on the convergence within the SOR loop. 
If the code, for any reason, does not converge,
flow conditions in the non-converging cells are dumped in the
corresponding test file.
The input flag \OBJ{lpr} controls the information written into the test files. 
Please notice that opening hundreds
of test files can cause a system failure. We recommend that, for {\em nproc}
larger than some tens, the level of verbosity is set to 0.

\subsubsection{The \FIL{pdac.err} file}

\FIL{pdac.err} is a free textual file where \PDAC\ writes all messages 
related to severe errors. Note that this file is not cleared at every 
run but it is opened in {\it APPEND} mode.
It is connected to the Fortran unit 8. 

\subsection{Output data files: \FIL{output.nnnn}}

All scalar and vector fields related to the evolution of the simulated
system are saved into the \FIL{output.nnnn} files at regular intervals of
simulated time. The user can choose the interval of time through
the control parameter \OBJ{tpr} (see Sect. \ref{section:input_par} ).
The extension ``nnnn'' is a progressive number, ranging from 0000 to 9999, 
incremented each time the fields are written. 
The output files are associated to the Fortran unit 12. 
The format of these file can be selected by the user.
The logical control parameter \OBJ{formatted\_output} controls if the
output will be written in formatted textual format
or in binary (single precision) format. In general, the output files 
are meant for post-processing analysis, it is therefore useful to know 
their layout. In both formatted and binary forms, the fields are saved
in this order:

\begin{verbatim}

time            ! Simulated time

p               ! Thermodynamic pressure
ug              ! Gas velocity along x
vg              ! Gas velocity along y (only for 3D simulations)
wg              ! Gas velocity along z
tg              ! Gas temperature

xgc             ! Molar fraction of gas components
...             ! (for all gas components)

eps(s)          ! Volumetric fraction of particle class (s)
us(s)           ! Particle velocity along x
vs(s)           ! Particle velocity along y (only for 3D simulations)
ws(s)           ! Particle velocity along z
ts(s)           ! Particle temperature
...             ! (for all solid phases)

\end{verbatim}

where the first line is a scalar real value representing the 
simulated time at which the fields are written and the other lines
represent the fields (of dimension {\tt nx*ny*nz}) that are written 
using the Fortran format specifier {\tt FORMAT( 5(G14.6E3,1X) )}.
The components of the gas and solid velocities in the y dimension
({\tt vg} and {\tt vs}) are present in the output file only if the simulation
type is 3D.

When the file is written in the binary unformatted format,
time and each variable is written in a separate record.
In particular time is written in a record of lenght 1
while all other fields in records of lenght {\tt nx*ny*nz}.

\subsection{Other output files}
\begin{itemize}
\item \FIL{mesh.dat} contains the georeferenced mesh points
corresponding to the centered and staggered cell locations.
\item \FIL{body.dat} contains the drag and
lift forces exterted by the flow on a bluff body, and is generated 
if the flag \OBJ{ibl} is set to 1.
\item \FIL{pdac.chm} contains information on mass residuals and is written when
the flag \OBJ{imr} is set to 1.
\item \FIL{improfile.dat} contains, for each cell in the computational
domain, its vertical distance from the topography. It is written
only if a DEM file is used to acquire the topography (flag \OBJ{itp} set
to 1).
\item \FIL{vf.dat} contains, for each cell in the computational domain,
the fraction of volume (from 0.0 to 1.0) that is above the topography.
This information is useful when the {\em immersed boundary} technique is
used to represent the topography, i.e., when the flag \OBJ{immb} is set to 1. 
\item \FIL{fptx.dat}, \FIL{fpty.dat}, \FIL{fptz.dat}, contain information on
all forcing points in the three coordinate directions when the 
{\em immersed boundary} technique is used to represent the topography, 
i.e., when the flag \OBJ{immb} is set to 1.
\item \FIL{export\_topography.dat} contains the 2D matrix of the elevations
of the topographic profile interpolated over the centers of the XY mesh.
\end{itemize}

\subsection{Restart files}

When a simulation is started ``from scratch'', 
initial conditions are set from the input file \FIL{pdac.dat}.\\

Nevertheless, in many situations, it is useful to restart a simulation
from a previous stop. \PDAC\ allows restarting from a specific restart
file \FIL{pdac.res} or from standard output files \FIL{output.nnnn}.
In the former case, the double precision of the results should be maintained
(this means that the results after restarting are identical to those
obtained if the simulation hadn't been stopped). In the latter case,
since the output files are written in 4 bytes format, the double
precision is not guaranteed.

\subsubsection{The \FIL{pdac.res} file}

The double precision restart is selected by setting the 
parameter \OBJ{restart\_mode} to ``restart`` in the namelist {\tt \&control}
(see Sect. \ref{section:input_par} ). \PDAC\ stops if this 
restart mode is selected but the file \FIL{pdac.res} is not present 
in the directory where the executable is running.

The restart file is written by the code and
contains the dump of all independent fields which are saved when a
simulation run is interrupted for any reason. Therefore it is not 
usually edited by the user. 
The file is in double precision unformatted format 
to maintain the numerical 
accuracy of the internal representation of numbers and to save disk space. 
Depending on the size of the problem envisaged, the restart file can 
therefore be very large, so that you must ensure that your file-system 
can manage it.

The restart file is connected to the Fortran unit 10, it is
opened in read-mode at the beginning of a restart simulation run,
to read in the status of a previous simulation interrupted.
During a simulation, the \FIL{pdac.res} file is opened in 
write-mode and overwritten at regular intervals
of simulated time (to allow a restart in case of unexpected stop or crash). 
The rate at which \FIL{pdac.res} is overwritten
with a new dump of the simulation status is specified by
the user through the control parameter \OBJ{tdump} 
(see Sect. \ref{section:input_par} ).
All the appropriate measures should be taken in order
not to lose this file. We suggest to make a copy of the file in 
a safe device (such as tapes) before every run, since any error occurring
when the file is opened in write mode could make you losing it.

\subsubsection{Restart from output files}
The single precision restart is selected by setting the 
\OBJ{restart\_mode} parameter to ``outp\_recover`` 
in the namelist {\tt \&control}
and by specifying the output number through the parameter \OBJ{nfil}.
Since the output file can be either formatted (ASCII format) or not
(unformatted binary), the user is required to specify the output
type throught the parameter \OBJ{formatted\_input} in the  namelist 
{\tt \&control} (see Sect. \ref{section:input_par}).
\PDAC\ stops if this 
restart mode is selected but the corresponding output file is not present 
in the directory where the executable is running.
The initial time in the input file does not need to be set
since it is read from the output when restarting.

