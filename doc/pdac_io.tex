
\section{Input and Output Files}
\label{section:files}

The \PDAC\  code in a simulation run opens several I/O files.
There are two input files, a textual simulation parameters file 
"pdac.dat" and a binary restart file "pdac.res". In particular
"pdac.dat" should be compiled by the user and contains the
namelists and cards with the parameters of the simulation
like the time step and mesh size, "pdac.res" is written
and read by \PDAC\ itself and contain the status of all
fields and variables required to restart a simulation run
interrupted for some reasons. The size of "pdac.dat" is
few hundreds of bytes, while "pdac.res" could be very large,
depending on the size of the simulated system. 

\PDAC\ writes several output files, there are information
files and data files. Information files ("pdac.log", "pdac.tst" and
"pdac.err" ) contains logging, testing and debugging information
meant for the user to control the progress of the simulations,
like the number of step performed so far or the number of iteration
each step takes to converge. Data files ("OUTPUT.nnnn") contains the values of the 
different fields saved at regular interval of time, in different 
files distinguished by an auto incremented index ( the "nnnn" in the file name ).

NOTE that the file "pdac.res" is both of input and output, 
it is opened in read mode at the beginning of a continuation run
and it opened in write mode at regular interval to save the status
of the simulation. We suggest to save a copy of "pdac.rea" before
starting a run, in this way you will not risk to lose the restart
information, caused by a system problem when the file is opened
in write mode or caused by an error in the restart configuration file.

The I/O files are always read and written in the running directory,
i.e. the directory from which \PDAC\ has been run.
In a parallel execution all files are opened, read, written and closed
by the root processor (the one with MPI id equal to 0),
root processor gather data from other processor before writing
and scatter data to other processor after having read from file.
All other processors opens only its own version of pdac.log, pdac.tst
and pdac.err with the names pdac.logNNN, pdac.tstNNN and pdac.errNNN,
where NNN is the MPI rank of the processor that open the file.
 
\subsection{\PDAC\ file formats}
\label{section:formats}

\subsubsection{pdac.dat file}
\label{section:padc_dat}

This file contains the parameters that define the physical system
you want to simulate with \PDAC\. "pdac.dac" is not read from standard 
input but is connected explicitly to the fortran unit 5. The file should
be present in the running directory of \PDAC\, and it is always read
at the beginning of the simulation run even if the run is a 
continuation/restart of a previous simulation run. 
The parameters and options in the "pdac.dat" file are specified
in a set of F90 namelists followed by cards for topology, initial and boundary
conditions.
The options and values specified determine the exact behavior of
\PDAC, what features are active or inactive, how long the simulation
should continue, etc.  Section \ref{section:configsyntax} describes how
options are specified within a \PDAC\ configuration file.  Section
\ref{section:requiredparams} lists the parameters which are required
to run a basic simulation.
Several sample \PDAC\ parameter files are shown in section \ref{section:sample}.
The layout of pdac.dat file is as follow:

\begin{verbatim}

&control
  control_parameters1 = control_value1,
  control_parameters2 = control_value2,
  ...
/

&mesh
  mesh_parameters1 = mesh_value1,
  mesh_parameters2 = mesh_value2,
  ...
/

&particles
  particles_parameters1 = particles_value1,
  particles_parameters2 = particles_value2,
  ...
/

&numeric
  numeric_parameters1 = numeric_value1,
  numeric_parameters2 = numeric_value2,
  ...
/

'ROUGHNESS'
roughness_parameters

'MESH'
nonuniform_mesh_distances

'FIXED_FLOWS'
fixed_flows_parameters

'INITIAL_CONDITIONS'
initial_conditions_parameters

\end{verbatim}

Note that the namelists ( \&control, \&mesh, \&particles and \&numeric )
and the cards ( 'ROUGHNESS', 'MESH', 'FIXED\_FLOWS' and 'INITIAL\_CONDIDION' )
should appear in exactly this order. 
Parameters lines in the namelists are optional, if a parameter is omitted
it will take a default values (see \ref{section:input_parameters} ).
On the contrary parameter lines in the cards are not optional and
should always be present.
Commet are allowed in the namelist using the "!" character to identify 
that wath follow is a comment. Commet are not allowed within the cards
parameter, but you could insert free text (not including cards and namelists names)
between different namelists and cards.
Parameters within cards are in general distributed on more lines depending
on the simulation parameters (i.e. 2D or 3D symulation type), parameter layout 
within each cards are described in details in section \ref{section:input_parameters}.

\subsubsection{pdac.log file}
\label{section:padc_log}

The file pdac.log as the extension suggest is used by \PDAC\ to log information
concerning the advancing of the simulations. pdac.log is excluseively of
output, it is connected explicitly to the fortran unit 6. Note that \PDAC\ 
does not use standard output stream, but on many
system the unit 6 is associated by default to the standard output, therefore
on these systems pdac.log is actually a redirected standard output stream.
Is important also to underline that when a new pdac.log is opened \PDAC\ cancel 
any existing pdac.log file.
pdac.log is a free text file with self-explanatory informations.

\subsubsection{pdac.tst file}
\label{section:padc_tst}

This file, like pdac.log, is a free text file, and contains debugging
and testing informations, that are meant for developers only,
therefore many informations present there are of no use for the user.
The file is connected to the fortran unit 7, and is cleared every new run.

\subsubsection{pdac.err file}
\label{section:padc_err}

Like pdac.log and pdac.tst also pdac.err is a free textual file,
it is connected to the fortran unit 8. In this file \PDAC\ writes
all messages related to severe errors that cause the program to stop.
Note that at variance with pdac.log and pdac.tst this file is not 
cleared at every run, but it is opened in append mode.

\subsubsection{pdac.res file}
\label{section:padc_res}

The pdac.res file is the file where \PDAC\ dumps all fields and
parameters that are required to restart a simulation, from the 
value of time parameter at which pdac.res has been written, exactly as
if no interruption occurred. The file is in binary format to maintain
the numerical accuracy of the internal representation of numbers
and to save space. This file is not meant for post processing
and analysis of the fields, since its internal layout could
vary easily from one version of the code to the other.
The pdac.res file is connected to the fortran unit 9, it is
opened in read mode at the beginning of a restart simulation run,
to read in the status of a previous simulation to be continued.
Then pdac.res file is opened in write mode at regular interval
of simulated time. The rate at which pdac.res is written
with a new dump of the simulation status is specified by
the user through the control parameter "tdump" (see \ref{section:input_parameters} ).
It is very inportant to understand the importance of this file,
and that all the appropriate measure should be taken in order
not to lose it. We suggest to meke a copy of the file in 
a safe device (like tapes) before every run, since if somthing appens
when the file is opened in write mode, you lose the possibility
to continue the simulation.


\subsubsection{OUTPUT.nnnn file}

\subsection{Post processing files}
\label{section:file_postprocessing}

\subsubsection{Post processing input files}

\subsubsection{Post processing output files}

\subsection{General remark}

\noindent Caveat:

\noindent 1. ..... MAXIMUM DIMENSION

\noindent 2. ..... DISK SPACE

\noindent 3. ..... AVOIDING DATA LOSS
