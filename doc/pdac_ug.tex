%%%%%%%%%%%%%%%%%%%%%%%%%%%%%%%%%%%%%%%%%%%%%%%%%%%%%%%%%%%%%%%%%%%%%%%%%%%%
%                                                                          %
%              (C) Copyright 2003  INGV and CINECA                         %
%                      All Rights Reserved                                 %
%								  	   %
%%%%%%%%%%%%%%%%%%%%%%%%%%%%%%%%%%%%%%%%%%%%%%%%%%%%%%%%%%%%%%%%%%%%%%%%%%%%

\documentclass[11pt]{article}
\usepackage{graphicx}
\usepackage{makeidx}
\usepackage{psfig}

\makeindex

% define margins, etc
\topmargin	0.1in
\oddsidemargin	0in
\evensidemargin	0in
\textheight	8.80in
\textwidth	6.50in
\marginparsep	0.25cm
\headheight	0in
\headsep	0in
%\footskip	0.5in
%\footheight	0in%

% define macros
%
% generally useful macros
%
\newcommand{\htmlimage}[1] {}
\newcommand{\REFAND} {\&}
\newcommand{\ETALNP}{\mbox{\it et al}}
\newcommand{\ETAL}{\mbox{\ETALNP{\it.}}}
\newcommand{\eqnref}[1] {\mbox{eq (\ref{#1})}}
\newcommand{\mycite}[2] {\cite{#2}}

\newcommand{\PDAC} {PDAC}
\newcommand{\PDACNAME} {Pyroclastic Dispersal Analysis Code}
\newcommand{\PDACDATE} {\today}
\newcommand{\PDACAUTHORS} {
 A.~Neri, T.~Esposti Ongaro, C.~Cavazzoni }
\newcommand{\PDACVERSION} {2.0}
\newcommand{\PDACADDRESS} {\underline{pdac@pi.ingv.it}}
\newcommand{\PDACURL} {\underline{http://www.pi.ingv.it/PDAC}}
% title of PDAC paper
\newcommand{\PDACPAPER} {PDAC: A transient, multiphase flow code
for the 3D simulation of pyroclastic dispersal dynamics}


%
% macros for style conventions when describing the program.
%

% name of class or object in program
\newcommand{\OBJ}[1] {{\bf\tt#1}}

% function arguments
\newcommand{\FA}[2] {{\rm{\bf#1}\ {\it#2}}}

% global function name
\newcommand{\FN}[3] {{\rm\bf#1}\ {\tt #2(}#3{\tt)}}

% class member function name
\newcommand{\FNO}[4] {{\rm\bf#2}\ \OBJ{#1::}{\tt#3(}#4{\tt)}}

% list item, for optional components, parameters, etc.
\newcommand{\TTLISTITEM}[1] {\item {\tt #1} \\}
\newcommand{\RMLISTITEM}[1] {\item {\rm #1} \\}
\newcommand{\BOLDLISTITEM}[1] {\item {\bf #1} \\}
\newcommand{\EMLISTITEM}[1] {\item {\em #1} \\}
\newcommand{\LISTITEM}[1] {\RMLISTITEM{#1}}

%
% other generally useful macros
%

\newcommand{\UGTITLE} {User's Guide}
\newcommand{\RMTITLE} {Reference Manual}
\newcommand{\UGDESC} {
The \PDAC\ (\PDACNAME)\ {\em\UGTITLE\/} describes how to run and use the 
various features of the \PDACNAME\ \PDAC.  
This guide includes the capabilities of the program, how 
to use these capabilities, the necessary input files and 
formats, and how to run the program both on uniprocessor 
machines and in parallel.}

\newcommand{\RMDESC} {
The \PDAC\ (\PDACNAME)\ {\em\RMTITLE\/} is intended for the user who wants
to modify any features of \PDAC\ or for the programmer who wants to 
cooperate to the \PDAC\ development. The manual describes how the model 
equations are discretized and solved, provides the explicative list of 
all variables used in the numerical code and gives a brief description of the 
parallelization strategy, so that the advanced user can optimize the
code efficiency on some computational platform.}

\newcommand{\PG}{\PDAC\ {\it Programmer's Guide\/}}
\newcommand{\UG}{\PDAC\ {\it User's Guide\/}}
\newcommand{\RM}{\PDAC\ {\it Reference Manual\/}}
\newcommand{\prettypar}{

\smallskip

}

\newcommand{\eg}{{\it e.g.\/}}
\newcommand{\ie}{{\it i.e.\/}}

\newcommand{\KEY}[1]{{\tt #1}}
\newcommand{\IKEY}[1]{{\tt #1\index{#1 psfgen command}}}
\newcommand{\OKEY}[1]{$[${\tt #1}$]$}
\newcommand{\ARG}[1]{$<${\em #1}$>$}
\newcommand{\OARG}[1]{$[${\em #1}$]$}
\newcommand{\ARGDEF}[2]{$<${\em #1}$>$: #2}
\newcommand{\KEYDEF}[2]{{\tt #1}: #2}
\newcommand{\COMMAND}[4]{%
  #1 \\ {\bf Purpose:} #2 \\ {\bf Arguments:} #3 \\ {\bf Context:} #4 }

\newcommand{\icommand}[1]{#1\index{#1 command}}

\newcommand{\PDACCONF}[4]{%
%  \addcontentsline{toc}{subparagraph}{#1}%
  {\bf \tt #1 } $<$ #2 $>$ \\%
  \index{#1 parameter}
  {\bf Acceptable Values: } #3 \\%
  {\bf Description: } #4%
}

\newcommand{\PDACCONFWDEF}[5]{%
%  \addcontentsline{toc}{subparagraph}{#1}%
  {\bf \tt #1 } $<$ #2 $>$ \\%
  \index{#1 parameter}
  {\bf Acceptable Values: } #3 \\%
  {\bf Default Value: } #4 \\%
  {\bf Description: } #5%
}


\setcounter{secnumdepth}{3}
%% \setcounter{tocdepth}{5} %% very detailed table of contents
\setcounter{tocdepth}{4}

%
% the document itself
%

\begin{document}

% EXPLORIS heading
\input{Exploris_3.2_head}

% initial pages for the guide - title, TOC, etc.
%%%%%%%%%%%%%%%%%%%%%%%%%%%%%%%%%%%%%%%%%%%%%%%%%%%%%%%%%%%%%%%%%%%%%%%%%%%%

% title page

\thispagestyle{empty}

\vspace*{0.3in}

\begin{centering}
  \rule{6in}{0.04in}				\\	\vspace{0.25in}
  {\Huge \PDAC\ \UGTITLE}			\\	\vspace{0.25in}
  {\Large Version \PDACVERSION}          	\\	\vspace{0.20in}
  {\Large \PDACNOTES}                    	\\	\vspace{0.20in}
  \rule{6in}{0.04in}				\\	\vspace{0.25in}
  {\Large \UGAUTHORS}				\\	\vspace{0.20in}
  \PDACDATE					\\	\vspace{0.20in}
  \rule{6in}{0.04in}				\\	\vspace{0.25in}
  {\large       Istituto Nazionale di Geofisica e Vulcanologia} \\ 
  {\large       Pisa - Italy}      \\ 
\end{centering}
\vspace{0.2in}

\begin{center}
  {\Large \bf Description}
\end{center}

\noindent \UGDESC



% copyright and permissions notices
%\newpage
%
\thispagestyle{empty}

\section*{PDAC (\PDACNAME )\\
Non-Transferable, Non-Commercial Use License}

\subsubsection*{\PDAC: \PDACNAME\ LICENSE AGREEMENT}

Upon execution of this Agreement by the party identified below (“Licensee”), The Istituto Nazionale di Geofisica e Vulcanologia (“INGV”), on behalf of the \PDAC\ Development Team (“PDAC Team”), will provide \PDAC\ in Executable Code and/or Source Code form (“Software”) to Licensee, subject to the following terms and conditions. For purposes of this Agreement, Executable Code is the compiled code, which is ready to run on Licensee’s computer. Source code consists of a set of files which contain the actual program commands that are compiled to form the Executable Code. 

\begin{enumerate}
\item Copyright and nature of the license

The main authors of \PDAC\ are Augusto Neri, Giovanni Macedonio and Tomaso Esposti Ongaro (the PDAC Team). The intellectual property of this Software is owned by the authors at INGV, and all right, title and interest, including copyright, remain with INGV. INGV grants, and Licensee hereby accepts, a non-exclusive, non-transferable license to use the Software for academic, research and internal business purposes only e.g. not for commercial use (see Paragraph 5 below), without a fee. 
Because substantial funds in the development of \PDAC\ have been provided by the European Commission  and Ministero dell’Istruzione, dell’Universit\`a e della Ricerca (Italy), any use, distribution or sublicense of the Software is subject to the specific policies of these institutions.

\item Copies and modifications

Licensee agrees to reproduce the copyright notice and other proprietary markings on all copies of the Software. Licensee has no right to transfer or sublicense the Software to any unauthorized person or entity. However, Licensee does have the right to make complementary works that interoperate with \PDAC\ and to freely distribute such complementary works.
Licensee may modify the Software to make derivative works, for its own academic, research, and internal business purposes. Licensee’s distribution of any derivative work is also subject to the same restrictions on distribution and use limitations that are specified herein for INGV Software. Prior to any such distribution the Licensee shall require the recipient of the Licensee’s derivative work to first execute a license for \PDAC\ with INGV in accordance with the terms and conditions of this Agreement. Any derivative work should be clearly marked and renamed to notify users that it is a modified version and not the original \PDAC\ code distributed by INGV. 

\item Disclaimer of Warranty

Except as expressly set forth in this Agreement, THIS SOFTWARE IS PROVIDED “AS IS” AND INGV MAKES NO REPRESENTATIONS AND EXTENDS NO WARRANTIES OF ANY KIND, EITHER EXPRESS OR IMPLIED, INCLUDING BUT NOT LIMITED TO WARRANTIES OR MERCHANTABILITY OR FITNESS FOR A PARTICULAR PURPOSE, OR THAT THE USE OF THE SOFTWARE WILL NOT INFRINGE ANY PATENT, TRADEMARK, OR OTHER RIGHTS. LICENSEE ASSUMES THE ENTIRE RISK AS TO THE RESULTS AND PERFORMANCE OF THE SOFTWARE AND/OR ASSOCIATED MATERIALS. LICENSEE AGREES THAT INGV SHALL NOT BE HELD LIABLE FOR ANY DIRECT, INDIRECT, CONSEQUENTIAL, OR INCIDENTAL DAMAGES WITH RESPECT TO ANY CLAIM BY LICENSEE OR ANY THIRD PARTY ON ACCOUNT OF OR ARISING FROM THIS AGREEMENT OR USE OF THE SOFTWARE AND/OR ASSOCIATED MATERIALS. 

\item Citation

The user agrees that any reports or published results obtained with the Software will acknowledge its use by the appropriate citation as follows: 

\begin{quote}
Neri, A., Esposti Ongaro, T., Macedonio, G., and Gidaspow, D., 2003. Multiparticle simulation of collapsing volcanic columns and pyroclastic flows. J. Geophys. Res., 108, B4, 2202, doi:10.1029/2001JB000508
\end{quote}

\begin{quote}
Esposti Ongaro, T., Cavazzoni, C., Erbacci, G., Neri, A., and Salvetti, M.V., 2007. A parallel multiphase flow code for the 3D simulation of explosive volcanic eruptions. Parallel Computing, doi:10.1016/j.parco.2007.04.003.
\end{quote}

One copy of each publication or report will be supplied to INGV at the addresses listed below in Contact Information.

Electronic documents will include a direct link to the official page: {\tt http://vmsg.pi.ingv.it}

\item Commercial use

Should Licensee wish to make commercial use of the Software, Licensee will contact INGV at the contact infromation addresses listed below to negotiate an appropriate license for such use. Commercial use includes: (1) integration of all or part of the Software into a product for sale, lease or license by or on behalf of Licensee to third parties, or (2) distribution of the Software to third parties that need it to commercialize product sold or licensed by or on behalf of Licensee. 

\item Development

\PDAC\ is being distributed as a research and teaching tool and as such, the PDAC Team encourages contributions from users of the code that might, at INGV’ sole discretion, be used or incorporated to make the basic operating framework of the Software a more stable, flexible, and useful product. Licensees that wish to contribute their code to become an internal portion of the Software may contact the PDAC Team at the addresses listed below in Contact Information.

\end{enumerate}
\newpage
\subsubsection*{Contact Information}

The best contact path for licensing issues is by e-mail to \PDACADDRESS\ or send correspondence to:
\begin{verse}
                             Tomaso Esposti Ongaro\\
                             Istituto Nazionale di Geofisica e Vulcanologia\\
			     Sezione di Pisa\\
                             32, Via della Faggiola\\
			     56126, Pisa - Italy\\
                             FAX: +39-0508311942\\
                             email: ongaro@pi.ingv.it
\end{verse}
\begin{verse}
                             Augusto Neri\\
                             Istituto Nazionale di Geofisica e Vulcanologia\\
			     Sezione di Pisa\\
                             32, Via della Faggiola\\
			     56126, Pisa - Italy\\
                             FAX: +39-0508311942\\
                             email: neri@pi.ingv.it
\end{verse}




% table of contents
\newpage
\tableofcontents

% list of figures
\newpage
\listoffigures

% list of tables
% \listoftables
%
% There are currently NO tables in either the User Guide or 
% the Programmer's Guide.  
%

\newpage


% Introduction
\section{Introduction}
\label{section:intro}

\PDAC\ is a numerical code that has been developed to solve the multiphase
flow equations for the simulation of pyroclastic atmospheric dispersal dynamics.
It is specifically designed for high performance computing on UNIX platforms.
This document describes how to use 
\PDAC, its features, and the platforms on which it runs.
The document is divided into six sections:
\begin{description}
\item[Section \ref{section:intro}] gives an overview of \PDAC.
\item[Section \ref{section:files}] describes \PDAC\ file formats.
\item[Section \ref{section:input_par}] lists basic simulation options.
\item[Section \ref{section:input_sample}] provides sample configuration files.
\item[Section \ref{section:run}] gives details on running \PDAC.
\item[Section \ref{section:avail}] gives details on installing \PDAC.
\end{description}

We have tried to make this document complete and easy to understand as well 
as \PDAC\ itself easy to install and run.
We welcome any suggestion for improving the documentation and code itself
at \PDACADDRESS\

\subsection{The \PDAC\ main features}

The \PDAC\ code solves the multiphase flow transport equations for
density, momentum and energy of a gas-pyroclast mixture in a steady
standard atmosphere.
The gas phase can be composed of several chemical components leaving
the crater - such as water vapor, carbon dioxide, etc. - and
atmospheric air (considered as a single chemical component).
The pyroclasts are described by $N$ phases of solid particles,
each one characterized by a diameter, density, specific
heat, thermal conductivity, and viscosity, and considered
representative of a granulometric class commonly present in
the eruptive mixture. Momentum and energy exchange between the gas
and the different particulate phases are expressed through semi-empirical
correlation equations.

\prettypar
\PDAC\ has several important features that make it a useful tool for the
simulation of volcanic columns and pyroclastic flows: 

\begin{itemize}

\item{\bf Robust over a wide range of flow regimes}
\begin{itemize}
\item from weak to strong gas-particle coupling;
\item from laminar to turbulent regimes;
\item from quasi-isothermal to highly buoyant flows;
\item from subsonic (incompressible) to supersonic (compressible) regimes.
\end{itemize}

\item{\bf Unlimited number of particles classes}\\
The number of particle classes used to approximate an observed distribution
of grain sizes is limited only by the computational resources available.
Up to 6 particulate phases have been used successfully. The size and density
of particles is nevertheless constrained by the multiphase flow assumptions,
that limit the diameter of grain particles from about $10 \mu m$ to a few
millimeters.

\item{\bf Up to 7 gas species}\\
Up to 7 gas species can be considered as chemical components of the gas phase. The 
molecular composition allowed are: $O_2,N_2,CO_2,H_2,H_2O,Air,SO_2$.
Atmospheric air is considered as a single component with averaged properties.
Gas species are treated as tracers and no chemical reactions are allowed up to
now. Future versions of \PDAC\ will hopefully consider interphase mass transfer
(such as water vapour condensation) and chemical reactions.

\item{\bf Atmospheric stratification}\\
A standard stratified atmosphere can be assigned as initial condition.
The temperature gradient within the different atmospheric layers are
the Earth averaged values, whereas pressure is computed by assuming hydrostatic
equilibrium. No vertical stratification of humidity is assumed at this time. 
Cross wind can be imposed in a Cartesian reference frame.

\item{\bf Control of simulation options}\\
The simple textual input file allows \PDAC\ to be started from prescribed
initial conditions or from a binary restart file, representing the state
of the main flow fields when the code stops for any reason. When the code
is started from scratch, vent inlet flow conditions the volcano topography, 
and the atmospheric conditions need to be  specified as input conditions. 
When the code is restarted, the final state saved in the restart file is used 
as initial condition. The computational mesh and all numerical parameters 
are easily specified in the input file.

\end{itemize}

\subsection{New features in version \PASTPDACVERSION}

The numerical code \PDAC\ is based on previous numerical codes developed
by Dobran et al. (1993), Neri and Macedonio (1996) and Neri et al. (2003)
for the multiphase flow simulation of collapsing volcanic columns. We based
the release \PASTPDACVERSION\ on the Neri et al. (2003) model (considered 
here as version 1.0). If you are interested in knowing more about
the history of \PDAC\, consult the \PDAC\ web site at 
\PDACURL\ or contact the development team at 
\PDACADDRESS\

\subsubsection*{New modular structure}
The release \PASTPDACVERSION\ of \PDAC\ has been fully rewritten in 
Fortran90 by adopting a modular approach. Dynamic allocation of memory
allows the use of a precompiled \PDAC\ code. In addition the more readable 
source code will help the programmers to mantain and develop \PDAC\ more easily.
If you wish to implement a new feature in \PDAC\ you are encouraged to contact 
the developers team at \PDACADDRESS\ for guidance.

\subsubsection*{Parallel implementation and improved serial performance}
The parallel implementation of \PDAC\ includes load balancing sheme 
and ad-hoc communication routines that allow \PDAC\ to scale efficiently 
on many processors, thus reducing the execution time almost linearly.
The data distribution method can be chosen from input to improve the 
load-balancing and thus the scalability. All routines have been 
optimized in order to improve the efficiency of the memory access.
Optimized subroutines have been implemented in 3D for three phases and
first-order simulations, with considerable performance gain.

\subsubsection*{Portable on most architectures}
Any machine with a message passing library compatible with the MPI 
standard should be able to run \PDAC\ in parallel.
In recent years, distributed memory parallel computers have been offering
cost-effective computational power.  \PDAC\ was designed to run efficiently
on such parallel machines using large number of cells. 
\PDAC\ is particularly well suited also for the increasingly popular Beowulf-class PC 
clusters, which are quite similar to the workstation clusters for which is was 
originally designed.
Future versions of \PDAC\ will also make efficient use of clusters of 
multi-processor workstations or PCs.

\subsubsection*{Second-order spatial and third-order temporal discretizations}
Second-order spatial discretization schemes have been introduced in order
to improve the code accuracy and to reduce the so-called numerical 
dissipation which represents a limit in the simulation of turbulent flows.
High-order schemes are implemented as corrections of the first-order
scheme previously used. Few parameters have to be specified in the input
file to control numerical accuracy. Up to third-order low-storage 
Runge-Kutta temporal discretizations can be selected by the input file.

\subsubsection*{Dynamic Smagorinsky model}
Dynamic Smagorinsky sub-grid scale turbulence model 
(Germano et al., {\it Phys. Fluids}, 1991) has been implemented. This 
procedure allows to compute the Smagorinsky constant from the simulated
fluid flow. An input flag allows the selection of the standard or dynamic model.

\subsubsection*{Improved control on convergence and errors}
More controls on the convergence and timing are added to reduce and
easily identify the source of convergence errors.

\subsubsection*{Possible fully implicit solution of model equations}
Fully implicit solution or implicit pressure coupling of the energy
equation can be selected by setting appropriate input parameters. This can
be useful in the evaluation of the accuracy of the numerical solution
but it comes out to be computationally very expensive.

\subsubsection*{Easy model settings}
Running inviscid simulation, or testing different turbulence models or closure 
relations can be easily controlled from the input files.

\subsubsection*{More friendly I/O}
A slef-explanatory Input file is now available. Pre-processing of OUTPUT files
is also introduced. Binary or ascii formatted output files can be selected.

\subsection{New features in version \PDACVERSION}

\subsubsection*{Extension to 3D}
Model equations have been rewritten in 3D Cartesian formulation. The
finite-volumes discretization schemes and the skeleton of the solution
algorithm have been mantained and the solution procedures have been extended
to 3D rectilinear, non-uniform, cartesian mesh.
New grid-decomposition schemes suited to 3D meshes have been devised, while
mantaining the low-level parallelization strategy used in 2D.
The old 2D Cartesian and cylindrical code has been nested into the 3D routines.

\subsubsection*{Parallel optimization}
Detailed benchmarks have been carried out on several (High Performance 
Computing (HPC) platforms to asses the efficiency of different 
domain-decomposition schemes with different mesh sizes. A low-level
profiling on the code and sub-routines performances has been carried out
to evaluate the bottle-necks in the run execution.
Optimized routines have been implemented for different parameter configurations.
The different levels of optimization can be selected from input.

\subsection*{Automatic grid generation}
New procedures for the definition of a non-uniform rectilinear mesh
have been implemented. Mesh parameters (cell size increase rate, domain
size, resolution, etc.) are set from input.

\subsubsection*{Integration of a 3D topography and UTM geo-referencing}
New routines for the integration of a real volcano topography from a 
standard ASCII Digital Elevation Model (DEM) have been integrated in the
setup procedures. These include interpolation, filtering, averaging
functions. The georeferencing of the computational mesh based on the 
UTM coordinates of the volcanic vent has been introduced.

\subsubsection*{New initial conditions at the vent}
Vent conditions can now be set from scratch with a simple procedure.
The vent is automatically located on the topography without the need
to specify the inlet cells.

\subsubsection*{New boundary conditions}
New boundary conditions for the free in-outflow boundaries
and for the topography have been implemented. 
Non-reflecting atmospheric boundary conditions have been set.
The topographic (no-slip) boundary
conditions have been introduced by adopting an extension to multiphase
compressible flows of the {\em immersed boundaries} technique.

\subsection{Added features in release \RELEASEVERSION}

Release \RELEASEVERSION fixed some minor bugs and re-organized
the module hierarchy. Some new boundary conditions have been
also implemented.

\subsubsection*{New initial conditions for dome explosions}
An automatic procedure to impose the initial conditions for the
explosion of a pressurized, confined volume of particles has been
implemented. This conditions can be representative of the initial
state of an exploding lava dome. The grain size distribution, 
the pressurization model, the dome porosity and permeability can
be easily set.

\subsubsection*{New automatic routines for the bluff-body problem}
An automatic procedure to tag the blunt blocks and to compute the
drag and lift forces have been implemented. This allow to compute,
for example, the action exerted by a pyroclastic flow on a building.

\subsection{User feedback}

If you have problems installing or running \PDAC\ after
reading this document, please send a
complete description of the problem by email to \PDACADDRESS.
If you discover and fix a problem not described in this manual we would
appreciate if you would tell us about this as well, so that we can alert
other users and incorporate the correction into the public distribution.
\prettypar
We are interested in making \PDAC\ more useful to the volcanological
community.  Your suggestions are welcome at \PDACADDRESS\ 
We also appreciate hearing from you about how you are using \PDAC\ 
in your work.

\subsection{Collaborations}
The formulation of the high-order numerical schemes, for both spatial
and temporal discretizations, has been developed in collaboration with
Maria Vittoria Salvetti (Dip.to di Ingegneria Aerospaziale, Universit\`a 
di Pisa) and Fran\c{c}ois Beux (Scuola Normale Superiore, Pisa). The 
immersed boundaries technique has been implemented in collaboration with
Mattia De'Michieli Vitturi (Dip.to di Matematica, Universit\`a di Pisa, now 
at INGV, Pisa). 
We also credit Maria Vittoria Salvetti for the formulation of the turbulence 
Large Eddy Simulation (LES) model and Sara Barsotti (INGV - Pisa) 
for the implementation.

\subsection{Acknowledgments}

The developement of the \PDAC\ release \PDACVERSION\ has been supported 
by European project EXPLORIS (EVR1-CT2002-40026), Gruppo Nazionale per
la Vulcanologia (Italy) and Ministero dell'Istruzione, dell'Universit\`a 
e della Ricerca (Italy).
\prettypar
The authors would also like to thank the High Performance Computing
staff of CINECA-Italy, for providing the computational resources and expertises
needed for developing and running \PDAC.
\prettypar


% Input and output files
\newpage
\section{Input and Output Files}
\label{section:files}

This section describes all the files that are red or written by \PDAC.
Release \PDACVERSION\ does not have a post-processing tool integrated,
so that all interaction you have with the program is through textual files.
The I/O files are always red and written in the running directory,
i.e. the directory from which \PDAC\ has been run.
The level of verbosity, controlling the amount of information written
by the program during the execution, can be set before starting a run.
Note that a too high level of verbosity for very long-lasting simulations
could cause the program to stop.

\subsection{\PDAC\ input file formats}
\label{section:input_files}

\PDAC\ has two possible input files from which it gets simulation parameters
and the initial conditions. 

A textual parameters file \FIL{pdac.dat} should be compiled by the user
as descirbed below.  \PDAC\ will stop if this file is not present in the 
directory where the executable is running.
It contains the namelists and cards with the parameters of the simulation
and the various option controlling the program execution.

The other Input file that can be required by the program is the restart
file \FIL{pdac.res}. 
\PDAC\ will stop if the restart mode is selected but this file is not present 
in the directory where the executable is running.
This is a binary file written by the code itself and
contains the dump of all independent fields that are saved when a
simulation run is interrupted for any reason. Therefore it cannot be edited by 
the user. Depending on the size of the problem envisaged, the restart file can 
be very large, so that you must ensure that your file-system can manage it.

In a parallel execution all input files are opened, red, and closed
by the root processor (the one with MPI id equal to 0), that
is charged of broadcasting data to the other computing units.

\subsubsection{The \FIL{pdac.dat} file}
\label{section:padc_dat}

This file contains the parameters that define the physical system
you want to simulate with \PDAC. \FIL{pdac.dat} is not red from standard 
input but is connected explicitly to the Fortran unit 5. The file should
be present in the running directory of \PDAC\, and it is always red
at the beginning of the simulation run even if the run is a 
continuation/restart of a previous simulation run. 
The parameters and options in the \FIL{pdac.dat} file are specified
in a set of F90 namelists followed by cards for topography, 
initial and boundary conditions.
The options and values specified determine the exact behavior of
\PDAC, what features are active or inactive, how long the simulation
should last, etc.  
Sect. \ref{section:input_par} lists the parameters required to run a 
simulation, their default values and the range of variability.
Several sample \PDAC\ parameter files are shown in Sect. 
\ref{section:input_sample}.\\

The layout of \FIL{pdac.dat} file is as follow:

\begin{verbatim}

&control
  control_parameters1 = control_value1,
  control_parameters2 = control_value2,
  ...
/

&model
  model_parameters1 = model_value1,
  model_parameters2 = model_value2,
  ...
/

&mesh
  mesh_parameters1 = mesh_value1,
  mesh_parameters2 = mesh_value2,
  ...
/

&particles
  particles_parameters1 = particles_value1,
  particles_parameters2 = particles_value2,
  ...
/

&numeric
  numeric_parameters1 = numeric_value1,
  numeric_parameters2 = numeric_value2,
  ...
/

'ROUGHNESS'
roughness_parameters

'MESH'
nonuniform_mesh_distances

'FIXED_FLOWS'
fixed_flows_parameters

'INITIAL_CONDITIONS'
initial_conditions_parameters

\end{verbatim}

Within a namelist, the parameters are specified as
\begin{verbatim}
keyword      =     value
\end{verbatim}
Blank lines in the namelists are ignored.  Comments are prefaced by
a {\tt !} and may appear after the keyword assignment:
\begin{verbatim}
keyword  = value          !  This is a comment
\end{verbatim}
or may be at the beginning of a line:
\begin{verbatim}
! keyword  = value   This is a keyword commented out
\end{verbatim}
Some keywords require several values, that are specified as an array of values
separated by a comma:
\begin{verbatim}
keyword  = value_1, value_2, value_3,
\end{verbatim}

Note that the namelists ( \&control, \&model, \&mesh, \&particles and \&numeric )
and the cards ( \OBJ{'ROUGHNESS'}, \OBJ{'MESH'}, \OBJ{'FIXED\_FLOWS'} and \OBJ{'INITIAL\_CONDITIONS'} )
should appear in exactly this order. 
Parameters lines in the namelists are optional, if a parameter is omitted
it will take a default values (see Sect. \ref{section:input_par} ).
On the contrary parameter lines in the cards are not optional and
should always be present.
Commet are allowed in the namelist using the "!" character to identify 
that what follows is a comment. Commet are not allowed within the cards
parameter, but you could insert free text (not including cards and namelists names)
between different namelists and cards.
Parameters within cards are in general distributed on more lines depending
on the simulation parameters. The layout of the input parameters 
within each cards is described in details in Sect. \ref{section:input_par}.

\subsubsection{The \FIL{pdac.res} file}
\label{section:pdac_res}

\PDAC\ dumps all fields and parameters that are required to restart a 
simulation from exactly the same instant on the file \FIL{pdac.res}.
The file is in double precision binary format to maintain the numerical 
accuracy of the internal representation of numbers and to save space. 
This file is not meant for post-processing
and analysis of the fields, since its internal layout could
vary easily from one version of the code to the other.
The restart file is connected to the Fortran unit 9, it is
opened in {\it READ} mode at the beginning of a restart simulation run,
to read in the status of a previous simulation interrupted.
Then \FIL{pdac.res} file is opened in {\it WRITE} mode at regular intervals
of simulated time. The rate at which \FIL{pdac.res} is overwritten
with a new dump of the simulation status is specified by
the user through the control parameter "tdump" 
(see Sect. \ref{section:input_par} ).
All the appropriate measures should be taken in order
not to lose this file. We suggest to make a copy of the file in 
a safe device (such as tapes) before every run, since any error occurring
when the file is opened in write mode could make you losing it.

\subsection{\PDAC\ output file formats}
\label{section:output_files}

\PDAC\ writes several output files which can be subdivided into 
information and data files. 
Information files (\FIL{pdac.log}, \FIL{pdac.tst} and
\FIL{pdac.err} ) contain logging, testing and debugging information
meant for the user to control the progress of the simulation.
Data files (\FIL{OUTPUT.nnnn}) contain the values of all the 
independent physical fields saved at regular interval of time, in different 
files distinguished by an incremental index (the ``nnnn'' in the file name).
Obviously the restart file described above can be considered also as an 
output data file.

Data output files are opened,
written and closed by the root processor (the one having MPI rank equal to 0), 
that gathers data from the other computing units before writing.
All other information output files 
are opened, written and closed by all processors, and can be distinguished
by an extension indicating the rank of the processor. 

\subsubsection{The \FIL{pdac.log} file}
\label{section:padc_log}

The file \FIL{pdac.log} is used by \PDAC\ to log information
concerning the progress of the simulation. \FIL{pdac.log} is 
connected explicitly to the Fortran unit 6. 
\FIL{pdac.log} is a free text file with self-explanatory information.
Note that \PDAC\ does not use standard output stream but on many
systems the unit 6 is associated by default to the standard output. Therefore,
on these systems, \FIL{pdac.log} is actually a redirected standard output stream.
It is important also to underline that when a new \FIL{pdac.log} is opened 
\PDAC\ cancels any existing \FIL{pdac.log} file.

\subsubsection{The \FIL{pdac.tst} file}
\label{section:padc_tst}

The file \FIL{pdac.tst}, like \FIL{pdac.log}, is a free text file that contains debugging
and testing informations. This is meant for developers only,
therefore the information present in it is not particularly useful to the common user.
The file is connected to the Fortran unit 7 and is cleared every new run.

\subsubsection{The \FIL{pdac.err} file}
\label{section:padc_err}

Like \FIL{pdac.log} and \FIL{pdac.tst}, \FIL{pdac.err} is a free textual file.
It is connected to the Fortran unit 8. In this file \PDAC\ writes
all messages related to severe errors that cause the program to stop.
Note that this file is not cleared at every run but it is opened in 
{\it APPEND} mode.

\subsubsection{Output file}

All scalar and vector fields related to the evolution of the simulated
system are saved into the \FIL{OUTPUT.nnnn} files at regular intervals of
simulated time. The user can choose the interval of time through
the control parameter \OBJ{tpr} (see Sect. \ref{section:input_par} ).
The extension ``nnnn'' is a progressive number, ranging from 0000 to 9999, 
incremented each time the fields are written. 
The output files are associated to the Fortran unit 12. 
The format of these file can be selected by the user.
The logical control parameter \OBJ{formatted\_output} controls if the
output will be written in formatted textual format
or in binary (single precision) format. In general, the output files 
are meant for post-processing analysis, it is therefore useful to know 
their layout. In both formatted and binary forms, the fields are saved
in this order:

\begin{verbatim}

time            ! Simulated time

p               ! Thermodynamic pressure
ug              ! Gas velocity along x
vg              ! Gas velocity along y (only for 3D simulations)
wg              ! Gas velocity along z
tg              ! Gas temperature

xgc             ! Molar fraction of gas components
...             ! (for all gas components)

eps(s)          ! Volumetric fraction of particle class (s)
us(s)           ! Particle velocity along x
vs(s)           ! Particle velocity along y (only for 3D simulations)
ws(s)           ! Particle velocity along z
ts(s)           ! Particle temperature
...             ! (for all solid phases)

\end{verbatim}

where the first line is a scalar real value representing the 
simulated time at which the fields are written and the oter lines
represent the fields (of dimension {\tt nx*ny*nz}) that are written 
using the Fortran format specifier {\tt FORMAT( 5(G14.6E3,1X) )}.
The components of the gas and solid velocities in the y dimension
({\tt vg} and {\tt vs}) are present in the output file only if the simulation
type is 3D.

When the file is written in the binary unformatted format,
time and each variable is written in a separate record.
In particular time is written in a record of lenght 1
while all other fields in records of lenght {\tt nx*ny*nz}.

\subsection{Post-processing files}
\label{sect:pp}

\PDAC\ release \PDACVERSION\ provides the executable \FIL{pp.x} utility,
that is intended to be a pre-processing tool for visualization of
output files produced by \PDAC. Even though this tool is still under 
development its usage is shortly described in this section in order to make it
useful for the user, especially when the output of very large simulations
are too big to be processed directly by a visualization tool.
\FIL{pp.x} extracts from a series of \FIL{OUTPUT.nnnn} files the individual single fields,
with the possibility of downsize, interpolate, and crop the data
and stores them in separate files that can be easily plotted with
various graphical application.

\subsubsection{Post-processing input files}

The pre-processing tool must read the input file \FIL{pdac.dat}, to which
an additional namelist with the parameters that control the post-processing
is added. The name of the additional namelist is \OBJ{\&pp}, and
should be typed in between the namelist \&numeric and the first card
\OBJ{'ROUGHNESS'} . The layout of the \FIL{pdac.dat} file for the post-processing
is then:

\begin{verbatim}
...
...

&numeric
  numeric_parameters1 = numeric_value1, 
  numeric_parameters2 = numeric_value2,
  ...
/

&pp
  pp_parameter1 = pp_value1,
  pp_parameter2 = pp_value2,
  ...
/

'ROUGHNESS'
roughness_parameters

...
...
\end{verbatim}

Note that you can leave the \&pp namelist in the \FIL{pdac.dat} file
for the usage of main program too since it will be simply ignored.

\subsubsection{Post-processing output files}

The post-processing produces several output files
containing all individual fields and grid information useful
for visualization.
In particular, post-processing \FIL{pp.x} reads a sequence
of \FIL{OUTPUT.nnnn} files (the numbers indicating the first, the last, and
the increment of ``nnnn'' are red in input)
and writes out a sequence of files for each individual field.
The name of new files are of the form \FIL{filter.[fieldname].nnnn}.
The format of these file is the same of output, as specified above.
A list of all files produced by the post-processing is the following:

\begin{itemize}

\item \FIL{pdac.xml}  \\
      this file contains all the information contained in \FIL{pdac.dat}
      written using an XML compliant format

\item \FIL{pdac.grx} \\
      this file contains the list of cell centers (excluding border cells) 
      in x-direction

\item \FIL{pdac.gry} \\
      this file contains the list of cell centers (excluding border cells) 
      in y-direction

\item \FIL{pdac.grz} \\
      this file contains the list of cell centers (excluding border cells) 
      in z-direction

\item \FIL{filter.pgas.nnnn} \\
      these files contain the gas pressure field 

\item \FIL{filter.ug.nnnn} \\
      these files contain the gas velocity field in x-direction
      for the gas phase

\item \FIL{filter.vg.nnnn} \\
      these files contain the gas velocity field in y-direction
      for the gas phase
      This file is not present in 2D simulations

\item \FIL{filter.wg.nnnn} \\
      these files contain the gas velocity field in z-direction
for the gas phase

\item \FIL{filter.tgas.nnnn} \\
      these files contain the gas temperature field 

\item \FIL{filter.xgGG.nnnn} \\
      these files contain the molar fraction fields for all the gas 
      components (GG runs from 01 to ngas) 

\item \FIL{filter.epSS.nnnn} \\
      these files contain the particle volume fraction fields for all the 
      solid phases (SS runs from 01 to nsolid) 

\item \FIL{filter.tsSS.nnnn} \\
      these files contain the particles temperature fields
      for all solid phases (SS run from 01 to nsolid)

\item \FIL{filter.usSS.nnnn} \\
      these files contain the particles velocity fields in x-direction
      for all solid phases (SS run from 01 to nsolid)

\item \FIL{filter.vsSS.nnnn} \\
      these files contain the particles velocity fields in y-direction
      for all solid phases (SS run from 01 to nsolid)
      These files are not present in 2D simulations

\item \FIL{filter.wsSS.nnnn} \\
      these files contain the particles velocity fields in z-direction
      for all solid phases (SS run from 01 to nsolid)


\end{itemize}

Note that when running the post-processing all 
\FIL{OUTPUT.nnnn} files in the selected range should 
be present in the running directory of \FIL{pp.x}

%\subsection{General remark}
%
%\noindent Caveat:
%
%\noindent 1. ..... MAXIMUM DIMENSION
%
%\noindent 2. ..... DISK SPACE
%
%\noindent 3. ..... AVOIDING DATA LOSS


% Input simulation parameters
% (INPUT)
\newpage
\section{Simulation Parameters}
\label{section:input_par}

\subsection{\PDAC\ input namelists}
\label{section:namelists}

\subsubsection{\OBJ{control} namelist}
The \OBJ{control} namelist contains the parameters that control the program
flow, such as the start mode, the initial and final time, the output printing.

\begin{itemize}
\item
\PDACCONFWDEF{run\_name}{run identificative}{any string}{{\tt pdac\_run\_2d}}
{This string identifies the simulation. Its length must not exceed 80 characters. It must be enclosed by quotes.}

\item
\PDACCONFWDEF{job\_type}{2d or 3d}{{\tt '2d','2D','3d','3D'}}{{\tt '2D'}}
{When 2D is selected, the model equations are simplified by assuming
some simmetry in the physical domain (translation in one cartesian 
direction or cylindrical symmetry)}

\item
\PDACCONFWDEF{restart\_mode}{flag for restart}{{\tt 'from\_scratch'}, {\tt 'restart'}, {\tt 'check\_geom'}, {\tt 'check\_init'}, {\tt 'outp\_recover'}}
{{\tt 'from\_scratch'}}
{If {\tt restart} is selected, initial conditions are set up by using the fields
dumped (in double precision) into the restart file {\tt pdac.res} (see above). 
Otherwise, initial conditions are set up from input data in {\tt pdac.dat}.
If {\tt 'check\_geom'}, or {\tt 'check\_init'} are selected, the program stops
after the mesh and topography or the initial conditions are set, respectively.
{\tt 'outp\_recover'} can be set to restart from the output file specified
by the \OBJ{nfil} parameter.} 

\item
\PDACCONFWDEF{time}{initial time}{double real}{0.0}
{Inital simulation time (in seconds)}

\item
\PDACCONFWDEF{tstop}{end time}{double real}{100.0}
{Time (in seconds) at which the simulation stops}

\item
\PDACCONFWDEF{dt}{time step}{double real}{0.01}
{Time advancement step (in seconds). 
{\tt dt} is constrained by the CFL condition
$dt < C_{max}\frac{\Delta x}{V_{max}}$, where $\Delta x$ is
the size of the computational grid and $V_{max}$ is the maximum
velocity. The maximum CFL number $C_{max}\approx 0.2$ has been 
found empirically.}

\item
\PDACCONFWDEF{lpr}{level of verbosity}{1,2,3}{2}
{increases the level of verbosity in warning and error messages 
in {\tt pdac.log}, {\tt pdac.err}, {\tt pdac.tst} files}

\item
\PDACCONFWDEF{imr}{flag for mass residuals}{0/1}{0}
{ When 1 selected, writes out the total gas and particle mass in the domain
and the mass flow rate, in a specific file \FIL{pdac.chm}}

\item
\PDACCONFWDEF{isrt}{flag for runtine sampling of the pressure}{0/1}{0}
{ When 1 selected, writes out in the file \FIL{pwav.dat} the pressure field at every time-step
in the locations specified by a file named \FIL{probes.dat}}

\item
\PDACCONFWDEF{tpr}{time interval for OUTPUT file}{$>$ dt}{1.0}
{OUTPUT files are printed every {\tt tpr} seconds of simulated time}

\item
\PDACCONFWDEF{tdump}{time interval for restart file}{$>$ dt}{20.0}
{restart file is overwritten every {\tt tdump} seconds of simulated time}

\item
\PDACCONFWDEF{nfil}{number of first OUTPUT file}{any positive integer up to 9999}{0}
{Initial value for numbering the output files or for recovering.
Output files are written with 4 digits extension ({\tt OUTPUT.XXXX}). 
The output time can be recovered as $({\tt XXXX - nfil})* {\tt tpr}$ }

\item
\PDACCONFWDEF{tau}{transient vent conditions}{double real}{0.0}
{Time needed to esablish the flow conditions at the vent (in seconds)}

\item
\PDACCONFWDEF{tau1}{transient vent conditions}{double real}{0.0}
{Time needed to esablish the flow conditions at the vent (in seconds)}

\item
\PDACCONFWDEF{tau2}{transient vent conditions}{double real}{0.0}
{Time needed to esablish the flow conditions at the vent (in seconds)}

\item
\PDACCONFWDEF{formatted\_input}{flag for INPUT format}{T/F}{T}
{Determine the format of input files for restart (valid for both options
{\em outp\_recover} and {\em restart}): T - formatted ascii, 
F - binary 4-bytes }

\item
\PDACCONFWDEF{formatted\_output}{flag for OUTPUT format}{T/F}{T}
{Determine the format of output files: T - formatted ascii, 
F - binary 4-bytes }

\item
\PDACCONFWDEF{max\_seconds}{maximum CPU time}{any real value}{20000.0}
{Maximum duration of a simulation (in seconds). If time exceeds this value
a restart file is written before the simulation is stopped (useful for
scheduling). Default time is set to 6 hours}
\end{itemize}

\subsubsection{\OBJ{model} namelist}
The \OBJ{model} namelist is intended for all switches that can be selected
by the user to modify the model equations, either by neglecting some terms
in the transport equations or by using different constitutive equations
and submodels.

\begin{itemize}

\item
\PDACCONFWDEF{icpc}{gas specific heat}{0,1}{1}
{1: specific heat depends on temperature}

\item
\PDACCONFWDEF{irex}{chemical reactions}{0,1}{0}
{The chemical reaction module has not yet been implemented}

\item
\PDACCONFWDEF{gas\_viscosity}{gas diffusive transport}{{\tt T/F}}{{\tt T}}
{T to solve the full set of model equations. F switches the diffusive
transport terms off for the gas phase (viscous and turbulent term in momentum 
equation, thermal diffusivity in enthalpy equation). 
Gas viscosity is computed anyway to include the gas-particle drag and
energy exchange terms.}

\item
\PDACCONFWDEF{part\_viscosity}{particle diffusive transport}{{\tt T/F}}{{\tt T}}
{Switches on/off the diffusive terms in the particle transport equations 
(viscous terms in the momentum equation and thermal conductivity).}

\item
\PDACCONFWDEF{iss}{turbulence model for particles}{0,1}{0}
{1: computes a gas-analogous sub-grid stress (turbulent viscosity)
for particles.}

\item
\PDACCONFWDEF{repulsive\_model}{flag for Coulombic repulsive model}{0,1}{1}
{The Coulombic repulsive model add a contribution to the diagonal part of
the solid viscous stress due to the repulsive interaction of particles 
at high concentrations. This term appears to be important for the solid
equation to be well-posed.}

\item
\PDACCONFWDEF{iturb}{turbulence model for gas}{0,1,2}{1}
{0 - no gas turbulence model; 1 - Smagorinsky sub-grid stress model;
2 - Smagorinsky sgs model with roughness closure at the walls.}

\item
\PDACCONFWDEF{modturbo}{Subgrid-scale model for gas turbulence}{1,2}{1}
{1 - Classical Smagrinsky model; 2 - Dynamic Smagorinsky model}

\item
\PDACCONFWDEF{cmut}{Smagorinsky constant}{usually between 0.1 and 0.4}{0.1}
{The exact value cannot be predicted {\it a priori}. The use of the dynamic
Smagorinsky model (see above) makes the assignment of this constant
non-necessary}

\item
\PDACCONFWDEF{gravx}{acceleration along x}{any real}{0.0}
{Body-force in x(r)-direction}

\item
\PDACCONFWDEF{gravy}{acceleration along y}{any real}{0.0}
{Body-force in y-direction}

\item
\PDACCONFWDEF{gravz}{acceleration along z}{any real}{-9.81}
{Body-force in z-direction (usually the value of the gravitational acceleration).
Values different from the default value could not be consistent with the 
atmospheric stratification, leading to instabilities. A value of 0.0 
suppresses atmospheric stratification}.

\item
\PDACCONFWDEF{ngas}{number of gas components}{1 to 7}{2}
{Seven gas species are defined, specifically: \\
$ 1) O_2, 2) N_2, 3), CO_2, 4) H_2, 5) H_2O, 6) Air, 7) SO_2$. 
Only gas species that are specified at the inlet or in the
atmosphere are considered by the model. The total number of gas 
species specified by this flag must be therefore consistent with 
input conditions, otherwise the program stops.}

\item
\PDACCONFWDEF{density\_specified}{flag for i.c.}{{\tt T/F}}{{\tt F}}
{If .TRUE., specify gas density instead of temperature for initial conditions.
Particle temperature is set equal to gas temperature.}

\end{itemize}

\subsubsection{\OBJ{mesh} namelist}
Here all parameters concerning the spatial discretization of the computational
domain must be selected. In \PDAC\ you are constrained to discretization 
on a rectilinear (non-)uniform mesh. Cylindrical coordinates can be
selected only in 2D.

\begin{itemize}
\item
\PDACCONFWDEF{nx}{Number of cells in x(r)-direction}{up to 512}{100}
{When 2D cylindrical coordinates are selected, ``x'' is used instead 
 of ``r''. This number includes the boundary ghost cells. The maximum number can be 
 increased by modifying the max\_size parameter in the ``dimensions'' module}

\item
\PDACCONFWDEF{ny}{Number of cells in y-direction}{up to 512}{1}
{Not used in 2D. It includes the boundary ghost cells.}

\item
\PDACCONFWDEF{nz}{Number of cells in z-direction}{up to 512}{100}
{z is the second space coordinate in 2D. It includes the boundary ghost cells.}

\item
\PDACCONFWDEF{n0x}{Number of uniform cells in x-direction}{up to 512}{10}
{Number of cells with minimum size in x-direction}

\item
\PDACCONFWDEF{n0y}{Number of uniform cells in y-direction}{up to 512}{10}
{Number of cells with minimum size in y-direction}

\item
\PDACCONFWDEF{n0z}{Number of uniform cells in z-direction}{up to 512}{10}
{Number of cells with minimum size in z-direction}

\item
\PDACCONFWDEF{npx}{Number of cells to add in x-direction }{up to 512}{0}
{Adds cells with maximum size \OBJ{dxmax} at the end of the x mesh.}

\item
\PDACCONFWDEF{nmx}{Number of cells to add in x-direction }{up to 512}{0}
{Adds cells with maximum size \OBJ{dxmax} at the beginning of the x mesh.}

\item
\PDACCONFWDEF{npy}{Number of cells to add in y-direction }{up to 512}{0}
{Adds cells with maximum size \OBJ{dymax} at the end of the y mesh.}

\item
\PDACCONFWDEF{nmy}{Number of cells to add in y-direction }{up to 512}{1}
{Adds cells with maximum size \OBJ{dymax} at the beginning of the y mesh.}

\item
\PDACCONFWDEF{npz}{Number of cells to add in z-direction }{up to 512}{1}
{Adds cells with maximum size \OBJ{dzmax} at the end of the z mesh.}

\item
\PDACCONFWDEF{nmz}{Number of cells to add in z-direction }{up to 512}{1}
{Adds cells with maximum size \OBJ{dzmax} at the beginning of the z mesh.}

\item
\PDACCONFWDEF{itc}{flag for cylindrical coordinates}{0,1}{0}
{In 2D simulations, itc=1 sets the grid and the coordinates to cylindrical,
by modifying the discretized equations.}

\item
\PDACCONFWDEF{iuni}{flag for uniform mesh}{0,1}{0}
{0 - non uniform mesh; 1 - uniform mesh, takes {\tt dx0, dy0, dz0} as the
cells size in the two/three coordinate directions.}

\item
\PDACCONFWDEF{dx0}{cell sizes in x(r)-direction}{any real $>$ 0.0}{10.D0}
{Size along x(r) of the cells in metres, when the flag for uniform mesh is selected}

\item
\PDACCONFWDEF{dy0}{cell sizes in y-direction}{any real $>$ 0.0}{10.D0}
{Size along y of the cells in metres, when the flag for uniform mesh is selected}

\item
\PDACCONFWDEF{dz0}{cell sizes in z-direction}{any real $>$ 0.0}{10.D0}
{Size along z of the cells in metres, when the flag for uniform mesh is selected}

\item
\PDACCONFWDEF{grigen}{flag for grid generation}{0/1}{0}
{Automatically builds a non-uniform mesh when 1 is selected}

\item
\PDACCONFWDEF{maxbeta}{Maximum increase rate for non-uniform cells}{any real $>$ 1.0}{1.2}
{The ratio between the size of two adjacent cells. Increasing the non-uniform rate can deteriorate the accuracy of the numerical solution.}

\item
\PDACCONFWDEF{dxmin}{Minimun cell size in x(r) direction}{any real $>$ 0.0}{10.0}
{The minimum cell size (in metres) determines the highest numerical resolution and constraints the time-step size accordingly to the CFL condition (see above)}

\item
\PDACCONFWDEF{dymin}{Minimun cell size in y direction}{any real $>$ 0.0}{10.0}
{The minimum cell size (in metres) determines the highest numerical resolution and constraints the time-step size accordingly to the CFL condition (see above)}

\item
\PDACCONFWDEF{dzmin}{Minimun cell size in z direction}{any real $>$ 0.0}{10.0}
{The minimum cell size (in metres) determines the highest numerical resolution and constraints the time-step size accordingly to the CFL condition (see above)}

\item
\PDACCONFWDEF{dxmax}{Maximum cell size in x(r) direction}{any real $>$ 0.0}{10.0}
{The cell size is increased at a given rate from minimum to maximum size, until the domain\_x is covered. The code stops if the maximum cell size is too small.}

\item
\PDACCONFWDEF{dymax}{Maximum cell size in y direction}{any real $>$ 0.0}{10.0}
{The cell size is increased at a given rate from minimum to maximum size, until the domain\_y is covered. The code stops if the maximum cell size is too small.}

\item
\PDACCONFWDEF{dzmax}{Maximum cell size in z direction}{any real $>$ 0.0}{10.0}
{The cell size is increased at a given rate from minimum to maximum size, until the domain\_z is covered. The code stops if the maximum cell size is too small.}

\item
\PDACCONFWDEF{alpha\_x}{Relative position of the mesh center}{[0.0:1.0]}{0.5}
{The center of the mesh corresponds to the position of the more refined part of the domain in the x(r) direction}

\item
\PDACCONFWDEF{alpha\_y}{Relative position of the mesh center}{[0.0:1.0]}{0.5}
{The center of the mesh corresponds to the position of the more refined part of the domain in the y direction}

\item
\PDACCONFWDEF{alpha\_z}{Relative position of the mesh center}{[0.0:1.0]}{0.0}
{The center of the mesh corresponds to the position of the more refined part of the domain in the z direction}

\item
\PDACCONFWDEF{center\_x}{UTM longitude of the mesh center}{}
{The UTM longitude is needed to reference the mesh when volcanic topography has to be read. Usually corresponds to the crater vent longitude}

\item
\PDACCONFWDEF{center\_y}{UTM latitude of the mesh center}{}
{The UTM longitude is needed to reference the mesh when volcanic topography has to be read. Usually corresponds to the crater vent latitude}

\item
\PDACCONFWDEF{domain\_x}{Domain size along x(r)}{any real $>$ 0.0}{10000}
{Size (in metres) of the computational domain along x(r) for automatic mesh generation}

\item
\PDACCONFWDEF{domain\_y}{Domain size along x(r)}{any real $>$ 0.0}{10000}
{Size (in metres) of the computational domain along y for automatic mesh generation}

\item
\PDACCONFWDEF{domain\_z}{Domain size along x(r)}{any real $>$ 0.0}{10000}
{Size (in metres) of the computational domain along z for automatic mesh generation}

\item
\PDACCONFWDEF{zzero}{Grid bottom level}{any real}{0.0}
{Elevation of the mesh bottom above sea level. The atmosphere is correctly
described below 80 Km.}

\item
\PDACCONFWDEF{mesh\_partition}{domain decomposition criterion}{1,2,3}{1}
{1 - Layers decomposition, 2 - 2D Blocks decomposition, 3 - 3D Blocks decomposition}

\end{itemize}

\subsubsection{\OBJ{Boundaries} namelist}
The \OBJ{Boundaries} namelist contains the flags that specify the
conditions to be imposed on the domain boundaries. The domain boundaries
are identified by their compass position, i.e. west-east in x-direction,
south-north in y-direction and bottom-top in z-direction. The flag for the
use of the immersed boundaries technique for no-slip boundaries is also
included.
\begin{itemize}
\item
\PDACCONFWDEF{west}{Flag for west boundary}{integer $\le$ 20}{2}
{In 2D cylindrical, it must accounts for axial symmetry (flag=2). In 3D, 
inflow/ouflow conditions (flag = 4,6) are usually imposed.}

\item
\PDACCONFWDEF{east}{Flag for east boundary}{integer $\le$ 20}{2}
{In 2D and 3D inflow/ouflow conditions (flag = 4,6) are usually imposed.}

\item
\PDACCONFWDEF{south}{Flag for south boundary}{integer $\le$ 20}{2}
{Inflow/ouflow conditions (flag = 4,6) are usually imposed (just 3D).}

\item
\PDACCONFWDEF{north}{Flag for north boundary}{integer $\le$ 20}{2}
{Inflow/ouflow conditions (flag = 4,6) are usually imposed (just 3D).}

\item
\PDACCONFWDEF{bottom}{Flag for bottom boundary}{integer $\le$ 20}{3}
{Bottom boundary condition represents usually a no-slip solid surface}

\item
\PDACCONFWDEF{top}{Flag for top boundary}{integer $\le$ 20}{4}
{Please consider that top boundary can influence the behaviours of
the atmospheric conditions during the simulation.}

\item
\PDACCONFWDEF{immb}{Flag for immersed boundaries}{0/1}{0}
{When immb = 1, a generic solid boundary immersed in the computational domain
is described by the immersed-boundaries technique. Please notice that this 
procedure can be computationally expensive.}

\item
\PDACCONFWDEF{ibl}{Flag for blunt body}{0/1}{0}
{When ibl = 1, the code writes out the lift and drag forces at each time-step
on a rectangular block defined in the FIXED\_FLOWS card.}

\end{itemize}

\subsubsection{\OBJ{topography} namelist}
The present version of \PDAC can include (in 2D and 3D) an external file specifying a generic volcano topography. In 2D the topography can be given on a non-uniform mesh, whereas in 3D it is assumed that a standard ASCII DEM file is read.

\begin{itemize}
\item
\PDACCONFWDEF{dem\_file}{Name of the DEM external file}{string between single quotes}{'topo.dat'}
{Please controls that the ASCII file does not contain meta-characters (e.g. from Windows coding or decompression).}

\item
\PDACCONFWDEF{itp}{Flag for importing DEM file}{0/1}{0}
{If itp = 1, read the topography from a file.}

\item
\PDACCONFWDEF{ismt}{Flag for smoothing DEM}{1/2/3}{1}
{Smoothing of the DEM is performed accordingly to a median filter (1), a 
Gaussian filter (2) or a Barnes filter (3). The filtersize is defined
by the corresponding input parameter.}

\item
\PDACCONFWDEF{iavv}{Flag for averaging the topography}{0/1}{0}
{Averages the topography to obtain an axisymmetric topography.}

\item
\PDACCONFWDEF{itrans}{Flag for vertical mesh translation} {0/1}{0}
{Translates the mesh vertically up to the minimum topographic elevation (to save cells).}

\item
\PDACCONFWDEF{nocrater}{Flag for crater flattening}{T/F}{F}
{Allows to flatten a region of the volcano around the vent at a given rim\_quota (see below).}

\item
\PDACCONFWDEF{rim\_quota}{Elevation a.s.l. where the topography is flattened}{any real}{1000}
{If 'nocrater' is selected, flatten the topography at this quota.}

\item
\PDACCONFWDEF{filtersize}{High-pass filtering size (in metres) for the topography}{any real $>$ 0.0}{50}
{The filtering procedure is obtained by a combination of sub-sampling and averaging.}

\item
\PDACCONFWDEF{cellsize}{Resample the topography on a given mesh}{any real $>$ 0.0}{10}
{This value is usually set equal to the dxmin, dymin values}

\item
\PDACCONFWDEF{zrough}{roughness length}{real}{1.0}
{Default Roughness length. Typical values for ground
ranges from few centimeters to some metres.}

\end{itemize}

\subsubsection{\OBJ{inlet} namelist}
The flow conditions at the volcanic vent can be specified
explicitly in the \OBJ{'FIXED\_FLOWS'} card for both 2D and 3D simulations.
Alternatively, in 3D, the \OBJ{inlet} namelist can be used to impose the vent
conditions on a {\em circular} vent, without specifying the flow field in each 
cell of the vent. It also includes some {\em antialiasing} procedures to correct the mass flow rate
when the circular source is discretized on the cartesian mesh. The mass flow rate
can be adjusted also by feeding the jet from the partially filled cells with a randomly
fluctuating flow.

\begin{itemize}
\item
\PDACCONFWDEF{ivent}{Flag for inlet conditions on a circular vent}{0/1}{0}
{Vent conditions are automatically imposed on the topography if {\tt ibl=1}}

\item
\PDACCONFWDEF{iali}{Flag for anti-aliasing}{0/1/2/3}{0}
{Activates the anti-aliasing procedure to correct the mass flux in the cells
partially filled by the topography, on the vent boundary. 0: no antialias;
1: density antialias (including gas); 2: density antialias (only particles); 
3: velocity antialias}

\item
\PDACCONFWDEF{irand}{Flag for random switch}{0/1}{0}
{Activates the random-switch procedure. The inlet cells are switched on
(i.e. the flux is feeded) with a probability proportional to the area not 
occupied by the topography.}

\item
\PDACCONFWDEF{ipro}{Flag for flow radial profile}{0/1}{0}
{Allows to set a radial inflow profile read from an external file (see below).}

\item
\PDACCONFWDEF{rad\_file}{Name of the file defining the profile}{string between single quotes}{'profile.dat'}
{The input file is described in the i/o files section of this manual}

\item
\PDACCONFWDEF{xvent}{UTM longitude of the center of the vent}{any real}{0.0}
{If the mesh is generated from a DEM file, the vent center is translated in the more refined region
(i.e., in the mesh ``center\_x'')}

\item
\PDACCONFWDEF{yvent}{UTM latitude of the center of the vent}{any real}{0.0}
{If the mesh is generated from a DEM file, the vent is translated in the more refined region
(i.e., in the mesh ``center\_y'')}

\item
\PDACCONFWDEF{vent\_radius}{Radius of the vent}{any real $>$ 0.0}{100}
{The vent radius in metres}

\item
\PDACCONFWDEF{base\_radius}{The radius of the base of the crater}{any real $>$ 0.0}{200}
{This value must be equal, at least, to the double of the vent radius.}

\item
\PDACCONFWDEF{crater\_radius}{The external radius of the crater}{any real}{500}
{This value is approximate, and it is used to flatten the crater, if prescribed.}

\item
\PDACCONFWDEF{u\_gas}{Inlet gas velocity x}{any real}{0.0}
{}

\item
\PDACCONFWDEF{v\_gas}{Inlet gas velocity y}{any real}{0.0}
{}

\item
\PDACCONFWDEF{w\_gas}{Inlet (averaged) gas velocity z}{any real}{0.0}
{}

\item
\PDACCONFWDEF{wrat}{Ratio between the maximum and the averaged vertical gas veloctiy}{any real $>$ 1.0}{1.0}
{If this value is $>$ 1.0, the velocity has a maximum of 'wrat*w' on the axis and falls to zero at the vent rim
following a power-law profile. Note that this options shall significantly modify the mass flow rate.}

\item
\PDACCONFWDEF{p\_gas}{Pressure at the vent}{any real}{101325.0}
{}

\item
\PDACCONFWDEF{t\_gas}{Temperature at the vent}{any real $>$ 0.0}{288.15}{}

\item
\PDACCONFWDEF{u\_solid}{Inlet particle velocity x}{any real}{0.0}
{The values for different particle classes must be separated by commas.}

\item
\PDACCONFWDEF{v\_solid}{Inlet particle velocity y}{any real}{0.0}
{The values for different particle classes must be separated by commas.}

\item
\PDACCONFWDEF{w\_solid}{Inlet particle velocity z}{any real}{0.0}
{The values for different particle classes must be separated by commas.}

\item
\PDACCONFWDEF{ep\_solid}{Averaged particle fraction}{any real $>$ 0.0}{0.0}
{The values for different particle classes must be separated by commas.}

\item
\PDACCONFWDEF{t\_solid}{Particle temperature}{any real $>$ 0.0}{288.15}
{The values for different particle classes must be separated by commas.}

\item
\PDACCONFWDEF{vent\_O2}{Averaged concentration of O2 at the vent}{any real [0.0:1.0]}{}{}

\item
\PDACCONFWDEF{vent\_N2}{Averaged concentration of N2 at the vent}{any real [0.0:1.0]}{0.0}{}

\item
\PDACCONFWDEF{vent\_CO2}{Averaged concentration of CO2 at the vent}{any real [0.0:1.0]}{0.0}{}

\item
\PDACCONFWDEF{vent\_H2}{Averaged concentration of H2 at the vent}{any real [0.0:1.0]}{0.0}{}

\item
\PDACCONFWDEF{vent\_H2O}{Averaged concentration of H2O at the vent}{any real [0.0:1.0]}{1.0}{}

\item
\PDACCONFWDEF{vent\_Air}{Averaged concentration of Air at the vent}{any real [0.0:1.0]}{0.0}{}

\item
\PDACCONFWDEF{vent\_SO2}{Averaged concentration of SO2 at the vent}{any real [0.0:1.0]}{0.0}{}
\end{itemize}

\subsubsection{\OBJ{Dome} namelist}
This namelist contains the numerical and model parameters to define initial
conditions for dome explosion simulations. It includes the parameters for
the Woods et al. (2002) model for the dome pressurization.
This namelist is intended for both 2D/3D simulations. For 2D cylindrical
simulations, the dome center is always located on the symmetry axis.

\begin{itemize}
\item
\PDACCONFWDEF{idome}{Flag for inlet conditions on a spherical dome}{0/1/2}{0}
{0: no dome; 1: gas overpressure set through porous dome model; 2: constant overpressure.
Dome cells are automatically imposed over the topography if {\tt ibl=1}}

\item
\PDACCONFWDEF{idw}{Flag for hydrostatic pressure}{0/1}{0}
{1: add hydrostatic pressure of the dome edifice to the gas overpressure}

\item
\PDACCONFWDEF{xdome}{UTM longitude of the center of the dome}{any real}{0.0}
{The dome can be located out from the more refined region}

\item
\PDACCONFWDEF{ydome}{UTM latitude of the center of the dome}{any real}{0.0}
{The dome can be located out from the more refined region}

\item
\PDACCONFWDEF{dome\_volume}{Volume of the dome}{any real $>$ 0.0}{1.D6}
{Note: the dome volume includes the gas fraction (it does NOT correspond to the
Dense Rock Equivalent volume). The total mass in the dome resulting from the
automatic procedure depends on the mesh discretization and on the topography.}

\item
\PDACCONFWDEF{conduit\_radius}{minimal radius of the dome}{any real $>$ 0.0}{15.D0}
{The dome pressure is constant within the conduit\_radius}

\item
\PDACCONFWDEF{overpressure}{Constant overpressure can be imposed within the dome}{any real $>$ 0.0}{100.D5}
{This value is used if {\tt idome = 2}}

\item
\PDACCONFWDEF{particle\_fraction}{particle volumetric fractions}{array of real [0.0:1.0]}{0.7}
{The size of the array equals {\tt nsolid}. The sum of the particle volumetric fractions
must not exceed 1.0}

\item
\PDACCONFWDEF{gas\_flux}{Gas flux from the conduit}{any real $>$ 0.0}{400.0}
{}

\item
\PDACCONFWDEF{temperature}{Dome temperature in Kelvin}{any real $>$ 0.0}{1100.D0}
{}

\item
\PDACCONFWDEF{permeability}{Dome permeability}{any real $>$ 0.0}{1.D-12}{}

\item
\PDACCONFWDEF{dome\_gasvisc}{Gas viscosity}{any real $>$ 0.0}{1.D-5}{}

\item
\PDACCONFWDEF{dome\_O2}{O2 mass fraction}{any real [0.0:1.0]}{0.0}{}

\item
\PDACCONFWDEF{dome\_N2}{N2 mass fraction}{any real [0.0:1.0]}{0.0}{}

\item
\PDACCONFWDEF{dome\_CO2}{CO2 mass fraction}{any real [0.0:1.0]}{0.0}{}

\item
\PDACCONFWDEF{dome\_H2}{H2 mass fraction}{any real [0.0:1.0]}{0.0}{}

\item
\PDACCONFWDEF{dome\_H2O}{H2O mass fraction}{any real [0.0:1.0]}{1.0}{}

\item
\PDACCONFWDEF{dome\_Air}{Air mass fraction}{any real [0.0:1.0]}{0.0}{}

\item
\PDACCONFWDEF{dome\_SO2}{SO2 mass fraction}{any real [0.0:1.0]}{0.0}{}
\end{itemize}

\subsubsection{\OBJ{Atmosphere} namelist}
This namelist contains the parameters to build a stratified atmosphere from
the ground pressure and temperature. For seven atmospheric layers (troposphere,
tropopause, lower\_stratosphere, upper\_stratosphere, ozone\_layer, lower\_mesosphere and upper\_mesosphere) the distance of the top level from the ground and the temperature gradient must be given. Gas concentrations are constant. A constant wind velocity can also be specified.

\begin{itemize}
\item
\PDACCONFWDEF{wind\_x}{Component x of wind velocity}{any real}{0.0}
{}

\item
\PDACCONFWDEF{wind\_y}{Component y of wind velocity}{any real}{0.0}
{}
\item
\PDACCONFWDEF{wind\_z}{Component z of wind velocity}{any real}{0.0}
{}
\item
\PDACCONFWDEF{p\_ground}{Pressure at sea level}{any real}{101325.0}
{If stratification is switched off (gravity equals zero), this corresponds to the ambient pressure}

\item
\PDACCONFWDEF{t\_ground}{Temperature at sea level}{any real $>$ 0.0}{288.15}
{If stratification is switched off (gravity equals zero), this corresponds to the ambient temperature}

\item
\PDACCONFWDEF{void\_fraction}{Gas volume fraction in the atmosphere}{any real [0.0:1.0]}{1.0}{}

\item
\PDACCONFWDEF{max\_packing}{Maximum packing particle fraction}{any real [0.0:1.0]}{0.6413}{}

\item
\PDACCONFWDEF{atm\_O2}{Concentration of oxigen in the atmosphere}{any real [0.0:1.0]}{0.0}{}

\item
\PDACCONFWDEF{atm\_N2}{Concentration of nitrogen in the atmosphere}{any real [0.0:1.0]}{0.0}{}

\item
\PDACCONFWDEF{atm\_CO2}{Concentration of carbon dioxide in the atmosphere}{any real [0.0:1.0]}{0.0}{}

\item
\PDACCONFWDEF{atm\_H2}{Concentration of hydrogen in the atmosphere}{any real [0.0:1.0]}{0.0}{}

\item
\PDACCONFWDEF{atm\_H2O}{Concentration of water vapour in the atmosphere}{any real [0.0:1.0]}{0.0}{}

\item
\PDACCONFWDEF{atm\_Air}{Concentration of air in the atmosphere}{any real [0.0:1.0]}{1.0}
{The mass fraction (concentration) of each of the seven gas species must be
specified. The closure relation for the total mass fraction must be satisfied.
If this constrain is not satisfied, the Air mass fraction is
automatically corrected.}

\item
\PDACCONFWDEF{atm\_SO2}{Concentration of sulphur dioxide in the atmosphere}{any real [0.0:1.0]}{0.0}{}

\item
\PDACCONFWDEF{troposphere\_z}{Upper limit of the troposphere}{}{11000 m}{}

\item
\PDACCONFWDEF{troposphere\_grad}{Temperature gradient in the troposphere}{}{-6.5D-3}{}

\item
\PDACCONFWDEF{tropopause\_z}{Upper limit of the tropopause}{}{20000 m}{}

\item
\PDACCONFWDEF{tropopause\_grad}{Temperature gradient in the tropopause}{}{0.0}{}

\item
\PDACCONFWDEF{lower\_stratosphere\_z}{Upper limit of the lower stratosphere}{}{32000 m}{}

\item
\PDACCONFWDEF{lower\_stratosphere\_grad}{Temperature gradient in the loser stratosphere}{}{1.D-3}{}

\item
\PDACCONFWDEF{upper\_stratosphere\_z}{Upper limit of the upper stratosphere}{}{47000 m}{}

\item
\PDACCONFWDEF{upper\_stratosphere\_grad}{Temperature gradient in the upper stratosphere}{}{1.0D-3}{}

\item
\PDACCONFWDEF{ozone\_layer\_z}{Upper limit of the ozone layer}{}{51000 m}{}

\item
\PDACCONFWDEF{ozone\_layer\_grad}{Temperature gradient in the ozone layer}{}{0.0}{}

\item
\PDACCONFWDEF{lower\_mesosphere\_z}{Upper limit of the lower mesosphere}{}{71000 m}{}

\item
\PDACCONFWDEF{lower\_mesosphere\_grad}{Temperature gradient in the lower mesosphere}{}{-2.8D-3}{}

\item
\PDACCONFWDEF{upper\_mesosphere\_z}{Upper limit of the upper mesosphere}{}{80000 m}{}

\item
\PDACCONFWDEF{upper\_mesosphere\_grad}{Temperature gradient in the upper mesosphere}{}{-2.0D-3}{}
\end{itemize}

\subsubsection{\OBJ{particles} namelist}
In this namelist the number of solid phases and the physical properties of the
particles forming each phase are specified. Note that the order in which 
particle classes are specified must be the order of the solid-phases array 
storage, in order to be consistent with the initial conditions assignement.

\begin{itemize}

\item
\PDACCONFWDEF{nsolid}{number of solid phases}{up to 10}{2}
{The number of solid phases considered in the multiphase equations.
The maximum number can be increased by modifying the max\_nsolid parameter
in the ``dimensions'' module.}

\item
\PDACCONFWDEF{diameter}{effective diameter particles (in microns)}{real}{100 microns}
{The model works well for particle diameters smaller than few millimeters}

\item
\PDACCONFWDEF{density}{microscopic particle density}{real}{2700 Kg/m$^3$}
{Depends on materials and porosity}

\item
\PDACCONFWDEF{sphericity}{particle sphericity}{0.5 to 1.0}{1.0}
{Partially accounts for particle shapes}

\item
\PDACCONFWDEF{viscosity}{Empirical viscosity coefficient}{0.5 to 2.0}{0.5[Pa s]}
{Largest values apply to coarse particles}

\item
\PDACCONFWDEF{specific\_heat}{solid specific heat}{real}{1.2D3 [J/(K Kg)]}
{Depends on materials}

\item
\PDACCONFWDEF{thermal\_conductivity}{solid thermal conductivity for Fourier law}{real}{2.D0}
{Depends on materials}

\end{itemize}

\subsubsection{\OBJ{numeric} namelist}
By modifying these parameters, you can modify the way \PDAC\ solves the 
model equations. We recommend non-expert users to avoid modifying the
default values.

\begin{itemize}

\item
\PDACCONFWDEF{rungekut}{order of Runge-Kutta explicit integration}{1,2,3}{1}
{The low-storage Runge-Kutta algorithm is used for explicit time integration.
The coefficients used in the RK integration are well suited only up to the
third-order. High-order temporal integration is recommended for 
convergence and stability when high-order spatial discretization schemes 
are used.}

\item
\PDACCONFWDEF{beta}{degree of upwinding}{0.0 to 1.0}{0.25}
{When High Order spatial discretization schemes are used and the MUSCL 
beta-scheme (both limited or unlimited) is selected, the degree of upwinding
can be chosen: \OBJ{beta}=0.0 corresponds to a completely centered schemes:
\OBJ{beta}=1.0 corresponds to a completely upwinded method. The accuracy depends
on the exact value: for \OBJ{beta}=0.25 and \OBJ{beta}=0.33 the formal third-order
is achieved.}

\item
\PDACCONFWDEF{muscl}{flag for MUSCL technique}{0,1}{0}
{The MUSCL technique extends the first-order discretization
scheme to high-orders by reconstructing the flux profile more accurately.
0 - uses first-order upwind; 1 - uses MUSCL technique.}

\item
\PDACCONFWDEF{mass\_limiter}{limiter type for mass equations}{0,1,2,3,4}{0}
{Select the type of MUSCL reconstruction and the limiter:\\
0 - beta scheme unlimited; 1 - Van Leer limiter; 2 - Minmod limiter;
3 - Superbee limiter; 4 - beta limited.}

\item
\PDACCONFWDEF{vel\_limiter}{limiter type for momentum equations}{0,1,2,3,4}{0}
{Select the type of MUSCL reconstruction and the limiter:\\
0 - beta scheme unlimited; 1 - Van Leer limiter; 2 - Minmod limiter;
3 - Superbee limiter; 4 - beta limited.}
\item
\PDACCONFWDEF{inmax}{maximum number of inner iterations}{integer (small)}{8}
{The exact value does not affect strongly the convergence but can slow down
the simulation. Optimal value is set by default.}

\item
\PDACCONFWDEF{maxout}{maximum number of Gauss-Siedel iterations for convergence}{integer (large)}{5000}
{Convergence is usually reached within less than 100 iterations. If maxout is reached the code CRASH!es}

\item
\PDACCONFWDEF{omega}{over/under-relaxation parameter}{0.0 to 2.0}{1.0}
{Values between 0.0 and 1.0 under-relax the iterative procedure, whereas
values above 1.0 overrelax.}

\item
\PDACCONFWDEF{optimization}{optimization degree}{1/2/3}{1}
{Level 1: not optimized. Level 2: Optimized memory access. Level 3: Optimized for 3D and 2 particle classes.}

\item
\PDACCONFWDEF{rlim}{Multiphase limit}{very small real value}{$10^{-8}$}
{It is related to minimum particle concentration for which multiphase flow
equations of momentum are solved. Below this limit one-phase equations are solved
(often critical for convergence on the cloud margins).}

\item
\PDACCONFWDEF{flim}{Multiphase limit}{very small real value}{$10^{-6}$}
{It is related to minimum particle concentration for which multiphase flow
equations of energy are solved. Below this limit one-phase equations are solved
(often critical for convergence on the cloud margins).}

\item
\PDACCONFWDEF{implicit\_fluxes}{flag for implicit computation of fluxes}{{\tt T/F}}{F}
{By selecting T, convective fluxes are computed (at the first or second-order, 
depending on flag \OBJ{muscl}) within the iterative solver. This modification
significantly slows down the computation but could relax the CFL constraint
so that larger time-steps can be used.}

\item
\PDACCONFWDEF{implicit\_enthalpy}{flag for implicit computation of enthalpies}
{{\tt T/F}}{F}
{Selecting the implicit solution of enthalpies enhances the level of coupling
between the momentum and enthalpy equations. Nevertheless the convective part 
of the enthalpy equations can be left out from the iterative solver
by opportunely selecting the flag \OBJ{implicit\_fluxes}. The performance loss
in any case is significant.}

\item
\PDACCONFWDEF{update\_eosg}{Update the equation of state}{{\tt T/F}}{T}
{Updating the equation of state after convergence is consistent with the old
version of PDAC but gives wrong solutions in test case (e.g. 1D and 2D shock waves)}
\end{itemize}

\subsection{\PDAC\ input cards}
\label{section:cards}

\subsubsection{\OBJ{'MESH'} card}

The \OBJ{'MESH'} card contains the two (in 2D) or three (in 3D) arrays of cell 
sizes specified for a non-uniform rectilinear mesh. 

\begin{itemize}
\item
\PDACCONF{dx(1:nx)}{Array of the cell sizes in x(r)-direction}{real}
{Array of the cell sizes in x(r)-direction.}

\item
\PDACCONF{dy(1:ny)}{Array of the cell sizes in y-direction}{real}
{Array of the cell sizes in y-direction.
Not present in 2D.}

\item
\PDACCONF{dz(1:nz)}{Array of the cell sizes in z-direction}{real}
{Array of the cell sizes in z-direction.}
\end{itemize}

Array elements can be separated by commas, tabs, blanks or lay on different
lines.  When uniform mesh is selected by {\tt iuni=1} parameter in the 
{\tt mesh} namelist, the card can be empty (but the card 
name must always be present in the input file).

\subsubsection{\OBJ{'FIXED\_FLOWS'} card}

The \OBJ{'FIXED\_FLOWS'} card is designed to specify boundary conditions (b.cs.).
Its name is due to the possibility to assign within this card any region with
specified flow conditions (e.g. for inlet flow). B.cs. are imposed in 
mesh region determined by rectangular blocks. The number of blocks is the
first parameter specified in the card

\begin{itemize}
\item
\PDACCONF{number\_of\_block}{number of blocks}{integer}
{The number of blocks used to specify b.c.}
\item
\PDACCONF{nblu}{Block tagged for the blunt body routines}{array of integers}
{If {\tt ibl=1} this array must contain a sequence of [0/1] indicating which
blocks must be tagged for the computation of drag and lift forces. If
{\tt ibl=0} it must not be present.}
\end{itemize}

Each block is characterized
by a flag that specifies the kind of b.c., and by two (in 2D) or three (in 3D)
couples of integer that specify the first and the last cell of the block in 
the two (three) directions. Each block is therefore identified by 5 (in 2D)
or 7 (in 3D) integers:\\

\begin{itemize}
\item
\PDACCONF{block\_type}{type of b.c.}{1 to 7}
{1 - fluid cells: all equations are solved; 2 - free-slip;
 3 - no-slip; 4 - free in/outflow; 5 - specified flow (inlet); 
 6 - extrapolation free in/outflow}

\item
\PDACCONF{block\_bounds}{limits of blocks}{integer}
{two or three couples of integers specifying 
$ i_{min}, i_{max}, (j_{min}, j_{max}), k_{min}, k_{max}$ }

\end{itemize}

Blocks are used also to specify ``blocking cells'' (e.g. 
the volcano topography). When the block type is equal to 5 (specified
flow), inlet conditions must be specified as follows:\\

First line:

\begin{itemize}
\item
\PDACCONF{fixed\_vgas\_x}{gas velocity component along x}{real}
{Inlet gas velocity along x(r)-direction. Note that CFL condition (see above for \OBJ{dt}) 
must be satisfied for every component.}

\item
\PDACCONF{fixed\_vgas\_y}{gas velocity component along y}{real}
{Inlet gas velocity along x(r)-direction (not present in 2D!).}

\item
\PDACCONF{fixed\_vgas\_z}{gas velocity component along z}{real}
{Inlet gas velocity along z-direction.}

\item
\PDACCONF{fixed\_pressure}{pressure at inlet}{real}
{The thermodynamic pressure of the gas phase.}

\item
\PDACCONF{fixed\_gaseps}{inlet gas volumetric fraction}{ $\le 1.0$}
{This value must be consistent with the solid phases volumetric fractions.
The closure relation imposes that the sum equals one.}

\item
\PDACCONF{fixed\_gastemp}{inlet gas temperature}{ a positive real }
{No constraint is imposed on gas temperature, but the range of validity
of the equation of state must be check.}
\end{itemize}

Further {\tt nsolid} lines:
\begin{itemize}
\item
\PDACCONF{fixed\_vpart\_x}{particle velocity along x}{real}
{Inlet particle velocity along x(r)-direction.
Note that CFL condition (see above for \OBJ{tt}) 
must be satisfied for every component.}

\item
\PDACCONF{fixed\_vpart\_y}{particle velocity along y}{real}
{Inlet particle velocity along y-direction(not present in 2D!).}

\item
\PDACCONF{fixed\_vpart\_z}{particle velocity along z}{real}
{Inlet particle velocity along z-direction.}

\item
\PDACCONF{fixed\_parteps}{particle volumetric fraction}{ $\le 1.0$}
{Must satisfy closure relation of total volumetric fraction}

\item
\PDACCONF{fixed\_parttemp}{particle temperature}{ a positive real}
{Particle temperature is not constrained by a thermal equation of state}

\end{itemize}

Last line:

\begin{itemize}

\item
\PDACCONF{fixed\_gasconc(1:7)}{concentration of each gas species at inlet}{$\le 1.0$}
{The mass fraction (concentration) of each of the seven gas species must be
specified. The closure relation for the total mass fraction must be satisfied.
If this constrain is not satisfied, the {\tt default\_gas} mass fraction is
automatically corrected.}

\end{itemize}



% Sample configuration files
\newpage
\section{Sample configuration files}
\label{section:sample}
This section contains some simple example \PDAC\ configuration files to serve
as templates.
\prettypar
This file shows a simple configuration file for simulation of an axisymmetric
Plinian column with one particle class and without topography.  

\begin{verbatim}
&control
 run_name = 'Sub_Plinian_2D',
 job_type = '2D',
 restart_mode = 'from_scratch', ! ( from_scratch | restart )
 time =  0.00,
 tstop = 900.0,
 dt    = 0.01,
 tpr   = 10.0,
 tdump = 50.0
/

&model
  cmut = 0.33,
  iss  = 0,
  repulsive_model = 1,
  iturb = 1
/

&mesh
  itc = 1, 
  nx = 100,
  nz = 200
/

&particles
  nsolid = 1, 
  diameter = 50.
  density = 1500.
  sphericity = 1.0
  viscosity = 0.5
  specific_heat = 1.2D3
  thermal_conductivity = 2.0D0
/

&numeric
  rungekut = 1,
  muscl = 1,
  lim_type = 4,
  beta = 0.25,
  omega = 1.00
/

'ROUGHNESS'
 1, 1.D0, 5.D3

'MESH'

  30.00D0  30.00D0  30.00D0  30.00D0  30.00D0
  30.00D0  30.00D0  30.00D0  30.00D0  30.00D0
  30.00D0  30.00D0  30.00D0  30.00D0  30.00D0
  30.00D0  30.00D0  30.00D0  30.00D0  30.00D0
  30.00D0  30.00D0  30.00D0  30.00D0  30.00D0
  30.00D0  30.00D0  30.00D0  30.00D0  30.00D0
  32.95D0  35.90D0  38.85D0  41.80D0  44.74D0
  47.69D0  50.64D0  53.59D0  56.54D0  59.49D0
  62.44D0  65.39D0  68.34D0  71.29D0  74.23D0
  77.18D0  80.13D0  83.08D0  86.03D0  88.98D0
  91.93D0  94.88D0  97.83D0 100.78D0 103.72D0
 106.67D0 109.62D0 112.57D0 115.52D0 118.47D0
 121.42D0 124.37D0 127.32D0 130.27D0 133.21D0
 136.16D0 139.11D0 142.06D0 145.01D0 147.96D0
 150.91D0 153.86D0 156.81D0 159.76D0 162.70D0
 165.65D0 168.60D0 171.55D0 174.50D0 177.45D0
 180.40D0 183.35D0 186.30D0 189.24D0 192.19D0
 195.14D0 198.09D0 200.00D0 200.00D0 200.00D0
 200.00D0 200.00D0 200.00D0 200.00D0 200.00D0
 200.00D0 200.00D0 200.00D0 200.00D0 200.00D0

  30.00D0  30.00D0  30.00D0  30.00D0  30.00D0
  30.00D0  30.00D0  30.00D0  30.00D0  30.00D0
  30.00D0  30.00D0  30.00D0  30.00D0  30.00D0
  30.00D0  30.00D0  30.00D0  30.00D0  30.00D0
  30.00D0  30.00D0  30.00D0  30.00D0  30.00D0
  30.00D0  30.00D0  30.00D0  30.00D0  30.00D0
  30.00D0  30.00D0  30.00D0  30.00D0  30.00D0
  30.00D0  30.00D0  30.00D0  30.00D0  30.00D0
  31.09D0  32.19D0  33.28D0  34.38D0  35.47D0
  36.57D0  37.66D0  38.76D0  39.85D0  40.95D0
  42.04D0  43.14D0  44.23D0  45.33D0  46.42D0
  47.52D0  48.61D0  49.70D0  50.80D0  51.89D0
  52.99D0  54.08D0  55.18D0  56.27D0  57.37D0
  58.46D0  59.56D0  60.65D0  61.75D0  62.84D0
  63.94D0  65.03D0  66.13D0  67.22D0  68.31D0
  69.41D0  70.50D0  71.60D0  72.69D0  73.79D0
  74.88D0  75.98D0  77.07D0  78.17D0  79.26D0
  80.36D0  81.45D0  82.55D0  83.64D0  84.73D0
  85.83D0  86.92D0  88.02D0  89.11D0  90.21D0
  91.30D0  92.40D0  93.49D0  94.59D0  95.68D0
  96.78D0  97.87D0  98.97D0 100.06D0 101.16D0
 102.25D0 103.34D0 104.44D0 105.53D0 106.63D0
 107.72D0 108.82D0 109.91D0 111.01D0 112.10D0
 113.20D0 114.29D0 115.39D0 116.48D0 117.58D0
 118.67D0 119.77D0 120.86D0 121.95D0 123.05D0
 124.14D0 125.24D0 126.33D0 127.43D0 128.52D0
 129.62D0 130.71D0 131.81D0 132.90D0 134.00D0
 135.09D0 136.19D0 137.28D0 138.38D0 139.47D0
 140.56D0 141.66D0 142.75D0 143.85D0 144.94D0
 146.04D0 147.13D0 148.23D0 149.32D0 150.42D0
 151.51D0 152.61D0 153.70D0 154.80D0 155.89D0
 156.98D0 158.08D0 159.17D0 160.27D0 161.36D0
 162.46D0 163.55D0 164.65D0 165.74D0 166.84D0
 167.93D0 169.03D0 170.12D0 171.22D0 172.31D0
 173.41D0 174.50D0 175.59D0 176.69D0 177.78D0
 178.88D0 179.97D0 181.07D0 182.16D0 183.26D0
 184.35D0 185.45D0 186.54D0 187.64D0 188.73D0
 189.83D0 190.92D0 192.02D0 193.11D0 194.20D0
 195.30D0 196.39D0 197.49D0 198.58D0 199.68D0
 200.00D0 200.00D0 200.00D0 200.00D0 200.00D0

'FIXED_FLOWS'
5
5, 2  , 3  , 1  , 1
0.0, 110., 1.01325D5, 0.999, 900.0
0.0, 110., 0.001, 900.0
0.0, 0.0, 0.0, 0.0, 0.0, 1.0, 0.0  
3, 4  , 100, 1  , 1
2, 1  , 1  , 1  , 200
6, 100, 100, 2  , 200
6, 2  , 100, 200, 200

'INITIAL_CONDITIONS'
0.0, 0.0, 1.01325D5, 1. 0.6413, 288.15 
0.0, 0.0
0.0, 0.0, 0.0, 0.0, 0.0, 1.0, 0.0
\end{verbatim}

\newpage
Input file for the simulation of the classical
Sod problem (analogous to the laboratory shock-tube experiment)
used to check numerical discretization schemes on the standard
1D Riemann problem.

\begin{verbatim}

&control
 run_name = 'shock_tube',
 job_type = '2D',
 restart_mode = 'from_scratch',
 time =  0.00,
 tstop = 0.20,
 dt    = 0.001,
 tpr   = 0.05,
 tdump = 5.0,
 nfil = 0
/

&model
 density_specified = T,
 gas_viscosity = F,
 iturb = 0,
 gravz = 0,
 ngas = 1
/

&mesh
  iuni = 1,
  nx = 3,
  nz = 102,
  dx0 = 1.0d-2,
  dz0 = 1.0d-2
/

&particles
  nsolid = 1, 
  diameter = 200.D0,
  density = 1500.D0,
  sphericity = 1.0,
  viscosity = 1.0,
  specific_heat = 1.2D3,
  thermal_conductivity = 2.0D0,
/

&numeric
  rungekut = 1,
  beta = 0.25,
  lim_type = 4,
  muscl = 0,
  omega = 1.00
/

'ROUGHNESS'

'MESH'

'FIXED_FLOWS'
6
2, 1, 3, 102, 102
2, 1, 1, 1, 102
2, 3, 3, 1, 102
1, 2, 2, 2, 31
0.0, 0.75, 1.D0, 1.0, 1.0  ! Last value is the density !
0.0, 0.0, 0.0, 0.0
0.0, 0.0, 0.0, 0.0, 0.0, 1.0, 0.0
1, 2, 2, 32, 101
0.0, 0.0, 1.D-1, 1.0, 0.125  ! Last value is the density !
0.0, 0.0, 0.0, 0.0
0.0, 0.0, 0.0, 0.0, 0.0, 1.0, 0.0
5, 2, 2, 1, 1
0.0, 0.75, 1.D0, 1.0, 1.0  ! Last value is the density !
0.0, 0.0, 0.0, 0.0
0.0, 0.0, 0.0, 0.0, 0.0, 1.0, 0.0

'INITIAL_CONDITIONS'
0.0, 0.0, 1.0D0, 1.0 0.6413, 288.15 
0.0, 0.0
0.0, 0.0, 0.0, 0.0, 0.0, 1.0, 0.0

\end{verbatim}

%\newpage
%Input file for inviscid simulation of a 2D axisymmetric
%supersonic jet of pure water vapour into a tank of still air 
%at atmospheric pressure, without particles. 
%
%\newpage
%Input file for 2D axisymmetric simulation of pyroclastic
%flows generation and propagation on the Northern flank of Vesuvius,
%including the section of the volcano topography.
%
%
\newpage
Input file for fully 3D simulation of a volcanic column,
without topography included.

\begin{verbatim}

&control
 run_name = 'Sub_Plinian',
 job_type = '3D',
 restart_mode = 'from_scratch',
 time =  0.00,
 tstop = 600.0,
 dt    = 0.01,
 tpr   = 10.0,
 tdump = 50.0
/

&model
/

&mesh
  nx = 100,
  nx = 100,
  nz = 100
/

&particles
  nsolid = 1, 
  diameter = 50.
  density = 1500.
  sphericity = 1.0
  viscosity = 0.5
  specific_heat = 1.2D3
  thermal_conductivity = 2.0D0
/

&numeric
  rungekut = 1,
  muscl = 0,
  omega = 1.00
/

'ROUGHNESS'

'MESH'

 200.00D0 200.00D0 196.39D0 192.02D0 187.64D0 
 183.26D0 178.88D0 174.50D0 170.12D0 165.74D0
 161.36D0 156.98D0 152.61D0 148.23D0 143.85D0 
 139.47D0 135.09D0 130.71D0 126.33D0 121.95D0
 117.58D0 113.20D0 108.82D0 104.44D0 100.06D0 
  95.68D0  91.30D0  86.92D0  82.55D0  78.17D0
  73.79D0  69.41D0  65.03D0  60.65D0  56.27D0
  51.89D0  47.52D0  43.14D0  38.76D0  34.38D0
  30.00D0  30.00D0  30.00D0  30.00D0  30.00D0
  30.00D0  30.00D0  30.00D0  30.00D0  23.17D0
  23.17D0  30.00D0  30.00D0  30.00D0  30.00D0
  30.00D0  30.00D0  30.00D0  30.00D0  30.00D0
  34.38D0  38.76D0  43.14D0  47.52D0  51.89D0
  56.27D0  60.65D0  65.03D0  69.41D0  73.79D0
  78.17D0  82.55D0  86.92D0  91.30D0  95.68D0
 100.06D0 104.44D0 108.82D0 113.20D0 117.58D0
 121.95D0 126.33D0 130.71D0 135.09D0 139.47D0 
 143.85D0 148.23D0 152.61D0 156.98D0 161.36D0
 165.74D0 170.12D0 174.50D0 178.88D0 183.26D0 
 187.64D0 192.02D0 196.39D0 200.00D0 200.00D0

 200.00D0 200.00D0 196.39D0 192.02D0 187.64D0 
 183.26D0 178.88D0 174.50D0 170.12D0 165.74D0
 161.36D0 156.98D0 152.61D0 148.23D0 143.85D0 
 139.47D0 135.09D0 130.71D0 126.33D0 121.95D0
 117.58D0 113.20D0 108.82D0 104.44D0 100.06D0 
  95.68D0  91.30D0  86.92D0  82.55D0  78.17D0
  73.79D0  69.41D0  65.03D0  60.65D0  56.27D0
  51.89D0  47.52D0  43.14D0  38.76D0  34.38D0
  30.00D0  30.00D0  30.00D0  30.00D0  30.00D0
  30.00D0  30.00D0  30.00D0  30.00D0  23.17D0
  23.17D0  30.00D0  30.00D0  30.00D0  30.00D0
  30.00D0  30.00D0  30.00D0  30.00D0  30.00D0
  34.38D0  38.76D0  43.14D0  47.52D0  51.89D0
  56.27D0  60.65D0  65.03D0  69.41D0  73.79D0
  78.17D0  82.55D0  86.92D0  91.30D0  95.68D0
 100.06D0 104.44D0 108.82D0 113.20D0 117.58D0
 121.95D0 126.33D0 130.71D0 135.09D0 139.47D0 
 143.85D0 148.23D0 152.61D0 156.98D0 161.36D0
 165.74D0 170.12D0 174.50D0 178.88D0 183.26D0 
 187.64D0 192.02D0 196.39D0 200.00D0 200.00D0

  30.00D0  30.00D0  30.00D0  30.00D0  30.00D0
  30.00D0  30.00D0  30.00D0  30.00D0  30.00D0
  30.00D0  30.00D0  30.00D0  30.00D0  30.00D0
  30.00D0  30.00D0  30.00D0  30.00D0  30.00D0
  30.00D0  30.00D0  30.00D0  30.00D0  30.00D0
  30.00D0  30.00D0  30.00D0  30.00D0  30.00D0
  32.95D0  35.90D0  38.85D0  41.80D0  44.74D0
  47.69D0  50.64D0  53.59D0  56.54D0  59.49D0
  62.44D0  65.39D0  68.34D0  71.29D0  74.23D0
  77.18D0  80.13D0  83.08D0  86.03D0  88.98D0
  91.93D0  94.88D0  97.83D0 100.78D0 103.72D0 
 106.67D0 109.62D0 112.57D0 115.52D0 118.47D0
 121.42D0 124.37D0 127.32D0 130.27D0 133.21D0 
 136.16D0 139.11D0 142.06D0 145.01D0 147.96D0
 150.91D0 153.86D0 156.81D0 159.76D0 162.70D0 
 165.65D0 168.60D0 171.55D0 174.50D0 177.45D0
 180.40D0 183.35D0 186.30D0 189.24D0 192.19D0 
 195.14D0 198.09D0 200.00D0 200.00D0 200.00D0
 200.00D0 200.00D0 200.00D0 200.00D0 200.00D0 
 200.00D0 200.00D0 200.00D0 200.00D0 200.00D0

'FIXED_FLOWS'
7
3, 1  , 100, 1 , 100, 1, 1
5, 49 , 52, 49, 52, 1, 1
0.0, 0.0, 110.00, 1.01325D5, 0.999, 900.0
0.0, 0.0, 110.00, 0.001, 900.0
0.D0, 0.D0, 0.D0, 0.D0, 1.D0, 0.D0, 0.D0
6, 1  , 1  , 1  , 100, 2  , 100
6, 100, 100, 1  , 100, 2  , 100
6, 1  , 100, 1  , 1  , 2  , 100
6, 1  , 100, 100, 100, 2  , 100
4, 1  , 100, 1  , 100, 100, 100

'INITIAL_CONDITIONS'
0.0, 0.0, 1.01325D5, 1. 0.6413, 288.15 
0.0, 0.0
0.0, 0.0, 0.0, 0.0, 0.0, 1.0, 0.0

\end{verbatim}


% Description of how to run PDAC
\newpage
\section{Running \PDAC}
\label{section:run}

\subsection{Getting started: what is needed}

Before running \PDAC\ the following elements are needed:
\begin{itemize}
\item The \PDAC\ executable pdac.x
\item The \PDAC\ parameter file pdac.dat.
\end{itemize}

Here it is assumed that you either succeded in compiling the source
code or that you got a precompiled executable suited for your architecture.
Be sure to have the authorization for running the executable on the
directory where it is installed. For further informations on compiling and 
on code portability, see section \ref{section:avail}.\\

PDAC runs on a variety of serial and parallel platforms.  While it is
trivial to launch a serial program, the execution of a parallel program 
depends on a platform-specific library such as MPI that allows to launch 
copies of the executable on other nodes and to provide access to a high 
performance network such as Myrinet (if available) or to a Ethernet connection.

On all parallel systems, either cluster of workstations or parallel 
supercomputers, \PDAC\ execution is relied to the local implementation
of MPI for data communications, and to standard system tools such as
mpirun to launch jobs. Therefore a working installation of MPI is a 
prerequisite in order to run \PDAC\ in parallel.
The \PDAC\ serial binaries can nevertheless be run directly (this is known 
as {\it standalone} mode) for single process runs, on most platforms
Since different versions of MPI libraries are very often incompatible, you will 
likely need to recompile \PDAC\ to run it in parallel. For serial runs
the provided non-MPI binaries should still work.

\subsection{Individual Unix Workstations}

In this section we will explain how to run \PDAC\ on your own
workstation without a queuing system like
OpenPBS or Loadleveler ( see below on this chapter ).
On individual Unix workstations the use of serial \PDAC\ is
quite easy, basically it behaves like any other Unix process
running on your machine.
To run \PDAC\ then follow the steps:

\begin{itemize}

\item create a running directory\
\begin{verbatim}
> mkdir sample_run
\end{verbatim}

\item copy executable (pdac.x) and input file (pdac.dat)
      to the running directory\
\begin{verbatim}
> cp pdac.x pdac.dat sample_run
\end{verbatim}

\item run pdac.x ( possibly in background )
\begin{verbatim}
> pdac.x &
\end{verbatim}

\end{itemize}

Nonetheless you are the only user of your workstation,
we suggest you to breakup the simulation in shorter partial
simulations ( few days maximum each ) and use the
restart capability od \PDAC\ (see section \ref{section:pdac_res}). 
In this way you reduce the probability of data/work loss due to 
unforseenable events (a system crash, a blackout, ecc...).
When restarting a simulation, remember to set the 
restart mode to ``restart'' and to save the pdac.res file,
in case that something goes wrong in the new simulation step.

For multiprocessor workstations, if you have a working MPI
library, you can compile \PDAC\ in parallel mode to use
all processors of the workstations. In this case,
to run \PDAC\ in parallel you should use an MPI loader 
(usually distributed with the MPI library), i.e. if
your MPI library is MPICH you can run \PDAC\ using the commad:

\begin{verbatim}
> mpirun -np 2 pdac.x
\end{verbatim}

where the command parameter ``-np 2'' specifies that you want
to run on two processors.

\subsection{Server and Parallel Supercomputers}

In this section we explain how to run \PDAC\ on Unix servers
and parallel supercomputers, giving you some practical example
on most popular architectures (at the time of writing).
The main differences you will find, with respect to running on 
individual workstations,
is represented by the presence of a queuing system that
manage the users requests. Usually you should also 
consider the limits the administrator has set to your user,
like disk space, memory or available running queues. 
If this limits are not taken in to account, this could cause
the simulation not to run or complete.
Below, as anticipated, we will give you few examples on how to run
on popular servers and supercomputers, but remember that 
these systems are highly customizable and what is presented 
probably should be modified to run on other supercomputers.

\subsubsection{IBM RS/6000 servers and parallel supercomputers}

On most of IBM servers and RS/6000 supercomputers, to run a simulation you
should write a job script to be submitted to the Loadleveler which
is the IBM queuing system.
Many systems allows you to run MPI program in interactive mode for
testing and debug, through the POE program.  
The options and environment variables for poe are various and arcane, 
so you should consult your local documentation for recommended settings.  
As an example for interactive mode run on the RS/6000 supercomputer
(Sp4) installed at CINECA at the shell prompt you should give the
command:

\begin{verbatim}
> poe ./pdac.x -procs 4 -nodes 1 
\end{verbatim}

to run on a single node using 4 procs.
For long run and simulation it is required that you write a job script,
and submit it to the loadleveler, here there is an example.

\begin{verbatim}
#!/bin/ksh
#@ job_type = parallel
#@ output = job.out
#@ error = job.err
#@ notification = never
#@ checkpoint = no
#@ restart = no
#@ wall_clock_limit =6:00:00
#@ resources = ConsumableCpus(1) ConsumableMemory(1024 mb)
#@ class = parallel
#@ network.MPI = csss,shared,US
#@ total_tasks = 8
#@ blocking = unlimited
#@ queue

cp pdac.dat /scratch/myrundir/pdac.dat
cp pdac.x /scratch/myrundir/pdac.x
cd /scratch/myrundir/
./pdac.x
\end{verbatim}

The above script ask for 8 processors to run 8 MPI tasks
with a limit of 1024Mbyte each for 6 hours on the parallel queue.
Then you should submit the job (suppose its filename is myjobfile):

\begin{verbatim}
> llsubmit myjobfile
\end{verbatim}

\subsubsection{Linux Clusters (Beowulf)}

The most common way to run program on a linux cluster 
is through the use of the combination of OpenPBS, MAUI scheduler
and mpirun/mpiexec MPI loaders .
For this kind of machine you could find an almost infinite number
of environments and system softwares, therefore keep the example below
as an indication, most probably it would not work on other systems. 
As for other servers, you need to write a job script with
the request for the queuing system and with the command you 
want to execute.

\begin{verbatim}
#!/bin/sh
#PBS -l nodes=16:ppn=2,walltime=6:00:00

cp pdac.dat /myrundir
cp pdac.x /myrundir
cd /myrundir
mpiexec -no-shmem -np 32 ./pdac.x
\end{verbatim}

with this script we are asking 16 nodes with 2 processor per node
and 6 hours of execution time in the queue. Then you need to
submit the job to the queuing system, for OpenPBS the command is:

\begin{verbatim}
> qsub myjobscript
\end{verbatim}

here we have supposed that the filename of the above script is
myjobscript.

\subsubsection{Compaq AlphaServer SC}

If your machine as a Quadrics interconnect you should use the 
Elan communication subsystem commands to run your parallel applications
or jobs. The main command is the ``prun'' command, that is similar
to mpirun of MPICH, but with prun you could run both job scripts
and interactive commands. Here there is an example :

\begin{verbatim}
> prun -n 8 ./pdac.x
\end{verbatim}

this command will run pdac.x on 8 processors, as soon as they will
be free, infact if there are other job (or commands) already running
your command will be queued.
There are additional options.  Consult your local documentation.

\subsubsection{SGI Origin systems}

SGI Origin system are shared memory parallel machines, with
real single system image framework, in other words the user
can see or access all the processors directly from the 
login environment. Differently from other architectures,
there is no distinction between login and execution nodes.
This has some benefits and drawbacks: benefits come from
the possibility to use the whole machine as a multiprocessors
workstation with an high number of processors, the drawbacks
are related to the possibility that a different job running
at the same time on the machine interfere with your job.
In general you can run there parallel applications 
in two different way, using a shell command like in your own
workstation and using a queuing system, which is usually NQS.
To run from the shell prompt use mpirun, i.e.: 

\begin{verbatim}
> mpirun -np 8 ./pdac.x
\end{verbatim}

This will execute pdac.x with 8 MPI task, that not necessarily
goes to different processors, it depends on system loads.
Although this way of execute parallel applications is very
friendly, it could cause a lot of interferences with system
activities and other applications, with the result of slowing
down the system. Then, quite often, the system administrator
set severe limits to this way of run, mainly on the number of
real available processors (no matter how many MPI tasks you are asking
for), execution time and memory usage.
Therefore on many systems to run large applications you need to 
interface with the NQS queuing system, so that
you have to write a script with the requests for processors and
memory and the submit the script to NQS.
An example of job script for NQS is 

\begin{verbatim}
#!/bin/sh
#QSUB -l mpp_p=8
#QSUB -l p_mpp_t=2:00:00

cp pdac.dat /myrundir
cp pdac.x /myrundir
cd /myrundir
mpirun -np 8 ./pdac.x
\end{verbatim}

with this script you ask 8 processors for 2 hours, then
you could submit the script with 
(suppose its filename is myjobfile):

\begin{verbatim}
> qsub myjobfile
\end{verbatim}

\subsection{Grid environment}

Few simulations on a Grid environment (such as Condor) have been performed.
Some attention must be payed to the settings of MPI environment in Condor,
to the need of static compiling and to possible errors relied to the
read-write opening of the restart file. Further study will be devoted in
the next future to the possibility of running \PDAC\ on a distributed
system.

\subsection{Memory usage considerations}
The use of parallel supercomputers or PC clusters is mandatory not only
for the purpose of decreasing simulation time, but also since memory
occupation can be to high for a single workstation. As a rough estimate,
consider that, for a ``standard'' 3D simulation (with 2 solid phases) 
\PDAC\ requires a memory occupation of about 1.5 KBytes/cell. 

\subsection{Monitoring parallel speedup and scaling}

While \PDAC\ is designed to be a scalable program, particularly for
large simulations (100,000 cells or more), increasing the number of
processors to a simulation will provide little or no extra performance
above a certain threshold. This is known as ``speedup saturation'' and is
due to the increase of the ratio between the communication and computation
time. If you have access to a parallel machine you should
measure \PDAC\ s parallel speedup for a variety of processor counts when
running your particular simulation. The easiest and most accurate way
to do this is to look at the information contained in the timing report
in the Log files for the different routines. On many parallel architectures
fine-tuned performance monitoring can be performed through appropriate programs.


% How to get and install PDAC
\newpage
\section{\PDAC\ Availability and Installation}
\label{section:avail}

\PDAC\ distribution is governed by the License reported at the beginning
of this manual.

\PDAC\ \PDACVERSION\ is based on the MPI message passing interface
({\tt http://www.mpi-forum.org/})
and on a customized communication layer which have been ported to a wide 
variety of parallel platforms.
This section describes how to obtain and install \PDAC\ \PDACVERSION.

\subsection{How to obtain \PDAC}

\PDAC\ may be downloaded from \PDACURL.
You will be required to provide minimal registration information and
agree to a license before receiving access to the software.
Both source and binary distributions are available.

\subsection{Platforms on which \PDAC\ will currently run}

\PDAC\ should be portable to any serial platform with a
Fortran90 and C compilers and to any parallel platform with
the above compilers and the MPI library.
Precompiled \PDAC\ \PDACVERSION\ binaries are available for the following platforms:  
\begin{itemize}
\item Linux on Intel
\item Mac OS X (also called Darwin) on PowerPC processors
\item AIX on RS/6000 processors 
\item HP-UX on PA-RISC processors 
\item Solaris on Sparc processors (with and without MPI)
\item Tru64 Unix on Alpha processors (with and without MPI)
\item IBM RS/6000 SP (using MPI)
\item Compaq AlphaServer SC (using the Quadrics Elan library)
\item SGI Origin 2000 (with and without MPI) 
\end{itemize}

\subsection{Compiling \PDAC}

As mentionead above, we provide binaries for all
platforms to which \PDAC\ has been ported. However, it is 
recommended to recompile the code on your system, 
since hardware is changing rapidly and most probably 
your native Fortran90 compiler will optimize
the code further. 
On Linux parallel platform with MPI, recompiling 
is in practice mandatory, since there are too
many possible combinations of MPI libraries and networking
hardware for which the executables are incompatible.
Recompiling is also required if you wish  
to add or modify features of \PDAC.

\subsubsection{Directory structure}

The \PDAC\ source code is distributed with a 
directory structure with two levels, a single
main directory named \DIR{EXPLORIS} (in the official distribution), 
and several sub-directories.
In what follows the different directories are listed
and their content commented.

\begin{itemize}

\item directory \DIR{EXPLORIS}: 
      this is the main directory and contains the Fortran90 source files
      of the \PDAC\ main subroutines. This directory contains also the 
      \FIL{Makefile} and the \FIL{Machine} file (see below).

\item directory \DIR{EXPLORIS/comm}:
      this subdirectory contains the \PDAC\ communication layer,
      interfaced to MPI and few wrappers for system dependent
      features. In general, all low level subroutines that 
      need preprocessing are in this directory.

\item directory \DIR{EXPLORIS/doc}:
      this directory contains the \PDAC\ documentation and this manual.

\item directory \DIR{EXPLORIS/examples}:
      this directory contains several input files examples.
      
\item directory \DIR{EXPLORIS/utility}:
      this directory contains few utility scripts and programs
      to help compiling and porting the code.

\end{itemize}

\subsubsection{Files \FIL{Machine} and \FIL{Makefile}}

\PDAC\ source code comes with a makefile (named \FIL{Makefile} in the distribution)
that takes care of the compilation of \PDAC\ itself.
The \FIL{Makefile} file contains general commands that should be quite general
and platform independent. There are other \FIL{Makefile} files in the 
subdirectories to compile libraries, utilities and documentations.
All \FIL{Makefile} files include a configuration file named \FIL{Machine},
which contains all the system dependent parameters for the compilation,
like the compiler name and the compilations flags.
In the main directory there are several \FIL{Machine.arch} files already prepared
for a number of existing architectures. 
The \FIL{Machine.arch} files distributed with the source code
are configured and tested on the corresponding architecture, although your system
could have some particular feature which requires some change in the 
\FIL{Machine.arch} file, in this case you should edit the file to make the
proper changes. 
The main \FIL{Makefile} file includes also a hidden file (\FIL{.dependencies}) 
containing all the dependencies among the source Fortran90 files. The \PDAC\ 
source code is infact written using Fortran90 modules and source files
should be compiled in the proper order, determined by the dependencies
among modules. The file \FIL{.dependencies} is generated by the \FIL{shdep} script
distributed togheter with the code (see next section for more details ).

\subsubsection{Making dependencies and compiling}

All steps required to compile the code are now listed and commented.

\begin{itemize}

\item Edit \FIL{Machine} file

      The very first thing you should do to compile \PDAC\ is to choose
      the \FIL{Machine} file for your architecture among those distributed with
      the source code and named \FIL{Machine.arch}, where the "arch" suffix
      vary on the list of the supported architectures. For example, if
      your machine is an IBM Sp4, you should copy the file \FIL{Machine.sp4}
      onto the \FIL{Machine} file. If there isn't the \FIL{Machine.arch} for 
      your architecture, choose the most close among those present
      and edit it to set the compiling parameters to the proper values.
      Once you have find your \FIL{Machine} file you should copy it into the
      file named \FIL{Machine} with the following command:\\
      {\tt >}\\
      {\tt > cp Machine.arch Machine}\\
      {\tt >}\\
      Then you can edit \FIL{Machine} to change compiling parameters,
      such as the compiler options or the executable type (serial or parallel).
      For a serial executable you should unset the -D\_\_MPI precompiler macro
      subsituting it with -D\_\_SERIAL, and substitute the parallel compilers
      with the scalar one, changing the variables FC, MPIFC and LINKER
      in the \FIL{Machine} file.

\item Build the \FIL{.dependencies} file

      Before compiling \PDAC\ executable you shold build the 
      dependencies file (.dependencies). To do this, you should
      move to the directory  EXPLORIS/utility and type \\
      {\tt > make all}\\
      then, supposing that you are in the EXPLORIS directory, 
      the exact sequence of command is:\\
      {\tt > cd utility}\\
      {\tt > make all}\\
      {\tt > cd ..}\\

\item Compiling \FIL{pdac.x} and \FIL{pp.x}

      At this point everything is ready to compile the \PDAC\ executable
      (\FIL{pdac.x}) and the post processing executable (\FIL{pp.x}), from
      the EXPLORIS directory type:\\
      {\tt > make pdac.x}\\
      {\tt >}\\
      {\tt > make pp.x}\\

\item Compiling this manual

      To compile the manual, you must have installed a Tex packages
      with standard \LaTeX\ extensions support, 
      then type the following commands:\\
      {\tt >}\\
      {\tt > cd doc}\\
      {\tt > latex pdac\_ug.tex}\\
      {\tt > latex pdac\_rm.tex}\\
      {\tt > dvips pdac\_ug.dvi -o pdac\_ug.ps }\\
      {\tt > dvips pdac\_rm.dvi -o pdac\_rm.ps }\\

\end{itemize}

\subsection{Documentation}

All available \PDAC\ documentation is available for downloading without
registration via the \PDAC\ web site \PDACURL.



\newpage
\addtocontents{toc}{\contentsline {section}{References}{\thepage}}
\bibliographystyle{abbrv}
%\bibliography{group,tbpub,ug}
\bibliography{ug}

\newpage
\addtocontents{toc}{\contentsline {section}{Index}{\thepage}}
\printindex

% that's all, folks
\end{document}
