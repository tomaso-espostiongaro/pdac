\section{Data and modules}
The computer code PDAC is organized in a modular structure.

Modules can contain only the definition and type declaration
of global array variables (as velocities, pressure, densities or
temperatures) and their allocation routines. These modules allow
global variables to be used by different program routines. They are
used instead of the {\em deprecated} {\tt COMMON} block of Fortran77, improving
the compile-time controls, the portability, and the data-flow control.

Other modules refer to an independent block of the algorithm, and then contain
variable definitions, constants, procedures that are needed only
within the scope of the module (variables that can be declared as {\tt PRIVATE}). 
These include, for example, the module for boundary conditions, computation 
of fluxes, invertion of the multiphase matrix.

In general, modules contain the definition of variables 
and the procedures acting on these variables (these include,
for example, the definition of new data-types for the definition of 
the parallel environment variables or the setting of the grid, but also
specific functions within the program, as those for computing the atmospheric 
profile, or the turbulence model or particles properties.)

 This modular structure is particularly suited for global
data that are needed in many steps of computation and to give a more 
coherent organization of the data in the program flow.

A short description of modules is given below, with the description of the main 
variable used and the routines acting on them. Variables that are used only within one
module or a routine are not necessarily presented here.
Modules are presented in alphabetical order with their name in {\bf bold}, whereas the corresponding
file name is in parentheses. 
Variables declarations are indicated in {\tt typewriter} font in tables, 
with the array dimensions as arguments, whereas subroutines and functions
are presented by using the {\em emphasized} font.\\[5mm]
%
%%%%%%%%%%%%%%%%%%%%%%%%%%%%%%%%%% A T M O S P H E R E %%%%%%%%%%%%%%%%%%%%%%%%%%%%%%%%%%%%%%%%%%%%%%%%%
%
{\large {\bf atmosphere}} (atmosphere.f)\\[5mm]
\begin{tabular}{|p{6cm}|p{6cm}|} \hline
VARIABLE & MEANING\\\hline
\tt u0         & initial gas velocity in the atmosphere ($r(x)$ component)\\\hline
\tt v0         & $y$ component\\\hline
\tt w0         & $z$ component\\\hline
\tt us0        & initial particles velocity in the atmosphere ($r(x)$ component)\\ \hline
\tt vs0        & $y$ component\\\hline
\tt ws0        & $z$ component\\\hline
\tt p0         & pressure at ground level at $t=0$\\ \hline
\tt temp0      & temperature at ground level at $t=0$\\ \hline
\tt ep0        & initial particle volumetric fraction in the atmosphere\\\hline
\tt epsmx0     & maximum packing volumetric fraction \\\hline
\tt gravx, gravy, gravz & $r$(or $x$), $y$ and $z$ components of gravitational acceleration\\ \hline
\end{tabular}\\

\begin{itemize}
\item {\em atm} computes the vertical stratification of the atmosphere using 
 empirical coefficients
\end{itemize}
%
%%%%%%%%%%%%%%%%%%%%%%%%%%%%%%%%%% B O U N D A R Y   C O N D I T I O N S %%%%%%%%%%%%%%%%%%%%%%%%%%%%%%%
%
{\large {\bf boundary\_conditions}} (bdry.f)\\

\begin{itemize}
\item {\em boundary} computes boundary conditions by checking every cells neighbours 
\item {\em inoutflow} computes the free inflow/outflow conditions for East, North, and Top boundaries
\item {\em outinflow} computes the free inflow/outflow conditions for West, South, and Bottom boundaries
\end{itemize}
%
%%%%%%%%%%%%%%%%%%%%%%%%%%%%%%%%%% C O N T R O L   F L A G S %%%%%%%%%%%%%%%%%%%%%%%%%%%%%%%%%%%%%%%%%%%
%
{\large {\bf control\_flags}} (control.f)\\[5mm]
\begin{tabular}{|p{6cm}|p{6cm}|}\hline
VARIABLE & MEANING\\\hline
\tt  job\_type    & job type ('2D' or '3D') \\\hline
\tt  nfil         & number of the first OUTPUT file \\\hline
\end{tabular}\\[5mm]
%
%%%%%%%%%%%%%%%%%%%%%%%%%%%%%%%%%% D O M A I N   D E C O M P O S I T I O N  %%%%%%%%%%%%%%%%%%%%%%%%%%%%
%
{\large{\bf domain\_decomposition}} (decomp.f)\\[5mm]
\begin{tabular}{|p{6cm}|p{6cm}|}\hline
VARIABLE & MEANING\\\hline
\tt mesh\_partition   & \\ \hline
\tt layer\_map   & \\ \hline
\tt block2d\_map & \\ \hline
\tt block3d\_map & \\ \hline
\tt rcv\_map   & \\ \hline
\tt snd\_map & \\ \hline
\tt proc\_map & \\ \hline
\tt ncint    & number of cells belonging to one processor\\ \hline
\tt ncext    & number of ghost cells of one processor\\ \hline
\tt ncdom    & total number of cells of one processor\\ \hline
\tt myij(:,ncint)   &  mapping of neighbour cells on local indexes\\ \hline
\tt myinds(:,ncint) &  mapping of neighbour cells including boundary conditions\\ \hline
\end{tabular}\\[5mm]
\begin{itemize}
\item{\em partition} distributes computational cells among processors following the 
prescribed strategy, and builds the processors map, that is, a function that
 for each cells, identified by its global index, gives the number of 
processor it owns and the local index in the processor sub-domain (and viceversa);\\
\item{\em ghost}, given the processor maps, identifies each processor boundaries 
(ghost cells) and
allocates the buffers to be sent or received in the communication calls;\\
\item{\em data\_exchange} invokes MPI send-receive calls to exchange communication 
 buffers \\
\item{\em data\_collect} 
\item{\em data\_distribute} 
\end{itemize}
%
%%%%%%%%%%%%%%%%%%%%%%%%%%%%%%%%%% G A S   S O L I D   D E N S I T Y %%%%%%%%%%%%%%%%%%%%%%%%%%%%%%%%%%%
%
{\large {\bf gas\_solid\_density}} (dens.f)\\[5mm]
\begin{tabular}{|p{6cm}|c|p{6cm}|}\hline
VARIABLE & REPRESENTATION & MEANING\\\hline
\tt rog(ncint) & $(\rho_g)_{ij}=\frac{P_{ij}m}{R_m(T_g)_{ij}}$ &  Thermodynamic gas density \\\hline
\tt rgp(ncdom) & $(\rho_g')_ij=(\epsilon_g\rho_g)_ij $&  bulk gas density \\\hline
\tt rgpn(ncint) & $(\rho_g')^{n}_ij=(\epsilon_g\rho_g)^{n}_ij $&  bulk gas density at time $n$ \\\hline
\tt rlk(ncl,ncdom) & $(\rho_k')_{ij}=(\epsilon)_{ij}\rho_k$ &  bulk density of solid phase $k$ \\\hline
\tt rlkn(ncl,ncint) & $(\rho_k')^{n}_{ij}=(\epsilon)^{n}_{ij}\rho_k$ &  bulk density of solid phase $k$ at time $n$\\\hline
\end{tabular}\\[5mm]
%
%%%%%%%%%%%%%%%%%%%%%%%%%%%%%%%%%% D I M E N S I O N S %%%%%%%%%%%%%%%%%%%%%%%%%%%%%%%%%%%%%%%%%%%%%%%%%
%
{\large {\bf dimensions}} (dimensions.f)\\[5mm]
\begin{tabular}{|p{6cm}|p{6cm}|}\hline
VARIABLE & MEANING\\\hline
\tt  nx    & number of cells in $r$ or $x$ direction\\\hline
\tt  ny    & number of cells in $y$ direction\\\hline
\tt  nz    & number of cells in $z$ direction\\\hline
\tt  ntot  & total number of cells \\\hline
\tt  no   & number of blocks for the definition of cells types\\\hline
\tt  nsolid    & number of solid phases\\\hline
\tt  ngas   & number of gas components\\\hline
\tt  nphase & number of phases\\\hline
\tt  nroughx = 2 & number of different roughness zones\\\hline
\end{tabular}\\[5mm]
%
%%%%%%%%%%%%%%%%%%%%%%%%%%%%%%%%%% M O M E N T U M   T R A N S F E R %%%%%%%%%%%%%%%%%%%%%%%%%%%%%%%%%%%
%
{\large{\bf momentum\_transfer}} (drag.f)\\
\begin{itemize}
\item{\em kdragg} computes the gas-particles drag coefficient for $\epsilon_g \geq 0.8$;
\item{\em kdragl} computes the gas-particles drag coefficient for $\epsilon_g < 0.8$;
\item{\em kdrags} computes gas-particles drag coefficient;
\item{\em ppdrag} computes particle-particles drag coefficient;
\item{\em inter} computes particle-particle drag coefficient.
\end{itemize}
%
%%%%%%%%%%%%%%%%%%%%%%%%%%%%%%%%%% E O S   G A S  %%%%%%%%%%%%%%%%%%%%%%%%%%%%%%%%%%%%%%%%%%%%%%%%%%%%%%
%
{\large {\bf eos\_gas}} (eosg.f)\\[5mm]
\begin{tabular}{|p{6cm}|c|p{6cm}|}\hline
VARIABLE & REPRESENTATION & MEANING\\\hline
\tt cg(ncint)          & $(C_{P_g})_{ij}^n=\sum\limits_{kg=1}^{\tt ngas}(C_{P_kg})_{ij}^n$ &  gas specific heat at constant P\\\hline
\tt ygc(ngas,ncint)    & $(y_i)_{ij}^n$&  mass fraction of gas species $i$\\ \hline
\tt xgc(ngas,ncint)    & $(\chi_i)_{ij}^n$&  molar fraction of gas species $i$\\ \hline
\tt rgpgc(ngas,ncdom)  & $(y_{i}\epsilon_g\rho_g)_{ij}$& bulk density of gas species $i$ \\ \hline
\tt rgpgcn(ngas,ncint) & $(y_{i}\epsilon_g\rho_g)_{ij}^{n}$& bulk density of gas species $i$ at time $n$ \\ \hline
\end{tabular}\\

\begin{itemize}
\item {\em mole} computes the molar fraction of gas components from mass fraction;\\
\item {\em mas} computes the mass fraction of gas components from molar fraction;\\
\item {\em csound} computes the adiabatic (squared) gas speed of sound;\\
\item  {\em thermal\_eosg}computes the mixture gas density from pressure and temperature 
using the equation of state;\\
\item  {\em caloric\_eosg} computes the temperature using enthalpy;
\item  {\em cnvertg} computes thermodynamic mean quantitites at t=0;
\end{itemize}
%
%%%%%%%%%%%%%%%%%%%%%%%%%%%%%%%%%% E O S   S O L I D %%%%%%%%%%%%%%%%%%%%%%%%%%%%%%%%%%%%%%%%%%%%%%%%%%
%
{\large{\bf eos\_solid}} (eosl.f)\\
\begin{itemize}
\item {\em caloric\_eosl} computes solid temperatures from enthalpies;\\
\item {\em converts} computes thermodynamic mean quantitites at t=0;\\
\end{itemize}
%
%%%%%%%%%%%%%%%%%%%%%%%%%%%%%%%%%% C O N V E C T I V E   M A S S   F L U X E S %%%%%%%%%%%%%%%%%%%%%%%%
%
{\large{\bf convective\_mass\_fluxes}} (flux\_m.f)\\
\begin{itemize}
\item {\em masf} computes the mass fluxes of gas and particles by using donor-cell scheme;\\
\item {\em fmas} computes the mass fluxes of gas and particles by using muscl schemes;\\
\end{itemize}
%
%%%%%%%%%%%%%%%%%%%%%%%%%%%%%%%%%% C O N V E C T I V E   F L U X E S   S C %%%%%%%%%%%%%%%%%%%%%%%%%%%%
%
{\large{\bf convective\_fluxes\_sc}} (flux\_sc.f)\\
\begin{itemize}
\item {\em fsc} computes the convective fluxes of gas and particles enthalpy (or any other scalar field)
 by using muscl schemes;\\
\end{itemize}
%
%%%%%%%%%%%%%%%%%%%%%%%%%%%%%%%%%% C O N V E C T I V E   F L U X E S   U %%%%%%%%%%%%%%%%%%%%%%%%%%%%%%
%
{\large{\bf convective\_fluxes\_u}} (flux\_u.f)\\
\begin{itemize}
\item {\em flu} computes the x(r)-momentum fluxes of gas and particles by using muscl schemes;\\
\end{itemize}
%
%%%%%%%%%%%%%%%%%%%%%%%%%%%%%%%%%% C O N V E C T I V E   F L U X E S   V %%%%%%%%%%%%%%%%%%%%%%%%%%%%%%
%
{\large{\bf convective\_fluxes\_v}} (flux\_v.f)\\
\begin{itemize}
\item {\em flv} computes the y-momentum fluxes of gas and particles by using muscl schemes;\\
\end{itemize}
%
%%%%%%%%%%%%%%%%%%%%%%%%%%%%%%%%%% C O N V E C T I V E   F L U X E S   W %%%%%%%%%%%%%%%%%%%%%%%%%%%%%%
%
{\large{\bf convective\_fluxes\_w}} (flux\_w.f)\\
\begin{itemize}
\item {\em flw} computes the z-momentum fluxes of gas and particles by using muscl schemes;\\
\end{itemize}
%
%%%%%%%%%%%%%%%%%%%%%%%%%%%%%%%%%% E N T H A L P Y   M A T R I X %%%%%%%%%%%%%%%%%%%%%%%%%%%%%%%%%%%%%%
%
{\large{\bf enthalpy\_matrix}} (ftem.f)\\[5mm]
\begin{tabular}{|p{6cm}|c|p{6cm}|}\hline
VARIABLE & REPRESENTATION & MEANING\\\hline
\tt bt(nphase)  & $({\bf B_H})_{i+\frac{1}{2},j}$ & explicit vector 
of enthalpies \\ \hline
\tt at(nphase,nphase) & $({\bf A_H})_{i+\frac{1}{2},j}$ & phase matrix of enthalpies\\\hline
\tt hv               & $(Q_k)_{ij}^n$ & gas-particles temperature exchange coefficient\\\hline
\end{tabular}\\
\begin{itemize}
\item{\em ftem} computes the elements of the enthalpies phases matrix\\
\item{\em invdm} inverts the matrix\\
\end{itemize}
%
%%%%%%%%%%%%%%%%%%%%%%%%%%%%%%%%%% G A S   C O N S T A N T S %%%%%%%%%%%%%%%%%%%%%%%%%%%%%%%%%%%%%%%%%%
%
{\large {\bf gas\_constants}} (gas.f)\\[5mm]
\begin{tabular}{|p{6cm}|c|p{6cm}|}\hline
VARIABLE & REPRESENTATION & MEANING\\\hline
\tt  ckg(ngas)    & $k_g(T)$ & temperature-dependent thermal conductivity of gas components\\\hline
\tt  mmug(ngas)   & $\mu_g$  & temperature-dependent viscosity of gas components\\\hline
\tt  mmugs(ngas)  & $\sigma$ & Lennard-Jones potential\\\hline
\tt  mmugek(ngas) & $\epsilon_k$ & Lennard-Jones potential\\\hline
\tt  gmw(ngas)    & $m_g$    & molecular weight of gas components\\\hline
\tt  phij(:,:)    & $\phi_{i,j}$ & weight function for the mixture mean viscosity/conductivity\\\hline
\tt  present\_gas(ngas)  & & logical control  \\\hline
\tt  default\_gas        & & default gas component in atmosphere \\\hline
\tt  gammaair         & $\gamma$ & air adiabatic constant \\\hline
\tt  c\_joule, c\_erg & & conversion factors (calories/joules, calories/erg) \\\hline
\tt  rgas             & $R$ & universal gas constant \\\hline
\tt  tzero            & $T_0$ & reference temperature \\\hline
\tt  hzerog, hzeros   &$h_{0g}$, $h_{0s}$& gas and solid enthalpies at reference temperature \\\hline
\end{tabular}\\[5mm]
%
%%%%%%%%%%%%%%%%%%%%%%%%%%%%%%%%%% G R I D %%%%%%%%%%%%%%%%%%%%%%%%%%%%%%%%%%%%%%%%%%%%%%%%%%%%%%%%%%%%
%
{\large{\bf grid}} (grid.f)\\[5mm]
\begin{tabular}{|p{6cm}|p{6cm}|}\hline
VARIABLE & MEANING\\\hline
\tt x(nx), xb(nx) & physical x or radial coordinates of the cell centers and borders \\ \hline
\tt zb(nz) & physical vertical coordinates of the cell borders \\ \hline
\tt dx(nx) & cell dimensions in $x(r)$-direction\\ \hline
\tt dy(ny) & cell dimensions in $y$-direction\\ \hline
\tt dz(nz) & cell dimensions in $z$-direction\\ \hline
\tt inx(nx), inxb(nx) & inverse of $x$-coordinates \\ \hline
\tt inxb(nx) & inverse of $xb$-coordinates \\ \hline
\tt indx(nx) & inverse of cells $x$-dimension\\ \hline
\tt indy(ny) & inverse of cells $y$-dimension\\ \hline
\tt indz(nz) & inverse of cells $z$-dimension\\ \hline
\tt itc       & input flag for cylindrical or cartesian coordinates\\ \hline
\end{tabular}\\[5mm]
\begin{itemize}
\item{\em grid\_setup} computes the dimensions of the physical domain\\
\item{\em flic} assigns a type flag to each cell\\
\end{itemize}
%
%%%%%%%%%%%%%%%%%%%%%%%%%%%%%%%%%% S P E C I F I C   H E A T   M O D U L E %%%%%%%%%%%%%%%%%%%%%%%%%%%%
%
{\large{\bf specific\_heat\_module}} (hcapgs.f)\\[5mm]
\begin{tabular}{|p{6cm}|c|p{6cm}|}\hline
VARIABLE & REPRESENTATION & MEANING\\\hline
\tt cp(ngas,ncint) & $((C_P)_{kg})^n_{ij}$ & temperature-dependent thermal capacity of gas species $kg$\\\hline
\tt ck(ncl,ncint) & $((C)_{k})^n_{ij}$ & thermal capacity of particle class $k$\\\hline
\end{tabular}\\
\begin{itemize}
\item{\em hcapg} computes thermal capacities of gas components 
\item{\em hcaps} computes thermal capacities of particles 
\end{itemize}
%
%%%%%%%%%%%%%%%%%%%%%%%%%%%%%%%%%% D I F F U S I V E   F L U X E S %%%%%%%%%%%%%%%%%%%%%%%%%%%%%%%%%%
%
{\large{\bf heat\_diffusion}} (hotc.f)\\
\begin{itemize}
\item{\em hotc} computes thermal conductivity of gas or particles;
\end{itemize}
%
%%%%%%%%%%%%%%%%%%%%%%%%%%%%%%%%%% T I L D E   E N E R G Y %%%%%%%%%%%%%%%%%%%%%%%%%%%%%%%%%%%%%%%%%%
%
{\large {\bf tilde\_energy}} (htilde.f)\\[5mm]
\begin{tabular}{|p{6cm}|c|p{6cm}|}\hline
VARIABLE & REPRESENTATION & MEANING\\\hline
\tt rhg(ncint) & $\widetilde{(\rho_g'h_g)_{ij}^n}$ & explicit (tilde) terms in the gas enthalpy equation\\\hline
\tt rhs(ncl,ncint)& $\widetilde{(\epsilon_s \rho_s h_s)_{ij}^n}$ & explicit (tilde) terms in the particles enthalpy equation\\\hline
\tt egfe(ncdom) & $\left[ (\rho_g')h_g u_g x \right]_{E}$ &EAST convective flux of gas enthalpy\\\hline
\tt egfn(ncdom) & $\left[ (\rho_g')h_g v_g \right]_{N}$ &NORTH convective flux of gas enthalpy\\\hline
\tt egft(ncdom) & $\left[ (\rho_g')h_g w_g \right]_{T}$ &TOP convective flux of gas enthalpy\\\hline
\tt esfe(ncl,ncdom)& $\left[ \epsilon_s\rho_sh_su_s x\right]_{E}$ &EAST convective flux of particles enthalpy\\\hline
\tt esfn(ncl,ncdom)& $\left[ \epsilon_s\rho_sh_sv_s \right]_{N}$ &NORTH convective flux of particles enthalpy\\\hline
\tt esft(ncl,ncdom)& $\left[ \epsilon_s\rho_sh_sw_s \right]_{T}$ &TOP convective flux of particles enthalpy\\\hline
\tt hgfe(ncdom)&$ (\epsilon_g \kappa_g \nabla T_gx)_{E}$ &EAST diffusive flux of gas enthalpy\\\hline
\tt hgfn(ncdom)&$ (\epsilon_g \kappa_g \nabla T_g)_{N}$ &NORTH diffusive flux of gas enthalpy\\\hline
\tt hgft(ncdom)&$ (\epsilon_g \kappa_g \nabla T_g)_{T}$ &TOP diffusive flux of gas enthalpy\\\hline
\tt hsfe(ncl,ncdom)&$ (\epsilon_s \kappa_s \nabla T_sx)_{E}$ &EAST diffusive flux of particles enthalpy\\\hline
\tt hsfn(ncl,ncdom)&$ (\epsilon_s \kappa_s \nabla T_s)_{N}$ &NORTH diffusive flux of particles enthalpy\\\hline
\tt hsft(ncl,ncdom)&$ (\epsilon_s \kappa_s \nabla T_s)_{T}$ &TOP diffusive flux of particles enthalpy\\\hline
\end{tabular}\\
\begin{itemize}
\item{\em htilde} computes the explicit (tilde) terms in the enthalpy transport equation for gas and particles; 
convective fluxes are calculated by using the upwinding technique ({\em convective\_fluxes\_sc} module); 
diffusive fluxes are computed by central differencing ({\em heat\_diffusion} module)\\
\end{itemize}
%
%%%%%%%%%%%%%%%%%%%%%%%%%%%%%%%%%% H E A T   T R A N S F E R %%%%%%%%%%%%%%%%%%%%%%%%%%%%%%%%%%%%%%%%
%
{\large{\bf heat\_transfer}} (hvs.f) \\
\begin{itemize}
\item{\em hvs} computes the gas-particles interphase heat transfer coefficient.
\end{itemize}
%
%%%%%%%%%%%%%%%%%%%%%%%%%%%%%%%%%% I N D I J K   M O D U L E %%%%%%%%%%%%%%%%%%%%%%%%%%%%%%%%%%%%%%%%
%
{\large{\bf indijk\_module}} (indijk.f) \\
\begin{itemize}
\item{\em indijk\_setup} Set the numeration order and the position of the stencil elements in 3D
\end{itemize}
%
%%%%%%%%%%%%%%%%%%%%%%%%%%%%%%%%%% I N P U T   M O D U L E %%%%%%%%%%%%%%%%%%%%%%%%%%%%%%%%%%%%%%%%%%
%
%
%%%%%%%%%%%%%%%%%%%%%%%%%%%%%%%%%% I 0  R E S T A R T %%%%%%%%%%%%%%%%%%%%%%%%%%%%%%%%%%%%%%%%%%%%%%%
%
{\large{\bf io\_restart}} (io.f)\\
\begin{itemize}
\item{\em tapewr} writes the restart file
\item{\em taperd} reads the restart file
\item{\em tapebc} calls the MPI BroadCast function to distribute data among processors
\item{\em write\_array} collect data from different processors and write output file
\item{\em read\_array}  read output or restart file and  distribute data among processors
\end{itemize}
%
%%%%%%%%%%%%%%%%%%%%%%%%%%%%%%%%%% I T E R A T I V E   S O L V E R %%%%%%%%%%%%%%%%%%%%%%%%%%%%%%%%%%
%
{\large{\bf iterative\_solver}} (iter.f)\\[5mm]
\begin{tabular}{|p{6cm}|c|p{6cm}|}\hline
VARIABLE & REPRESENTATION & MEANING\\\hline
\tt rgfe(ncdom) & $\left[ (\rho_g')u_gx \right]_{E} $ & convective gas mass flux in $x(r)$ direction\\\hline
\tt rgfn(ncdom) & $\left[ (\rho_g')v_g \right]_{N} $ & convective gas mass flux in $y$ direction\\\hline
\tt rgft(ncdom) & $\left[ (\rho_g')w_g \right]_{T} $ & convective gas mass flux in $z$ direction\\\hline
\tt rsfe(ncl,ncdom) & $\left[ \epsilon_s\rho_su_sx \right]_{E} $ & convective particles mass flux in $x(r)$ direction\\\hline
\tt rsfn(ncl,ncdom) & $\left[ \epsilon_s\rho_sv_s \right]_{N} $ & convective particles mass flux in $y$ direction\\\hline
\tt rsft(ncl,ncdom) & $\left[ \epsilon_s\rho_sw_s \right]_{T} $ & convective particles mass flux in $z$ direction\\\hline
\tt conv(ncint) & $10^{-8}\times(\rho_g')_{ij}$ & convergence criterion\\\hline
\tt dg(ncint) & $(D_g)_{ij}$ & residual of the gas mass conservation equation \\\hline
\tt abeta(ncint) & $\left(\frac{dD_g}{dP}\right)_{ij}$ & derivative of the mass residual for the Newton solver\\\hline
\tt omega & $\omega$ & overrelaxation parameter \\\hline
\tt inmax, maxout && maximum number of inner and outer iteration\\\hline
\end{tabular}\\
\begin{itemize}
\item{\em iter} is the iterative kernel of the solution algorithm;
\item{\em padjust} adjusts pressure using the Newton's and the secant methods;
\item{\em newp} computes the new pressure using the bi-secant method;
\item{\em betas} computes the derivative of the mass residual and the convergence parameter.
\end{itemize}
%
%%%%%%%%%%%%%%%%%%%%%%%%%%%%%%%%%% F L U X   L I M I T E R S %%%%%%%%%%%%%%%%%%%%%%%%%%%%%%%%%%%%%%%%%%
%
{\large{\bf flux\_limiters}} (limiters.f)\\[5mm]
\begin{tabular}{|p{6cm}|p{6cm}|}\hline
VARIABLE &  MEANING\\\hline
\tt  muscl    & activate muscl procedure for upwinding \\\hline
\tt  beta     & upwinding coefficient \\\hline
\tt  lim\_type & type of limiters \\\hline
\end{tabular}\\
\begin{itemize}
\item{\em limiter} compute the value of flux limiter by including different high order upwinding schemes
\end{itemize}
%
%%%%%%%%%%%%%%%%%%%%%%%%%%%%%%%%%% P H A S E S   M A T R I X %%%%%%%%%%%%%%%%%%%%%%%%%%%%%%%%%%%%%%%%%%
%
{\large{\bf phases\_matrix}} (matrix.f)\\[5mm]
\begin{tabular}{|p{6cm}|c|p{6cm}|}\hline
VARIABLE & REPRESENTATION & MEANING\\\hline
\tt bu1,bv1,bw1 & $({\bf B_U},{\bf B_V})_{i-1/2,j-1/2}$ & explicit vectors of W, S, B momenta\\\hline
\tt bu, bv & $({\bf B_U}, {\bf B_V})_{i+1/2,j+1/2}$ & explicit vectors of E, N, T momenta\\\hline
\tt au1, av1 &  $({\bf A_U}, {\bf A_V})_{i-1/2,j-1/2}$ & phase matrix of W, S, B momenta\\\hline
\tt au, av &  $({\bf A_U}, {\bf A_V})_{i+1/2,j+1/2}$ &phase matrix of E, N, T momenta\\\hline
\tt rlim && lowest limit for particle concentration \\\hline
\end{tabular}\\
\begin{itemize}
\item{\em assemble\_all\_matrix} computes all matrix elements\\
\item{\em assemble\_matrix} computes only RIGHT and TOP matrix elements\\
\item{\em solve\_all\_velocities} inverts the phase matrix using Newton method\\
\item{\em solve\_velocities} inverts the reduced phase matrix using Newton's method\\
\end{itemize}
%
%%%%%%%%%%%%%%%%%%%%%%%%%%%%%%%%%% O U T P U T   D U M P %%%%%%%%%%%%%%%%%%%%%%%%%%%%%%%%%%%%%%%%%%%%%%
%
{\large{\bf output\_dump}} (outp.f)\\
\begin{itemize}
\item{\em outp} dumps output variables by using the {\em write\_array} procedure defined in the {\em io\_restart}
 module
\end{itemize}
%
%%%%%%%%%%%%%%%%%%%%%%%%%%%%%%%%%% P A R T I C L E S   C O N S T A N T S %%%%%%%%%%%%%%%%%%%%%%%%%%%%%%
%
{\large{\bf particles\_constants}} (particles.f)\\[5mm]
\begin{tabular}{|p{6cm}|c|p{6cm}|}\hline
VARIABLE & REPRESENTATION & MEANING\\\hline
\tt dk(ncl) & $d_k$ &  particle diameters\\\hline
\tt rl(ncl) & $\rho_k$ & specific particle density\\\hline
\tt inrl(ncl)& $\frac{1}{\rho_k} $& inverse of specific density\\\hline
\tt phis(ncl)& $\phi_k$   & sphericity\\\hline
\tt cmus(ncl) & $c_k$ & coefficient for the correlation of the particles turbulent viscosity\\\hline
\tt cps(ncl) & $C_k$ &   thermal capacity\\\hline
\tt phi(ncl) & $\Phi_k$ & volume of particles maximum packing\\\hline
\tt kap(ncl) & $\kappa_k$ &  solid thermal conductivity\\\hline
\tt dkf(ncl,ncl) & $\frac{(d_k+d_{kk})^2}{(\rho_kd_k^3+\rho_{kk}d_{kk}^3)}$ &   Syamlal's particle-particle interaction coefficient \\\hline
\tt philim(ncl,ncl) &$ \frac{\Phi_k}{\Phi_k+(1-\Phi_k)\Phi_j}$& \\\hline
\tt epsl & $\epsilon_{kj}$ & volume of maximum packing for a mixture\\\hline
\tt epsu & $\epsilon_{kj}$ & volume of maximum packing for a mixture\\\hline
\end{tabular}\\
\begin{itemize}
\item{\em particles\_constants\_set} sets the correlation coefficients for particle-particle drag.
\end{itemize}
%
%%%%%%%%%%%%%%%%%%%%%%%%%%%%%%%%%% P R E S S U R E   E P S I L O N %%%%%%%%%%%%%%%%%%%%%%%%%%%%%%%%%%%%
%
{\large{\bf pressure\_epsilon}} (press.f)\\[5mm]
\begin{tabular}{|p{6cm}|c|p{6cm}|}\hline
VARIABLE & REPRESENTATION & MEANING\\\hline
\tt ep(ncdom) & $(\epsilon_g)_{ij}$ &  volumetric fraction of gas\\\hline
\tt pn(ncdom) & $P_{ij}^{n}$ &  gas pressure at time $n$\\\hline
\tt p(ncdom) & $P_{ij}$ &  gas pressure\\\hline
\end{tabular}\\[5mm]
%
%%%%%%%%%%%%%%%%%%%%%%%%%%%%%%%%%% R E A C T I O N S %%%%%%%%%%%%%%%%%%%%%%%%%%%%%%%%%%%%%%%%%%%%%%%%%%
%
{\large{\bf reactions}} (reactions.f)\\[5mm]
\begin{tabular}{|p{6cm}|p{6cm}|}\hline
VARIABLE & MEANING\\\hline
\tt irex & input flag to account for reactions or phase changes\\\hline 
\tt r1, r2, r3, r4, r5 & mass rate of phase change or reaction\\\hline
\tt h1, h2, h3, h4, h5 & enthalpy of phase change or reaction\\\hline
\end{tabular}\\[5mm]
%
%%%%%%%%%%%%%%%%%%%%%%%%%%%%%%%%%% R O U G H N E S S %%%%%%%%%%%%%%%%%%%%%%%%%%%%%%%%%%%%%%%%%%%%%%%%%%
%
{\large{\bf roughness}} (roughness.f)\\[5mm]
\begin{tabular}{|p{6cm}|p{6cm}|}\hline
VARIABLE & MEANING\\\hline
\tt zrough(nroughx) & roughness object\\\hline
\tt roucha & distance of roughness change\\\hline
\end{tabular}\\[5mm]
%
%%%%%%%%%%%%%%%%%%%%%%%%%%%%%%%%%% I N I T I A L   C O N D I T I O N S %%%%%%%%%%%%%%%%%%%%%%%%%%%%%%%%
%
{\large{\bf initial\_conditions}} (setup.f)\\[5mm]
\begin{tabular}{|p{6cm}|p{6cm}|}\hline
VARIABLE & MEANING\\\hline
\tt ugob(nnso) & gas $x(r)$ velocity on specified fluid cells\\\hline 
\tt vgob(nnso) & gas $y$ velocity on specified fluid cells\\\hline
\tt wgob(nnso) & gas $z$ velocity on specified fluid cells\\\hline
\tt epob(nnso) & gas volume fraction at specified fluid cells\\\hline
\tt  tgob(nnso) & gas temperature at specified fluid cells\\\hline
\tt pob(nnso) & gas pressure at specified fluid cells\\\hline
\tt upob(ncl,nnso) & particles $x(r)$ velocity at specified fluid cells\\\hline
\tt upob(ncl,nnso) & particles $y$ velocity at specified fluid cells\\\hline
\tt vpob(ncl,nnso) & particles $z$ velocity at specified fluid cells\\\hline
\tt epsob(ncl,nnso) & particles volume fraction at specified fluid cells\\\hline
\tt tpob(ncl,nnso) & particles temperature at specified fluid cells\\\hline
\tt ygc0(ngas) & gas components mass fraction in atmospheric air\\\hline
\tt ygcob(ngas,nnso) & gas components mass fraction at specified fluid cells\\\hline
\tt lpr & print log file (y/n) flag\\\hline
\tt zzero & height above sea level of the first cell\\\hline
\end{tabular}\\
\begin{itemize}
\item {\em setup} sets initial conditions.
\item {\em setc} sets useful constants
\end{itemize}
%
%%%%%%%%%%%%%%%%%%%%%%%%%%%%%%%%%% S E T   I N D E X E S %%%%%%%%%%%%%%%%%%%%%%%%%%%%%%%%%%%%%%%%%%%%%%
%
{\large{\bf set\_indexes}} (subscr.f)\\[5mm]
\begin{tabular}{|p{6cm}|p{6cm}|}\hline
VARIABLE & MEANING \\ \hline
\tt ipjk, imjk, ippjk, immjk, ijpk, ipjpk, imjpk, ijmk, ipjmk, imjmk, ijppk, ijmmk, ijkp, ipjkp, imjkp, ijpkp, ijmkp, ijkm, ipjkm, imjkm, ijpkm, ijmkm, ijkpp, ijkmm & neighbour cell indexes: {\tt p}(plus) or {\tt m}(minus) following the indexes {\tt i, j, k} indicates the relative position of the neighbour\\\hline
\tt ijke, ijkw, ijkee, ijkww, ijkn, ijken, ijkwn, ijks, ijkes, ijkws, ijknn, ijkss, ijkt, ijket, ijkwt, ijknt, ijkst, ijkb, ijkeb, ijkwb, ijknb, ijksb, ijktt, ijkbb & cell indexes including boundary conditions: standard compass notation {\tt e,n,t,w,s,b,} etc., following the index {\tt ijk} of the cell indicates the relative position of a neighbour \\\hline
\tt stencil & The stencil of a field computed on a cell includes all neighbours values needed for Finite Volumes calculation of convective and diffusive fluxes \\\hline
\end{tabular}\\
\begin{itemize}
\item{\em subscr} assigns indexes to neighbour cells, using arrays {\tt myij} and {\tt myinds} defined in the {\em domain\_decomposition} module\\
\item{\em nb} assemble the computational stencil of a field around a mesh point by considering boundary conditions
\item{\em rnb} assemble the computational stencil of a field around a mesh point
\end{itemize}
%
%%%%%%%%%%%%%%%%%%%%%%%%%%%%%%%%%% G A S   S O L I D   T E M P E R A T U R E %%%%%%%%%%%%%%%%%%%%%%%%%%
%
{\large {\bf gas\_solid\_temperature}} (temp.f)\\[5mm]
\begin{tabular}{|p{6cm}|c|p{6cm}|}\hline
VARIABLE & REPRESENTATION & MEANING\\\hline
\tt sieg(ncdom) & $(h_g)_{ij}$ &  gas specific enthalpy \\\hline 
\tt siegn(ncdom) & $(h_g)^{n}_{ij}$ &  enthalpy of gas at time $n$\\\hline
\tt tg(ncdom) & $(T_g)_{ij}$ &  gas temperature\\\hline
\tt sies(ncl,ncdom) & $(h_s)_{ij}$ &  specific enthalpy of solid phase $s$ \\\hline
\tt siesn(ncl,ncdom) & $(h_s)^{n}_{ij}$ &  enthalpy of solid phase $s$ at time $n$\\\hline
\tt ts(ncl,ncdom) & $(T_s)_{ij}$ &  temperature of solid phase $s$\\\hline
\end{tabular}\\[5mm]
%
%%%%%%%%%%%%%%%%%%%%%%%%%%%%%%%%%% T I L D E   M O M E N T U M %%%%%%%%%%%%%%%%%%%%%%%%%%%%%%%%%%%%%%%%
%
{\large{\bf tilde\_momentum}} (tilde.f)\\[5mm]
\begin{tabular}{|p{4cm}|c|p{8cm}|}\hline
VARIABLE & REPRESENTATION & MEANING\\\hline
\tt rug(ncdom)& $\widetilde{(\rho_g'u_g)_{ij}}$ &explicit (tilde) terms for gas $x(r)$-mo\-men\-tum equation\\\hline
\tt rvg(ncdom)& $\widetilde{(\rho_g'v_g)_{ij}}$ &explicit (tilde) terms for gas $y$-mo\-men\-tum equation\\\hline
\tt rwg(ncdom)& $\widetilde{(\rho_g'w_g)_{ij}}$ &explicit (tilde) terms for gas $z$-mo\-men\-tum equation\\\hline
\tt rus(ncl,ncdom)& $\widetilde{(\epsilon_s\rho_su_s)_{ij}}$ &explicit (tilde) terms for particles $x(r)$-mo\-men\-tum equation\\\hline
\tt rvs(ncl,ncdom)& $\widetilde{(\epsilon_s\rho_sv_s)_{ij}}$ &explicit (tilde) terms for particles $y$-mo\-men\-tum equation\\\hline
\tt rws(ncl,ncdom)& $\widetilde{(\epsilon_s\rho_sw_s)_{ij}}$ &explicit (tilde) terms for particles $z$-mo\-men\-tum equation\\\hline
\tt ugfe(ncdom) & $\left[ \rho_g' u_g u_g x \right]_{E}^n$ & EAST flux of gas $x(r)$-momentum \\\hline 
\tt ugfn(ncdom) & $\left[ \rho_g' u_g v_g \right]_{N}^n$ & NORTH flux of gas $x(r)$-momentum \\\hline
\tt ugft(ncdom) & $\left[ \rho_g' u_g w_g \right]_{T}^n$ & TOP flux of gas $x(r)$-momentum \\\hline
\tt vgfe(ncdom) & $\left[ \rho_g' v_g u_g r \right]_{E}^n$ & EAST flux of gas $y$-momentum \\\hline
\tt vgfn(ncdom) & $\left[ \rho_g' v_g v_g \right]_{N}^n$ & NORTH of gas $y$-momentum \\\hline
\tt vgft(ncdom) & $\left[ \rho_g' v_g w_g \right]_{T}^n$ & TOP flux of gas $y$-momentum \\\hline
\tt wgfe(ncdom) & $\left[ \rho_g' w_g u_g r \right]_{E}^n$ & EAST flux of gas $z$-momentum \\\hline
\tt wgfn(ncdom) & $\left[ \rho_g' w_g v_g \right]_{N}^n$ & NORTH of gas $z$-momentum \\\hline
\tt wgft(ncdom) & $\left[ \rho_g' w_g w_g \right]_{T}^n$ & TOP flux of gas $z$-momentum \\\hline
\tt usfe(ncdom) & $\left[ \epsilon\_s\rho_s u_s u_s x \right]_{E}^n$ & EAST flux of particle $x(r)$-momentum \\\hline 
\tt usfn(ncdom) & $\left[ \epsilon\_s\rho_s u_s v_s \right]_{N}^n$ & NORTH flux of particle $x(r)$-momentum \\\hline
\tt usft(ncdom) & $\left[ \epsilon\_s\rho_s u_s w_s \right]_{T}^n$ & TOP flux of particle $x(r)$-momentum \\\hline
\tt vsfe(ncdom) & $\left[ \epsilon\_s\rho_s v_s u_s x \right]_{E}^n$ & EAST flux of particle $y$-momentum \\\hline
\tt vsfn(ncdom) & $\left[ \epsilon\_s\rho_s v_s v_s \right]_{N}^n$ & NORTH of particle $y$-momentum \\\hline
\tt vsft(ncdom) & $\left[ \epsilon\_s\rho_s v_s w_s \right]_{T}^n$ & TOP flux of particle $y$-momentum \\\hline
\tt wsfe(ncdom) & $\left[ \epsilon\_s\rho_s w_s u_s x \right]_{E}^n$ & EAST flux of particle $z$-momentum \\\hline
\tt wsfn(ncdom) & $\left[ \epsilon\_s\rho_s w_s v_s \right]_{N}^n$ & NORTH of particle $z$-momentum \\\hline
\tt wsft(ncdom) & $\left[ \epsilon\_s\rho_s w_s w_s \right]_{T}^n$ & TOP flux of particle $z$-momentum \\\hline

\tt kpgv(ncl,ncint) & $(D_{g,k})^n_{ij}$ & gas-particles Drag coefficient\\\hline
\tt appu(sup, ncdom) & $\delta t\sum_{l=g,1}^{\tt ncl}(D_{l,k})^n$ & drag terms in the $x(r)$-momentum phase-matrix\\\hline
\tt appv(sup, ncdom) &  $\delta t\sum_{l=g,1}^{\tt ncl}(D_{l,k})^n$ & drag terms in the $y$-momentum phase-matrix\\\hline
\tt appw(sup, ncdom) &  $\delta t\sum_{l=g,1}^{\tt ncl}(D_{l,k})^n$ & drag terms in the $z$-momentum phase-matrix\\\hline
\end{tabular}\\
{\tt sup} = {\tt ((nphase)$^2$+(nphase))/2}
\begin{itemize}
\item{\em fieldn} stores all independent fields at time $n\cdot dt$ for explicit time-integration
\item{\em tilde} computes the explicit (tilde) terms in the momentum transport equation for gas and particles; fluxes are calculated with the donor-cell differencing technique through calls to the routines in the {\em eulerian\_fluxes} module.
\end{itemize}
%
%%%%%%%%%%%%%%%%%%%%%%%%%%%%%%%%%% T I M E   P A R A M E T E R S %%%%%%%%%%%%%%%%%%%%%%%%%%%%%%%%%%%%%%
%
{\large{\bf time\_parameters}} (time.f)\\[5mm]
\begin{tabular}{|p{6cm}|p{6cm}|}\hline
VARIABLE & MEANING\\\hline
\tt itd & input flag for restart \\\hline
\tt rungekut & order of Runge-Kutta explicit time integration (number of RK iterations) \\\hline
\tt time & start time and incremental time\\\hline
\tt tdump & time interval for output dumps \\\hline
\tt tpr & time interval for restart file dumps \\\hline
\tt tstop & maximum time \\\hline 
\tt dt & time step \\\hline
\end{tabular}\\[5mm]
%
%%%%%%%%%%%%%%%%%%%%%%%%%%%%%%%%%% T U R B U L E N C E %%%%%%%%%%%%%%%%%%%%%%%%%%%%%%%%%%%%%%%%%%%%%%%%
%
{\large{\bf turbulence}} (turbo.f)\\[5mm]
\begin{tabular}{|p{6cm}|c|p{6cm}|}\hline
VARIABLE & REPRESENTATION & MEANING\\\hline
\tt mugt(ncdom) & $(\mu_{gt})_{ij}$ &  turbulent viscosity (gas)\\\hline
\tt kapgt(ncdom) & $(\kappa_{gt})_{ij}$ &  turbulent conductivity (gas)\\\hline
\tt must(ncl, ncdom) & $(\mu_s)_{ij}$ &   granular turbulent viscosity (particles)\\\hline
\tt smag(ncint) & $c_S^2(\Delta x\Delta z)^{1/2}$ &  Smagorinsky length scale \\\hline
\tt scoeff(ncint) & $c_S(x,y)$ &  Dynamic Smagorinsky coefficient \\\hline
\tt iss & $ $ & flag for modeled turbulent particle viscosity \\\hline
\tt iturb & $ $ & flag for modeled furbulent gas viscosity \\\hline
\tt modturbo & $ $ & flag for classical or dynamic Smagorinsky model\\\hline
\tt cmut & $ c_S $ &Smagorinsky coefficient \\\hline
\tt pranumt & $ Pr $ & Prandtl number\\\hline
\end{tabular}\\
\begin{itemize}
\item{\em turbulence\_setup} sets values of Smagorinsky parameters in the whole domain.
\item{\em sgsg} computes the gas turbulent viscosity and conductivity by using standard Smagorinsky
or dynamic Smagorinsky models.
\item{\em sgss} computes solid turbulent viscosity.
\item{\em strain2d} computes strain tensor in 2d (cartesian and cylindrical)
\item{\em strain3d} computes strain tensor in 3d
\end{itemize}
%
%%%%%%%%%%%%%%%%%%%%%%%%%%%%%%%%%% G A S   S O L I D   V E L O C I T Y %%%%%%%%%%%%%%%%%%%%%%%%%%%%%%%%
%
{\large {\bf gas\_solid\_velocity}} (velocity.f)\\[5mm]
\begin{tabular}{|p{6cm}|c|p{6cm}|}\hline
VARIABLE & REPRESENTATION & MEANING\\\hline
\tt ug(ncdom) & $(u_g)^n_{ij}$ &   $r$ or $x$ component of gas velocity\\\hline
\tt vg(ncdom) & $(v_g)^n_{ij}$ &  $y$ component of gas velocity\\\hline
\tt wg(ncdom) & $(v_g)^n_{ij}$ &  $z$ component of gas velocity\\\hline
\tt us(ncl,ncdom) & $(u_s)^n_{ij}$ &   $r$ or $x$ component of gas velocity\\\hline
\tt vs(ncl,ncdom) & $(v_s)^n_{ij}$ &  $y$ component of particles velocity\\\hline
\tt ws(ncl,ncdom) & $(v_s)^n_{ij}$ &  $z$ component of particles velocity\\\hline
\end{tabular}\\[5mm]
%
%%%%%%%%%%%%%%%%%%%%%%%%%%%%%%%%%% G A S   S O L I D   V I S C O S I T Y %%%%%%%%%%%%%%%%%%%%%%%%%%%%%%
%
{\large{\bf gas\_solid\_viscosity}} (visc.f)\\[5mm]
\begin{tabular}{|p{6cm}|c|p{6cm}|}\hline
VARIABLE & REPRESENTATION & MEANING\\\hline
\tt mug(ncdom)& $(\nu_g)_{ij}$ &   molecular viscosity (gas)\\\hline
\tt mus(ncdom)& $(\nu_s)_{ij}$ &   particle viscosity coefficient \\\hline
\tt kapg(ncdom)& $(\kappa_g)_{ij}$ &   molecular thermal conductivity (gas)\\\hline
\tt gvisx(ncint)& $(\nabla_x\cdot{\bf T}_g)_{ij}$ & $r$ or $x$ gradient of viscous stress tensor (gas)\\\hline
\tt gvisy(ncint)& $(\nabla_y\cdot{\bf T}_g)_{ij}$&  $y$ gradient of viscous stress tensor (gas)\\\hline
\tt gvisz(ncint)& $(\nabla_z\cdot{\bf T}_g)_{ij}$&  $z$ gradient of viscous stress tensor (gas)\\\hline
\tt pvisx(ncint)& $(\nabla_x\cdot{\bf T}_s)_{ij}$&  $r$ or $x$ gradient of viscous stress tensor (particles)\\\hline
\tt pvisy(ncint)& $(\nabla_y\cdot{\bf T}_s)_{ij}$&  $y$ gradient of viscous stress tensor (particles)\\\hline
\tt pvisz(ncint)& $(\nabla_z\cdot{\bf T}_s)_{ij}$&  $z$ gradient of viscous stress tensor (particles)\\\hline
\end{tabular}\\
\begin{itemize}
\item{\em viscon}  computes temperature-dependent conductivity and viscosity
\item{\em viscg} computes the viscous stress tensor of gas
\item{\em viscs} computes the viscous stress tensor of particles
\end{itemize}
%
%%%%%%%%%%%%%%%%%%%%%%%%%%%%%%%%%% G A S   C O M P O N E N T S %%%%%%%%%%%%%%%%%%%%%%%%%%%%%%%%%%%%%%%%
%
{\large{\bf gas\_components}} (ygas.f)\\[5mm]
\begin{tabular}{|p{6cm}|c|p{6cm}|}\hline
VARIABLE & REPRESENTATION & MEANING\\\hline
\tt yfe(ncdom,ngas) & $\left[ (\rho_g')y_{ig}u_gx \right]_{E} $ & convective gas component mass flux in $x(r)$ direction\\\hline
\tt yfn(ncdom,ngas) & $\left[ (\rho_g')y_{ig}v_g \right]_{N} $ & convective gas component mass flux in $y$ direction\\\hline
\tt yft(ncdom,ngas) & $\left[ (\rho_g')y_{ig}w_g \right]_{T} $ & convective gas component mass flux in $z$ direction\\\hline
\end{tabular}\\
\begin{itemize}
\item{\em ygas} solves the transport equations of gas components.
\end{itemize}
\clearpage
