\section{\label{sect:solproc} Solution Procedure}
%
The solution of the model equations at a given time is obtained by time-marching
with a discrete time-step from the initial conditions. At each time-step the
coupled transport equations are solved by adopting an iterative solution 
procedure on the mesh.

\subsection{Main flow chart}

\begin{enumerate}
\item Startup: the parallel environment is initialized
\item Input: root processor reads input files and open I/O units
\item Broadcast: input data are broadcasted among parallel processes
\item Domain Decomposition: set the processors sub-domains, ghost cells, and communication routines
(after DD, all computation is carried out in parallel)
\item Setc: set constants and evaluate physical parameters
\item Setup: set initial conditions (standard atmosphere, topography, vent conditions)
\item Time Marching: solve the model equation to advance the solution from the prescribed initial conditions,
until the final time is reached
\item Hangup: finalize the parallel environment
\end{enumerate}

\subsection{Time Marching}
\begin{enumerate}
\item Compute boundary conditions
\item Compute gas density from the thermal equation of state
\item Store all fields at time $n\delta t$
\item Compute temperature dependent viscosity and thermal conductivity of gas
\item Start Runge-Kutta cycle
\item Compute Tilde terms and interphase coefficients.
\item Solve iteratively the coupled system equations to update velocity fields,
pressure, and volumetric fractions.
\item Solve the mass balance equation for gas species
\item Solve explicitly the enthalpy transport equation of all phases
\item Loop over the Runge-Kutta cycle or increase the time step and restart the time-marching procedure
\item Write output file and restart file when prescribed
\item EXIT if maximum time is reached
\end{enumerate}
\clearpage

%**************************************************************************
% ... user-defined objects ...
\setlength{\unitlength}{12pt}
\scriptsize
\newsavebox{\IO}
\savebox{\IO}(0,0)
{
\put(1,2){\line(1,0){5}}
\put(0,0){\line(1,0){5}}
\put(0,0){\line(1,2){1}}
\put(5,0){\line(1,2){1}}
}
\newsavebox{\IF}
\savebox{\IF}(0,0)
{
\put(0,-1){\line(2,1){2}}
\put(0,-1){\line(-2,1){2}}
\put(0,1){\line(2,-1){2}}
\put(0,1){\line(-2,-1){2}}
}
%..................................MAIN...................................
\begin{figure}
\begin{minipage}{7.5cm}
\begin{center}
{\normalsize \bf MAIN}\\[1cm]
\begin{picture}(6,25)
\multiput(3,23)(0,-3){6}{\vector(0,-1){1}}
\put(0,23){\makebox(6,2){\bf STARTUP}}
\put(3,24){\oval(6,2)}
\put(0,20){\makebox(6,2){\bf INPUT}}
\put(0,21){\usebox{\IO}}
\put(0,17){\makebox(6,2){\bf broadcast}}
\put(3,18){\oval(6,2)}
\put(0,14){\framebox(6,2){\parbox{30mm}{\begin{center} 
\bf domain \\ \bf decomposition \end{center}}}}
\put(0,11){\framebox(6,2){\bf SETC}}
\put(0,8){\framebox(6,2){\bf SETUP}}
\put(0,3){\framebox(6,4){\parbox{30mm}{\begin{center} 
\bf TIME \\ \bf MARCHING \end{center} }}}
\put(6,4.5){\vector(1,0){4.5}}
\put(3,3){\vector(0,-1){1}}
\put(0,0){\makebox(6,2){\bf HANGUP}}
\put(3,1){\oval(6,2)}
\end{picture}
\end{center}
\end{minipage}
%..................................TIMESWEEP..............................
\begin{minipage}{7.5cm}
\begin{center}
{\normalsize \bf TIME MARCHING}\\[1cm]
\begin{picture}(17,33)
\put(-2,-6){\framebox(21,40)}
\put(3,32.5){\vector(0,-1){0.5}}
\put(3,32.5){\line(1,0){11}}
\put(0,30){\framebox(6,2){\parbox{30mm}{\begin{center} 
\bf Boundary \\ \bf Conditions \end{center}}}}
\put(3,30){\vector(0,-1){1}}
\put(0,27){\framebox(6,2){\bf EOS}}
\put(3,27){\vector(0,-1){1}}
\put(0,24){\makebox(6,2){\bf OUTPUT}}
\put(0,25){\usebox{\IO}}
\put(3,24){\vector(0,-1){1}}
\put(3,22){\usebox{\IF}}
\put(1,21){\makebox(4,2){$t<t_{stop}$}}
\put(5,22){\vector(1,0){1}}
\put(5,22.5){no}
\put(8,22){\oval(4,2)}
\put(6,21){\makebox(4,2){\bf EXIT}}
\put(3,21){\vector(0,-1){1}}
\put(3.5,20.5){yes}
\put(0,15){\framebox(6,5)
{\parbox{25mm}{\begin{center} 
\bf EXPLICIT\\
\bf Viscosity\\
\bf Conductivity\\
\bf Fields at $\bf n\delta t$ \end{center} }}}
\put(6.5,17.5){\bf $dt_0=dt$}
\put(3,15){\vector(0,-1){2}}
\put(0,11){\framebox(6,2){\parbox{30mm}{\begin{center} 
\bf Momentum \\ \bf Fluxes \end{center}}}}
\put(0,10){\framebox(6,1){\bf Drag}}
\put(0,9){\framebox(6,1){\bf Gravity}}
\put(3,9){\vector(0,-1){1}}
\put(0,3){\framebox(6,5)
{\parbox{25mm}{\begin{center} 
\bf ITER \\
\bf Solver for \\
\bf Implicit\\
\bf Coupling
\end{center}}}}
\put(9,4){\makebox(4,2){\parbox{20mm}{\begin{center} 
\bf Runge-\\
\bf Kutta \\
\bf Cycle
\end{center}}}}
\put(9.2,3){\tiny {\bf $dt=\frac{dt}{N+1-k}$}}
\put(3,3){\vector(0,-1){1}}
\put(0,0){\framebox(6,2){\parbox{25mm}{\begin{center} 
\bf GAS \\
\bf Components
\end{center}}}}
\put(3,0){\vector(0,-1){1}}
\put(0,-2){\framebox(6,1){\bf Enthalpy}}
\put(3,-2){\line(0,-1){2.5}}
\put(3,-4.5){\vector(1,0){11}}
\put(5,-4.3){\bf $dt=dt_0; \quad t=t+dt$}
\put(3,-3){\vector(1,0){6}}
\put(9,-3){\vector(0,1){17}}
\put(9,14){\vector(-1,0){6}}
\put(14,-4.5){\vector(0,1){20.5}}
\put(14,16){\vector(0,1){16.5}}
\end{picture}
\begin{picture}(16,4.5)
\end{picture}
\end{center}
\end{minipage}
\vspace{1cm}
\caption{\label{fig:prog_chart} Flow chart of the PDAC2D {\em main} program flow and 
of the time-advancement procedure.  The whole time-advancement block is repeated at each 
time-step, from initial condition to the prescribed final time.} 
\end{figure}
\clearpage
%...................................ITER..............................
\normalsize
\subsubsection{Iterative solver}
\begin{enumerate}
\item Compute the sound speed of gas and the derivative of the mass residual with respect to pressure
\item Compute the coefficients of the interphase matrix derived from the implicit interphase coupling of the 
momentum trasport equations by using the pressure field at time $n\delta t$, and the explicit (Tilde) terms 
previously calculated. Invert the matrix, obtaining gas and particles velocity fields biassed by the wrong
pressure field.
\item Compute gas mass fluxes by using the guessed velocities
\item Start the mesh sweep
\item Compute the mass residual $D_g$ in a cell
\item If $D_g$ is less than  prescribed value, update the volumetric fraction of solids by using the mass balance
equation of particles, the void fraction, and proceed to next cell
\item Otherwise, correct the value of the pressure, according to one of the techniques described above
\item Correct the gas density with the new pressure
\item Use the new pressure to compute a new guessed velocity field (by inverting the interphase matrix)
\item Compute solid mass fluxes with the new velocities
\item Use the continuity equation of solids to compute the new particle volumetric fractions (void fraction 
is obtained by the closure equation)
\item Compute new gas mass fluxes by using the new velocities and the new void fraction
\item Compute the mass residual $D_g$
\item If $D_g$ is less than  prescribed value, proceed to next cell
\item Otherwise, restart the loop in the same cell from a new pressure correction, until the convergence criterion is
satisfied or a maximum number of inner iteration is reached
\item Proceed to the next cell in the same way
\item At the end of the mesh sweep, if convergence is not reached in every cell, restart the whole mesh iteration.
\end{enumerate}
\begin{figure}
\scriptsize
\setlength{\unitlength}{10pt}
\begin{center}
{\normalsize \bf ITER: Iterative Solver for Pressure Coupling}\\[1cm]
\begin{picture}(19,53)
\put(-2,-2){\framebox(24,57)}
\put(6,52){\framebox(6,2){\bf $C$, $\frac{dD_g}{dP}$}}
\put(9,52){\vector(0,-1){1}}
\put(6,49){\framebox(6,2){\parbox{30mm}
{\begin{center} 
\bf GUESS \\
\bf velocities\\
\end{center}}}}
\put(9,49){\vector(0,-1){1}}
\put(6,46){\framebox(6,2){\parbox{30mm}
{\begin{center} 
\bf Gas \\
\bf Mass Fluxes\\
\end{center}}}}
\put(9,46){\vector(0,-1){1}}
\put(9,44){\oval(6,2)}
\put(6,43){\makebox(6,2){\parbox{30mm}
{\begin{center} 
\bf START\\
\bf Mesh Sweep\\
\end{center}}}}
\put(9,43){\vector(0,-1){2}}
\put(6,39){\framebox(6,2){\parbox{30mm}
{\begin{center} 
\bf Gas Mass\\
\bf Residual $D_g$\\
\end{center}}}}
\put(9,39){\vector(0,-1){1}}
\put(9,37){\usebox{\IF}}
\put(7,35.9){\makebox(4,2){\bf $D_g< C$}}
\put(11,37){\line(1,0){7}}
\put(18,37){\vector(0,-1){2}}
\put(15,34){\framebox(6,1){$ \bf \epsilon_s$}}
\put(15,33){\framebox(6,1){$ \bf \epsilon_g = 1-\sum\epsilon_s$}}
\put(15,32){\framebox(6,1){$ \bf \rho_g'=\epsilon_g\rho_g$}}
\put(18,32){\line(0,-1){22}}
\put(18,24){\vector(0,-1){1}}
\put(18,10){\vector(-1,0){9}}
\put(9,36){\vector(0,-1){2}}
\put(6,32){\framebox(6,2){\bf Correct $P_{ij}$}}
\put(6,30){\framebox(6,2){\bf EOS $\rho_g(P,T)$}}
\put(9,30){\vector(0,-1){1}}
\put(6,27){\framebox(6,2){\parbox{30mm}
{\begin{center} 
\bf GUESS \\
\bf velocities\\
\end{center}}}}
\put(9,27){\vector(0,-1){1}}
\put(6,24){\framebox(6,2){\parbox{30mm}
{\begin{center} 
\bf Particles \\
\bf Mass Fluxes\\
\end{center}}}}
\put(9,24){\vector(0,-1){1}}
\put(6,22){\framebox(6,1){$ \bf \epsilon_s$}}
\put(6,21){\framebox(6,1){$ \bf \epsilon_g = 1-\sum\epsilon_s$}}
\put(6,20){\framebox(6,1){$ \bf \rho_g'=\epsilon_g\rho_g$}}
\put(9,20){\vector(0,-1){1}}
\put(6,17){\framebox(6,2){\parbox{30mm}
{\begin{center} 
\bf Gas \\
\bf Mass Fluxes\\
\end{center}}}}
\put(9,17){\vector(0,-1){1}}
\put(6,14){\framebox(6,2){\parbox{30mm}
{\begin{center} 
\bf Gas Mass\\
\bf Residual $D_g$\\
\end{center}}}}
\put(9,14){\vector(0,-1){1}}
\put(9,12){\usebox{\IF}}
\put(7,10.9){\makebox(4,2){\bf $D_g< C$}}
\put(9,11){\line(0,-1){2}}
\put(9,8){\oval(6,2)}
\put(6,7){\makebox(6,2){\bf NEXT cell}}
\put(9,7){\line(0,-1){1}}
\put(9,6){\vector(-1,0){7}}
\put(2,5){\bf Mesh LOOP}
\put(7,12){\vector(-1,0){4}}
\put(3,11){\bf Cell LOOP}
\put(2,6){\line(0,1){36}}
\put(2,23){\vector(0,1){1}}
\put(3,12){\line(0,1){23}}
\put(3,23){\vector(0,1){1}}
\put(2,42){\vector(1,0){7}}
\put(3,35){\vector(1,0){6}}
\put(9,6){\vector(0,-1){1}}
\put(9,4){\usebox{\IF}}
\put(7,2.9){\makebox(4,2){\bf conv?}}
\put(7,4){\vector(-1,0){7}}
\put(1,3){\small \bf SOR}
\put(0,4){\line(0,1){41.5}}
\put(0,23){\vector(0,1){1}}
\put(0,45.5){\vector(1,0){9}}
\put(9,3){\vector(0,-1){2}}
\put(9,0){\oval(6,2)}
\put(6,-1){\makebox(6,2){\bf RETURN}}
\end{picture}
\end{center}
\caption{\label{fig:iter_chart} Flow chart of the algorithm flow in the iterative solver:
at each time-step, converegence is reached by iteratively correct the pressure field in each
cell and re-calculating the velocity fields. Corrections are propagated to neighbour cells
by using Successive Over-Relaxation (SOR) technique until the gas mass residual is less than a prescribed value in the whole
computational domain.}
\end{figure}
\normalsize
%**************************************************************************
\clearpage

\subsection{Iterative solver}

The procedure starts with a guessed pressure field that may be represented
by either the initial pressure distribution of standard atmosphere (initial
conditions) or the pressure distribution of the previous time-step.
By using such a pressure field, momentum equations are solved explicitly to get the 
velocity of each phase.
The residual of the gas mass balance equation is then evaluated by using 
the old volumetric fraction and the velocity field just computed: 

\begin{eqnarray}
D_{i,j} =  (\epsilon_g \rho_g)_{i,j}^{*} - (\epsilon_g \rho_g)_{i,j}^{n}
-\frac {\delta t} {R_i \delta r_i} \langle R (\epsilon_g \rho_g) u_g \rangle_{i,j}^{*}
- \frac {\delta t} {\delta z_j } \langle (\epsilon_g \rho_g) v_g \rangle_{i,j}^{*}
\nonumber
\end{eqnarray}
%
and compared with the convergence criterion. Starred quantities are evaluated at current
iteration step. At the first iteration $*=n$, whereas $*=n+1$ when convergence is
reached in the whole computational domain.
In the code the convergence is reached when $D_{i,j}$ is $\le 10^{-8} \times 
(\epsilon_g \rho_g)_{i,j}^{n}$. 

If the convergence criterion is satisfied, the continuity equations of the solid phases are 
solved to determine the updated particulate volumetric fraction of each phase.
The new gas volumetric fraction is then computed from:
%
\begin{eqnarray}
\epsilon_g  = 1 - \sum_{k=1}^{n} \epsilon_{k}
\nonumber
\end{eqnarray}
%
and the calculation proceeds to the next cell.
If the residual is greater than the prescribed limit the {\em in-cell} correction procedure
is started, proceedings throughout the following steps:
\begin{itemize}
\item Pressure is first corrected by using one of the iterative techniques illustrated in section 
\ref{sect:pressure}. 
\item The new thermodynamic gas density is computed according to the prescribed
equation of state (EOS in Fig.~\ref{fig:iter_chart}). 
\item New velocities are computed by solving the system of coupled momentum transport 
equations with the new pressure and density, and the old volumetric fractions. 
\item Particle volumetric fractions can be now computed from solid mass transport 
equations with the updated velocity fields. The gas volumetric fraction is also computed 
from the closure equation.
\item The new gas volumetric fraction and the new velocities are used in the gas 
continuity  equation to compute the new mass residual.
\item If the gas mass residual is too large the in-cell correction is repeated, otherwise
the entire procedure is repeated in the next cell.
The iteration begins at the left-bottom corner fluid cell. The computation proceeds 
from left to right and from bottom to top until the entire computational domain is covered.
\item When all cells have been updated the procedure is repeated (by applying over-relaxation)
to propagate the corrections to the neighbour cells, until convergence is simultaneously reached
in all the cells. 
\end{itemize}
To illustrate the need for this external iteration (SOR iteration in Fig.~\ref{fig:iter_chart})
it is sufficient to examinate the pressure term in the momentum equations. 
When the pressure field in a cell is corrected, the pressure gradient in all
(first) neighbours needs to be modified as we use a (first order) finite difference representation
of gradients. In the mesh loop, the previously updated values are used when evaluating
the new fields in one cell, but the information coming from the following cells cannot be used 
in the same iteration. 
As a few tens of iterations are needed to reach simultaneous convergence
in all cells, in-cell pressure corrections are limited to a max number of inner iteration,
even if correction is not usually completed. In fact, a further correction could be needed in the next 
mesh loop, and therefore a good trade-off must be selected between the number of pressure corrections
({\tt inmax}) and the number of mesh iterations ({\tt maxout}).\\
%
\subsection{\label{sect:pressure} Pressure correction algorithm}
%
The above reported convergence criterion is reached by adjusting 
pressure through a combination of different numerical methods.
\begin{enumerate}
\item
The initial adjustment is done by using Newton's method (Fig.~\ref{fig:newtonm}). 
If $P_g$ is the gas pressure and the superscripts $s$ indicates the $s$-th iteration:
%
\begin{eqnarray}
(P_g)_{i,j}^{s+1} =  (P_g)_{i,j}^{s} - \omega dP_g  \label{eq:correction}\\
dP_g = \frac {D_{i,j}^s} {\left(\partial D_{i,j} / \partial (P_g)_{i,j}\right)^s}\label{eq:dDdP}
\end{eqnarray}
%
where the index $s$ indicates the iteration level and $\omega$ is a relaxation parameter.
The derivative of the residue with respect to pressure has been computed by (see below for
a derivation of this expression):
%
\begin{eqnarray}
\frac{\partial D_{i,j}} {\partial (P_g)_{i,j}} \approx 
\frac{\epsilon_g}{C_{i,j}^{2}} +
\frac{1}{R_i}\Big(\frac {\delta t} {\delta r_i}\Big)^2 \big( 
R_{i+\frac{1}{2}} (\epsilon_g)_{i+\frac{1}{2},j} +
R_{i-\frac{1}{2}} (\epsilon_g)_{i-\frac{1}{2},j} \big)+
\Big(\frac {\delta t} {\delta z_j}\Big)^2 \big( 
(\epsilon_g)_{i,j+\frac{1}{2}} + (\epsilon_g)_{i,j-\frac{1}{2}} \big)
\nonumber
\end{eqnarray}
%
where the sound speed of the gas phase $C_{i,j}$ is determined from the equation of state. This expression
is evaluated once at the beginning ($\frac{dD_g^o}{dP_g}$ in Fig.~\ref{fig:newtonm}) 
and it is not recalculated during the iteration, since its variations do not affect strongly 
the convergence of the method.\\
Newton's method is used until the residual changes sign.
\begin{figure}[h]
\centerline{\psfig{figure=./newtonm.eps,height=8cm}}
\caption{\label{fig:newtonm} Sketch of the Newton's method used to find the zero of the function
$D_g(p_g)$: the slope of the function is computed once and used to iteratively approach the zero.
If the value of the function $D_g$ changes sign (point 2) the iterative Newton's method is
stopped and the secant method of Fig.~\ref{fig:secantm} is used.}
\end{figure}
%
\item A new approximation of the solution is reached in a further stage by using the 
secant method (Fig.~\ref{fig:secantm}), i.e. by evaluating Eq.~\ref{eq:dDdP} with:
%
\begin{eqnarray}
\partial (p_g)_{i,j}^{s} / \partial D_{i,j} = \bigg( \frac {(p_g)_{i,j}^{s-1} - (p_g)_{i,j}^{s}}
{D_{i,j}^{s-1} - D_{i,j}^{s}} \bigg)
\nonumber
\end{eqnarray}
%
$(p_g)_{i,j}^{s-1}$ and $(p_g)_{i,j}^{s}$ are the two last values obtained by the Newton's 
method, and their corresponding residuals have opposite sign. The secant method is used once 
to obtain a third pressure between these two values. 
%
\begin{figure}[h]
\centerline{\psfig{figure=./secantm.eps,height=8cm}}
\caption{\label{fig:secantm} Sketch of the secant method used when the iterative Newton's method
changes sign: a third point between points P1 and P2 is found by tracing the secant between 1 and 2.}
\end{figure}
%
\item With three points surrounding the solution, the bi-secant method (Fig.~\ref{fig:bisecantm}) 
can be applied iteratively to fastly converge to the zero-residual pressure.
%
\begin{figure}[h]
\centerline{\psfig{figure=./bisecantm.eps,height=8cm}}
\caption{\label{fig:bisecantm} Sketch of the iterative bisecant method used with three points surrounding
the zero of the function $D_g(P_g)$. The two secants between 1-3 and 2-3 are traced to find points
A and B on the $D_g=0$ axis. Point 4 lay mid-way between A and B. In the case shown (where P4 is 
between P2 and P3) point 1 is discarded and the procedure is repeated with points 2,3,4.}
\end{figure}
%
\end{enumerate}
%

The total number of iterations is usually of the order of 3-4 for the pressure adjustment
in each cell, and of few tens for the outer iterations over the entire domain.\\[5mm]
%
\clearpage
\subsubsection{Derivative of gas mass residual}
%
A few manipulation and approximation of the momentum transport equations are necessary
to obtain the derivative of the gas mass-residual with respect to pressure, used in 
Eq.~\ref{eq:dDdP}.
The procedure is presented in 1D and cartesian coordinates although its extension to 2D 
is straighforward.

The residual is obtained from the mass transport equation as:
\begin{eqnarray}
\label{eq:resid}
(D_g)_i & = &  (\epsilon_g \rho_g)_i^{*} - (\epsilon_g \rho_g)_i^{n}
+ \frac {\delta t} {\delta x_i } \langle (\epsilon_g \rho_g) u_g \rangle_i^{*} \\ \nonumber
& = & (\epsilon_g \rho_g)_i^{*} - (\epsilon_g \rho_g)_i^{n}
+ \frac {\delta t} {\delta x_i } 
\left[ 
(\epsilon_g \rho_g)^n_{i+\frac{1}{2}}(u_g)^*_{i+\frac{1}{2}} -
(\epsilon_g \rho_g)^n_{i-\frac{1}{2}}(u_g)^*_{i-\frac{1}{2}}
\right]
\end{eqnarray}
In this equation the term $(\epsilon_g \rho_g)_i^{n}$ is explicitly given,
the terms $(\epsilon_g \rho_g)^n_{i+\frac{1}{2}}$ and $(\epsilon_g \rho_g)^n_{i-\frac{1}{2}}$ are
modeled by using some upwind scheme (such as the First Order Upwind), whereas starred terms
are guessed from some approximated equation. The first term $(\epsilon_g \rho_g)_i^{*}$ is
obtained from the previous iteration, or is equal to $(\epsilon_g \rho_g)_i^{n}$ in the
first iteration. The starred terms in the square bracket are evaluated by using the momentum
equation. As we cannot solve the momentum equation explicitly, we retain only the terms
that are relied to pressure variation (as we want to calculate the derivative with respect
to pressure variations!) so that the momentum equation can be written:
\begin{eqnarray}
(\epsilon_g \rho_g)^n_{i+\frac{1}{2}}(u_g)^*_{i+\frac{1}{2}} \approx
(\epsilon_g \rho_g)^n_{i+\frac{1}{2}}(u_g)^n_{i+\frac{1}{2}} -
\frac {\delta t} {\delta x_{i+\frac{1}{2}}}
(\epsilon_g)^n_{i+\frac{1}{2}}(p_{i+1}-p_i)^*
\nonumber
\end{eqnarray}
Now, if this equation is substituted into \ref{eq:resid}, we obtain:
\begin{eqnarray}
(D_g)_i & \approx &  (\epsilon_g \rho_g)_i^{*} - (\epsilon_g \rho_g)_i^{n} \\
& + & \frac{\delta t}{\delta x_i } 
\left[ 
(\epsilon_g \rho_g u_g)^n_{i+\frac{1}{2}} - (\epsilon_g \rho_g u_g)^n_{i-\frac{1}{2}} 
- \frac{\delta t}{\delta x_{i+\frac{1}{2}}} (\epsilon_g)^n_{i+\frac{1}{2}}(P_{i+1}-P_i)^*
+ \frac{\delta t}{\delta x_{i-\frac{1}{2}}} (\epsilon_g)^n_{i-\frac{1}{2}}(P_i-P_{i-1})^*
\right]
\nonumber
\end{eqnarray}
Deriving this expression with respect to $P_i$ we obtain the final expression, where
the first term was approximated as:
\begin{displaymath}
\frac{\partial (\epsilon_g \rho_g)_i^{*} } {\partial P_i } \approx 
(\epsilon_g)_i^n \frac{\partial (\rho_g)_i^{*} } {\partial P_i } = 
\frac{(\epsilon_g)_i^n}{C^2}
\end{displaymath}
where $C$ is the adiabatic speed of sound of the gas.\\
\begin{eqnarray}
\frac{\partial D_{i}} {\partial (p_g)_{i}} \approx
\frac{\epsilon_g}{C_{i}^{2}} +
\frac {\delta t} {\delta x_i} \left(
\frac{\delta t}{\delta x_{i+\frac{1}{2}}}(\epsilon_g)^n_{i+\frac{1}{2}} + 
\frac{\delta t}{\delta x_{i-\frac{1}{2}}}(\epsilon_g)^n_{i-\frac{1}{2}} \right)
\nonumber
\end{eqnarray}
%
