\section{Simulation Parameters}
\label{section:input_par}

\subsection{\PDAC\ input namelists}
\label{section:namelists}

\subsubsection{\OBJ{control} namelist}
The \OBJ{control} namelist contains the parameters that control the program
flow, such as the start mode, the initial and final time, the output printing.

\begin{itemize}
\item
\PDACCONFWDEF{run\_name}{run identificative}{any string}{{\tt pdac\_run\_2d}}
{This string identifies the simulation. Its length must not exceed 80 characters. It must be enclosed by quotes.}

\item
\PDACCONFWDEF{job\_type}{2d or 3d}{{\tt '2d','2D','3d','3D'}}{{\tt '2D'}}
{When 2D is selected, the model equations are simplified by assuming
some simmetry in the physical domain (translation in one cartesian 
direction or cylindrical symmetry)}

\item
\PDACCONFWDEF{restart\_mode}{flag for restart}{{\tt 'from\_scratch'}, {\tt 'restart'}, {\tt 'check\_geom'}, {\tt 'check\_init'}, {\tt 'outp\_recover'}}
{{\tt 'from\_scratch'}}
{If {\tt restart} is selected, initial conditions are set up by using the fields
dumped (in double precision) into the restart file {\tt pdac.res} (see above). 
Otherwise, initial conditions are set up from input data in {\tt pdac.dat}.
If {\tt 'check\_geom'}, or {\tt 'check\_init'} are selected, the program stops
after the mesh and topography or the initial conditions are set, respectively.
{\tt 'outp\_recover'} can be set to restart from the output file specified
by the \OBJ{nfil} parameter.} 

\item
\PDACCONFWDEF{time}{initial time}{double real}{0.0}
{Inital simulation time (in seconds)}

\item
\PDACCONFWDEF{tstop}{end time}{double real}{100.0}
{Time (in seconds) at which the simulation stops}

\item
\PDACCONFWDEF{dt}{time step}{double real}{0.01}
{Time advancement step (in seconds). 
{\tt dt} is constrained by the CFL condition
$dt < C_{max}\frac{\Delta x}{V_{max}}$, where $\Delta x$ is
the size of the computational grid and $V_{max}$ is the maximum
velocity. The maximum CFL number $C_{max}\approx 0.2$ has been 
found empirically.}

\item
\PDACCONFWDEF{lpr}{level of verbosity}{1,2,3}{2}
{increases the level of verbosity in warning and error messages 
in {\tt pdac.log}, {\tt pdac.err}, {\tt pdac.tst} files}

\item
\PDACCONFWDEF{imr}{flag for mass residuals}{0/1}{0}
{ When 1 selected, writes out the total gas and particle mass in the domain
and the mass flow rate, in a specific file \FIL{pdac.chm}}

\item
\PDACCONFWDEF{isrt}{flag for runtine sampling of the pressure}{0/1}{0}
{ When 1 selected, writes out in the file \FIL{pwav.dat} the pressure field at every time-step
in the locations specified by a file named \FIL{probes.dat}}

\item
\PDACCONFWDEF{tpr}{time interval for OUTPUT file}{$>$ dt}{1.0}
{OUTPUT files are printed every {\tt tpr} seconds of simulated time}

\item
\PDACCONFWDEF{tdump}{time interval for restart file}{$>$ dt}{20.0}
{restart file is overwritten every {\tt tdump} seconds of simulated time}

\item
\PDACCONFWDEF{nfil}{number of first OUTPUT file}{any positive integer up to 9999}{0}
{Initial value for numbering the output files or for recovering.
Output files are written with 4 digits extension ({\tt OUTPUT.XXXX}). 
The output time can be recovered as $({\tt XXXX - nfil})* {\tt tpr}$ }

\item
\PDACCONFWDEF{tau}{transient vent conditions}{double real}{0.0}
{Time needed to esablish the flow conditions at the vent (in seconds)}

\item
\PDACCONFWDEF{tau1}{transient vent conditions}{double real}{0.0}
{Time needed to esablish the flow conditions at the vent (in seconds)}

\item
\PDACCONFWDEF{tau2}{transient vent conditions}{double real}{0.0}
{Time needed to esablish the flow conditions at the vent (in seconds)}

\item
\PDACCONFWDEF{formatted\_input}{flag for INPUT format}{T/F}{T}
{Determine the format of input files for restart (valid for both options
{\em outp\_recover} and {\em restart}): T - formatted ascii, 
F - binary 4-bytes }

\item
\PDACCONFWDEF{formatted\_output}{flag for OUTPUT format}{T/F}{T}
{Determine the format of output files: T - formatted ascii, 
F - binary 4-bytes }

\item
\PDACCONFWDEF{max\_seconds}{maximum CPU time}{any real value}{20000.0}
{Maximum duration of a simulation (in seconds). If time exceeds this value
a restart file is written before the simulation is stopped (useful for
scheduling). Default time is set to 6 hours}
\end{itemize}

\subsubsection{\OBJ{model} namelist}
The \OBJ{model} namelist is intended for all switches that can be selected
by the user to modify the model equations, either by neglecting some terms
in the transport equations or by using different constitutive equations
and submodels.

\begin{itemize}

\item
\PDACCONFWDEF{icpc}{gas specific heat}{0,1}{1}
{1: specific heat depends on temperature}

\item
\PDACCONFWDEF{irex}{chemical reactions}{0,1}{0}
{The chemical reaction module has not yet been implemented}

\item
\PDACCONFWDEF{gas\_viscosity}{gas diffusive transport}{{\tt T/F}}{{\tt T}}
{T to solve the full set of model equations. F switches the diffusive
transport terms off for the gas phase (viscous and turbulent term in momentum 
equation, thermal diffusivity in enthalpy equation). 
Gas viscosity is computed anyway to include the gas-particle drag and
energy exchange terms.}

\item
\PDACCONFWDEF{part\_viscosity}{particle diffusive transport}{{\tt T/F}}{{\tt T}}
{Switches on/off the diffusive terms in the particle transport equations 
(viscous terms in the momentum equation and thermal conductivity).}

\item
\PDACCONFWDEF{iss}{turbulence model for particles}{0,1}{0}
{1: computes a gas-analogous sub-grid stress (turbulent viscosity)
for particles.}

\item
\PDACCONFWDEF{repulsive\_model}{flag for Coulombic repulsive model}{0,1}{1}
{The Coulombic repulsive model add a contribution to the diagonal part of
the solid viscous stress due to the repulsive interaction of particles 
at high concentrations. This term appears to be important for the solid
equation to be well-posed.}

\item
\PDACCONFWDEF{iturb}{turbulence model for gas}{0,1,2}{1}
{0 - no gas turbulence model; 1 - Smagorinsky sub-grid stress model;
2 - Smagorinsky sgs model with roughness closure at the walls.}

\item
\PDACCONFWDEF{modturbo}{Subgrid-scale model for gas turbulence}{1,2}{1}
{1 - Classical Smagrinsky model; 2 - Dynamic Smagorinsky model}

\item
\PDACCONFWDEF{cmut}{Smagorinsky constant}{usually between 0.1 and 0.4}{0.1}
{The exact value cannot be predicted {\it a priori}. The use of the dynamic
Smagorinsky model (see above) makes the assignment of this constant
non-necessary}

\item
\PDACCONFWDEF{gravx}{acceleration along x}{any real}{0.0}
{Body-force in x(r)-direction}

\item
\PDACCONFWDEF{gravy}{acceleration along y}{any real}{0.0}
{Body-force in y-direction}

\item
\PDACCONFWDEF{gravz}{acceleration along z}{any real}{-9.81}
{Body-force in z-direction (usually the value of the gravitational acceleration).
Values different from the default value could not be consistent with the 
atmospheric stratification, leading to instabilities. A value of 0.0 
suppresses atmospheric stratification}.

\item
\PDACCONFWDEF{ngas}{number of gas components}{1 to 7}{2}
{Seven gas species are defined, specifically: \\
$ 1) O_2, 2) N_2, 3), CO_2, 4) H_2, 5) H_2O, 6) Air, 7) SO_2$. 
Only gas species that are specified at the inlet or in the
atmosphere are considered by the model. The total number of gas 
species specified by this flag must be therefore consistent with 
input conditions, otherwise the program stops.}

\item
\PDACCONFWDEF{density\_specified}{flag for i.c.}{{\tt T/F}}{{\tt F}}
{If .TRUE., specify gas density instead of temperature for initial conditions.
Particle temperature is set equal to gas temperature.}

\end{itemize}

\subsubsection{\OBJ{mesh} namelist}
Here all parameters concerning the spatial discretization of the computational
domain must be selected. In \PDAC\ you are constrained to discretization 
on a rectilinear (non-)uniform mesh. Cylindrical coordinates can be
selected only in 2D.

\begin{itemize}
\item
\PDACCONFWDEF{nx}{Number of cells in x(r)-direction}{up to 512}{100}
{When 2D cylindrical coordinates are selected, ``x'' is used instead 
 of ``r''. This number includes the boundary ghost cells. The maximum number can be 
 increased by modifying the max\_size parameter in the ``dimensions'' module}

\item
\PDACCONFWDEF{ny}{Number of cells in y-direction}{up to 512}{1}
{Not used in 2D. It includes the boundary ghost cells.}

\item
\PDACCONFWDEF{nz}{Number of cells in z-direction}{up to 512}{100}
{z is the second space coordinate in 2D. It includes the boundary ghost cells.}

\item
\PDACCONFWDEF{n0x}{Number of uniform cells in x-direction}{up to 512}{10}
{Number of cells with minimum size in x-direction}

\item
\PDACCONFWDEF{n0y}{Number of uniform cells in y-direction}{up to 512}{10}
{Number of cells with minimum size in y-direction}

\item
\PDACCONFWDEF{n0z}{Number of uniform cells in z-direction}{up to 512}{10}
{Number of cells with minimum size in z-direction}

\item
\PDACCONFWDEF{npx}{Number of cells to add in x-direction }{up to 512}{0}
{Adds cells with maximum size \OBJ{dxmax} at the end of the x mesh.}

\item
\PDACCONFWDEF{nmx}{Number of cells to add in x-direction }{up to 512}{0}
{Adds cells with maximum size \OBJ{dxmax} at the beginning of the x mesh.}

\item
\PDACCONFWDEF{npy}{Number of cells to add in y-direction }{up to 512}{0}
{Adds cells with maximum size \OBJ{dymax} at the end of the y mesh.}

\item
\PDACCONFWDEF{nmy}{Number of cells to add in y-direction }{up to 512}{1}
{Adds cells with maximum size \OBJ{dymax} at the beginning of the y mesh.}

\item
\PDACCONFWDEF{npz}{Number of cells to add in z-direction }{up to 512}{1}
{Adds cells with maximum size \OBJ{dzmax} at the end of the z mesh.}

\item
\PDACCONFWDEF{nmz}{Number of cells to add in z-direction }{up to 512}{1}
{Adds cells with maximum size \OBJ{dzmax} at the beginning of the z mesh.}

\item
\PDACCONFWDEF{itc}{flag for cylindrical coordinates}{0,1}{0}
{In 2D simulations, itc=1 sets the grid and the coordinates to cylindrical,
by modifying the discretized equations.}

\item
\PDACCONFWDEF{iuni}{flag for uniform mesh}{0,1}{0}
{0 - non uniform mesh; 1 - uniform mesh, takes {\tt dx0, dy0, dz0} as the
cells size in the two/three coordinate directions.}

\item
\PDACCONFWDEF{dx0}{cell sizes in x(r)-direction}{any real $>$ 0.0}{10.D0}
{Size along x(r) of the cells in metres, when the flag for uniform mesh is selected}

\item
\PDACCONFWDEF{dy0}{cell sizes in y-direction}{any real $>$ 0.0}{10.D0}
{Size along y of the cells in metres, when the flag for uniform mesh is selected}

\item
\PDACCONFWDEF{dz0}{cell sizes in z-direction}{any real $>$ 0.0}{10.D0}
{Size along z of the cells in metres, when the flag for uniform mesh is selected}

\item
\PDACCONFWDEF{grigen}{flag for grid generation}{0/1}{0}
{Automatically builds a non-uniform mesh when 1 is selected}

\item
\PDACCONFWDEF{maxbeta}{Maximum increase rate for non-uniform cells}{any real $>$ 1.0}{1.2}
{The ratio between the size of two adjacent cells. Increasing the non-uniform rate can deteriorate the accuracy of the numerical solution.}

\item
\PDACCONFWDEF{dxmin}{Minimun cell size in x(r) direction}{any real $>$ 0.0}{10.0}
{The minimum cell size (in metres) determines the highest numerical resolution and constraints the time-step size accordingly to the CFL condition (see above)}

\item
\PDACCONFWDEF{dymin}{Minimun cell size in y direction}{any real $>$ 0.0}{10.0}
{The minimum cell size (in metres) determines the highest numerical resolution and constraints the time-step size accordingly to the CFL condition (see above)}

\item
\PDACCONFWDEF{dzmin}{Minimun cell size in z direction}{any real $>$ 0.0}{10.0}
{The minimum cell size (in metres) determines the highest numerical resolution and constraints the time-step size accordingly to the CFL condition (see above)}

\item
\PDACCONFWDEF{dxmax}{Maximum cell size in x(r) direction}{any real $>$ 0.0}{10.0}
{The cell size is increased at a given rate from minimum to maximum size, until the domain\_x is covered. The code stops if the maximum cell size is too small.}

\item
\PDACCONFWDEF{dymax}{Maximum cell size in y direction}{any real $>$ 0.0}{10.0}
{The cell size is increased at a given rate from minimum to maximum size, until the domain\_y is covered. The code stops if the maximum cell size is too small.}

\item
\PDACCONFWDEF{dzmax}{Maximum cell size in z direction}{any real $>$ 0.0}{10.0}
{The cell size is increased at a given rate from minimum to maximum size, until the domain\_z is covered. The code stops if the maximum cell size is too small.}

\item
\PDACCONFWDEF{alpha\_x}{Relative position of the mesh center}{[0.0:1.0]}{0.5}
{The center of the mesh corresponds to the position of the more refined part of the domain in the x(r) direction}

\item
\PDACCONFWDEF{alpha\_y}{Relative position of the mesh center}{[0.0:1.0]}{0.5}
{The center of the mesh corresponds to the position of the more refined part of the domain in the y direction}

\item
\PDACCONFWDEF{alpha\_z}{Relative position of the mesh center}{[0.0:1.0]}{0.0}
{The center of the mesh corresponds to the position of the more refined part of the domain in the z direction}

\item
\PDACCONFWDEF{center\_x}{UTM longitude of the mesh center}{}
{The UTM longitude is needed to reference the mesh when volcanic topography has to be read. Usually corresponds to the crater vent longitude}

\item
\PDACCONFWDEF{center\_y}{UTM latitude of the mesh center}{}
{The UTM longitude is needed to reference the mesh when volcanic topography has to be read. Usually corresponds to the crater vent latitude}

\item
\PDACCONFWDEF{domain\_x}{Domain size along x(r)}{any real $>$ 0.0}{10000}
{Size (in metres) of the computational domain along x(r) for automatic mesh generation}

\item
\PDACCONFWDEF{domain\_y}{Domain size along x(r)}{any real $>$ 0.0}{10000}
{Size (in metres) of the computational domain along y for automatic mesh generation}

\item
\PDACCONFWDEF{domain\_z}{Domain size along x(r)}{any real $>$ 0.0}{10000}
{Size (in metres) of the computational domain along z for automatic mesh generation}

\item
\PDACCONFWDEF{zzero}{Grid bottom level}{any real}{0.0}
{Elevation of the mesh bottom above sea level. The atmosphere is correctly
described below 80 Km.}

\item
\PDACCONFWDEF{mesh\_partition}{domain decomposition criterion}{1,2,3}{1}
{1 - Layers decomposition, 2 - 2D Blocks decomposition, 3 - 3D Blocks decomposition}

\end{itemize}

\subsubsection{\OBJ{Boundaries} namelist}
The \OBJ{Boundaries} namelist contains the flags that specify the
conditions to be imposed on the domain boundaries. The domain boundaries
are identified by their compass position, i.e. west-east in x-direction,
south-north in y-direction and bottom-top in z-direction. The flag for the
use of the immersed boundaries technique for no-slip boundaries is also
included.
\begin{itemize}
\item
\PDACCONFWDEF{west}{Flag for west boundary}{integer $\le$ 20}{2}
{In 2D cylindrical, it must accounts for axial symmetry (flag=2). In 3D, 
inflow/ouflow conditions (flag = 4,6) are usually imposed.}

\item
\PDACCONFWDEF{east}{Flag for east boundary}{integer $\le$ 20}{2}
{In 2D and 3D inflow/ouflow conditions (flag = 4,6) are usually imposed.}

\item
\PDACCONFWDEF{south}{Flag for south boundary}{integer $\le$ 20}{2}
{Inflow/ouflow conditions (flag = 4,6) are usually imposed (just 3D).}

\item
\PDACCONFWDEF{north}{Flag for north boundary}{integer $\le$ 20}{2}
{Inflow/ouflow conditions (flag = 4,6) are usually imposed (just 3D).}

\item
\PDACCONFWDEF{bottom}{Flag for bottom boundary}{integer $\le$ 20}{3}
{Bottom boundary condition represents usually a no-slip solid surface}

\item
\PDACCONFWDEF{top}{Flag for top boundary}{integer $\le$ 20}{4}
{Please consider that top boundary can influence the behaviours of
the atmospheric conditions during the simulation.}

\item
\PDACCONFWDEF{immb}{Flag for immersed boundaries}{0/1}{0}
{When immb = 1, a generic solid boundary immersed in the computational domain
is described by the immersed-boundaries technique. Please notice that this 
procedure can be computationally expensive.}

\item
\PDACCONFWDEF{ibl}{Flag for blunt body}{0/1}{0}
{When ibl = 1, the code writes out the lift and drag forces at each time-step
on a rectangular block defined in the FIXED\_FLOWS card.}

\end{itemize}

\subsubsection{\OBJ{topography} namelist}
The present version of \PDAC can include (in 2D and 3D) an external file specifying a generic volcano topography. In 2D the topography can be given on a non-uniform mesh, whereas in 3D it is assumed that a standard ASCII DEM file is read.

\begin{itemize}
\item
\PDACCONFWDEF{dem\_file}{Name of the DEM external file}{string between single quotes}{'topo.dat'}
{Please controls that the ASCII file does not contain meta-characters (e.g. from Windows coding or decompression).}

\item
\PDACCONFWDEF{itp}{Flag for importing DEM file}{0/1}{0}
{If itp = 1, read the topography from a file.}

\item
\PDACCONFWDEF{ismt}{Flag for smoothing DEM}{1/2/3}{1}
{Smoothing of the DEM is performed accordingly to a median filter (1), a 
Gaussian filter (2) or a Barnes filter (3). The filtersize is defined
by the corresponding input parameter.}

\item
\PDACCONFWDEF{iavv}{Flag for averaging the topography}{0/1}{0}
{Averages the topography to obtain an axisymmetric topography.}

\item
\PDACCONFWDEF{itrans}{Flag for vertical mesh translation} {0/1}{0}
{Translates the mesh vertically up to the minimum topographic elevation (to save cells).}

\item
\PDACCONFWDEF{nocrater}{Flag for crater flattening}{T/F}{F}
{Allows to flatten a region of the volcano around the vent at a given rim\_quota (see below).}

\item
\PDACCONFWDEF{rim\_quota}{Elevation a.s.l. where the topography is flattened}{any real}{1000}
{If 'nocrater' is selected, flatten the topography at this quota.}

\item
\PDACCONFWDEF{filtersize}{High-pass filtering size (in metres) for the topography}{any real $>$ 0.0}{50}
{The filtering procedure is obtained by a combination of sub-sampling and averaging.}

\item
\PDACCONFWDEF{cellsize}{Resample the topography on a given mesh}{any real $>$ 0.0}{10}
{This value is usually set equal to the dxmin, dymin values}

\item
\PDACCONFWDEF{zrough}{roughness length}{real}{1.0}
{Default Roughness length. Typical values for ground
ranges from few centimeters to some metres.}

\end{itemize}

\subsubsection{\OBJ{Inlet} namelist}
This namelist is intended for 3D simulations. It is useful to impose the vent
conditions on a {\em circular} vent, without specifying the flow field in each 
cell of the vent. It also includes some procedures to correct the description
of a circular source within a cartesian mesh.

\begin{itemize}
\item
\PDACCONFWDEF{ivent}{Flag for inlet conditions on a circular vent}{0/1}{0}
{Vent conditions are automatically imposed on the topography if {\tt ibl=1}}

\item
\PDACCONFWDEF{iali}{Flag for anti-aliasing}{0/1/2/3}{0}
{Activates the anti-aliasing procedure to correct the mass flux in the cells
partially filled by the topography, on the vent boundary. 0: no antialias;
1: density antialias (including gas); 2: density antialias (only particles); 
3: velocity antialias}

\item
\PDACCONFWDEF{irand}{Flag for random switch}{0/1}{0}
{Activates the random-switch procedure. The inlet cells are switched on
(i.e. the flux is feeded) with a probability proportional to the area not 
occupied by the topography.}

\item
\PDACCONFWDEF{ipro}{Flag for flow radial profile}{0/1}{0}
{Allows to set a radial inflow profile read from an external file (see below).}

\item
\PDACCONFWDEF{rad\_file}{Name of the file defining the profile}{string between sinfle quotes}{'profile.dat'}
{The input file is described in the i/o files section of this manual}

\item
\PDACCONFWDEF{xvent}{UTM longitude of the center of the vent}{any real}{0.0}
{The mesh is translated so that the vent is located in the more refined region}

\item
\PDACCONFWDEF{yvent}{UTM latitude of the center of the vent}{any real}{0.0}
{The mesh is translated so that the vent is located in the more refined region}

\item
\PDACCONFWDEF{vent\_radius}{Radius of the vent}{any real $>$ 0.0}{100}
{The radius in metres}

\item
\PDACCONFWDEF{base\_radius}{The radius of the base of the crater}{any real $>$ 0.0}{200}
{This value must be equal, at least, to the double of the vent radius.}

\item
\PDACCONFWDEF{crater\_radius}{The external radius of the crater}{any real}{500}
{This value is approximative, and it is used to flatten the crater, if prescribed.}

\item
\PDACCONFWDEF{u\_gas}{Inlet gas velocity x}{any real}{0.0}
{}

\item
\PDACCONFWDEF{v\_gas}{Inlet gas velocity y}{any real}{0.0}
{}

\item
\PDACCONFWDEF{w\_gas}{Inlet (averaged) gas velocity z}{any real}{0.0}
{}

\item
\PDACCONFWDEF{wrat}{Ratio between the maximum and the averaged vertical gas veloctiy}{any real $>$ 1.0}{1.0}
{If this value is $>$ 1.0, the velocity has a maximum on the axis and falls to zero at the vent rim. The average is constant.}

\item
\PDACCONFWDEF{p\_gas}{Pressure at the vent}{any real}{101325.0}
{}

\item
\PDACCONFWDEF{t\_gas}{Temperature at the vent}{any real $>$ 0.0}{288.15}{}

\item
\PDACCONFWDEF{u\_solid}{Inlet particle velocity x}{any real}{0.0}
{The values for different particle classes must be separated by commas.}

\item
\PDACCONFWDEF{v\_solid}{Inlet particle velocity y}{any real}{0.0}
{The values for different particle classes must be separated by commas.}

\item
\PDACCONFWDEF{w\_solid}{Inlet particle velocity z}{any real}{0.0}
{The values for different particle classes must be separated by commas.}

\item
\PDACCONFWDEF{ep\_solid}{Averaged particle fraction}{any real $>$ 0.0}{0.0}
{The values for different particle classes must be separated by commas.}

\item
\PDACCONFWDEF{t\_solid}{Particle temperature}{any real $>$ 0.0}{288.15}
{The values for different particle classes must be separated by commas.}

\item
\PDACCONFWDEF{vent\_O2}{Averaged concentration of O2 at the vent}{any real [0.0:1.0]}{}{}

\item
\PDACCONFWDEF{vent\_N2}{Averaged concentration of N2 at the vent}{any real [0.0:1.0]}{0.0}{}

\item
\PDACCONFWDEF{vent\_CO2}{Averaged concentration of CO2 at the vent}{any real [0.0:1.0]}{0.0}{}

\item
\PDACCONFWDEF{vent\_H2}{Averaged concentration of H2 at the vent}{any real [0.0:1.0]}{0.0}{}

\item
\PDACCONFWDEF{vent\_H2O}{Averaged concentration of H2O at the vent}{any real [0.0:1.0]}{1.0}{}

\item
\PDACCONFWDEF{vent\_Air}{Averaged concentration of Air at the vent}{any real [0.0:1.0]}{0.0}{}

\item
\PDACCONFWDEF{vent\_SO2}{Averaged concentration of SO2 at the vent}{any real [0.0:1.0]}{0.0}{}
\end{itemize}

\subsubsection{\OBJ{Dome} namelist}
This namelist contains the numerical and model parameters to define initial
conditions for dome explosion simulations. It includes the parameters for
the Woods et al. (2002) model for the dome pressurization.
This namelist is intended for both 2D/3D simulations. For 2D cylindrical
simulations, the dome center is located on the symmetry axis.

\begin{itemize}
\item
\PDACCONFWDEF{idome}{Flag for inlet conditions on a spherical dome}{0/1/2}{0}
{0: no dome; 1: gas overpressure set through porous dome model; 2: constant overpressure.
Dome cells are automatically imposed over the topography if {\tt ibl=1}}

\item
\PDACCONFWDEF{idw}{Flag for hydrostatic pressure}{0/1}{0}
{1: add hydrostatic pressure of the dome edifice to the gas overpressure}

\item
\PDACCONFWDEF{xdome}{UTM longitude of the center of the dome}{any real}{0.0}
{The dome can be located out from the more refined region}

\item
\PDACCONFWDEF{ydome}{UTM latitude of the center of the dome}{any real}{0.0}
{The dome can be located out from the more refined region}

\item
\PDACCONFWDEF{dome\_volume}{Volume of the dome}{any real $>$ 0.0}{1.D6}
{Note: the dome volume includes the gas fraction (it does NOT correspond to the
Dense Rock Equivalent volume). The total mass in the dome resulting from the
automatic procedure depends on the mesh discretization and on the topography.}

\item
\PDACCONFWDEF{conduit\_radius}{minimal radius of the dome}{any real $>$ 0.0}{15.D0}
{The dome pressure is constant within the conduit\_radius}

\item
\PDACCONFWDEF{overpressure}{Constant overpressure can be imposed within the dome}{any real $>$ 0.0}{100.D5}
{This value is used if {\tt idome = 2}}

\item
\PDACCONFWDEF{particle\_fraction}{particle volumetric fractions}{array of real [0.0:1.0]}{0.7}
{The size of the array equals {\tt nsolid}. The sum of the particle volumetric fractions
must not exceed 1.0}

\item
\PDACCONFWDEF{gas\_flux}{Gas flux from the conduit}{any real $>$ 0.0}{400.0}
{}

\item
\PDACCONFWDEF{temperature}{Dome temperature in Kelvin}{any real $>$ 0.0}{1100.D0}
{}

\item
\PDACCONFWDEF{permeability}{Dome permeability}{any real $>$ 0.0}{1.D-12}{}

\item
\PDACCONFWDEF{dome\_gasvisc}{Gas viscosity}{any real $>$ 0.0}{1.D-5}{}

\item
\PDACCONFWDEF{dome\_O2}{O2 mass fraction}{any real [0.0:1.0]}{0.0}{}

\item
\PDACCONFWDEF{dome\_N2}{N2 mass fraction}{any real [0.0:1.0]}{0.0}{}

\item
\PDACCONFWDEF{dome\_CO2}{CO2 mass fraction}{any real [0.0:1.0]}{0.0}{}

\item
\PDACCONFWDEF{dome\_H2}{H2 mass fraction}{any real [0.0:1.0]}{0.0}{}

\item
\PDACCONFWDEF{dome\_H2O}{H2O mass fraction}{any real [0.0:1.0]}{1.0}{}

\item
\PDACCONFWDEF{dome\_Air}{Air mass fraction}{any real [0.0:1.0]}{0.0}{}

\item
\PDACCONFWDEF{dome\_SO2}{SO2 mass fraction}{any real [0.0:1.0]}{0.0}{}
\end{itemize}

\subsubsection{\OBJ{Atmosphere} namelist}
This namelist contains the parameters to build a stratified atmosphere from
the ground pressure and temperature. For seven atmospheric layers (troposphere,
tropopause, lower\_stratosphere, upper\_stratosphere, ozone\_layer, lower\_mesosphere and upper\_mesosphere) the distance of the top level from the ground and the temperature gradient must be given. Gas concentrations are constant. A constant wind velocity can also be specified.

\begin{itemize}
\item
\PDACCONFWDEF{wind\_x}{Component x of wind velocity}{any real}{0.0}
{}

\item
\PDACCONFWDEF{wind\_y}{Component y of wind velocity}{any real}{0.0}
{}
\item
\PDACCONFWDEF{wind\_z}{Component z of wind velocity}{any real}{0.0}
{}
\item
\PDACCONFWDEF{p\_ground}{Pressure at sea level}{any real}{101325.0}
{If stratification is switched off (gravity equals zero), this corresponds to the ambient pressure}

\item
\PDACCONFWDEF{t\_ground}{Temperature at sea level}{any real $>$ 0.0}{288.15}
{If stratification is switched off (gravity equals zero), this corresponds to the ambient temperature}

\item
\PDACCONFWDEF{void\_fraction}{Gas volume fraction in the atmosphere}{any real [0.0:1.0]}{1.0}{}

\item
\PDACCONFWDEF{max\_packing}{Maximum packing particle fraction}{any real [0.0:1.0]}{0.6413}{}

\item
\PDACCONFWDEF{atm\_O2}{Concentration of oxigen in the atmosphere}{any real [0.0:1.0]}{0.0}{}

\item
\PDACCONFWDEF{atm\_N2}{Concentration of nitrogen in the atmosphere}{any real [0.0:1.0]}{0.0}{}

\item
\PDACCONFWDEF{atm\_CO2}{Concentration of carbon dioxide in the atmosphere}{any real [0.0:1.0]}{0.0}{}

\item
\PDACCONFWDEF{atm\_H2}{Concentration of hydrogen in the atmosphere}{any real [0.0:1.0]}{0.0}{}

\item
\PDACCONFWDEF{atm\_H2O}{Concentration of water vapour in the atmosphere}{any real [0.0:1.0]}{0.0}{}

\item
\PDACCONFWDEF{atm\_Air}{Concentration of air in the atmosphere}{any real [0.0:1.0]}{1.0}
{The mass fraction (concentration) of each of the seven gas species must be
specified. The closure relation for the total mass fraction must be satisfied.
If this constrain is not satisfied, the Air mass fraction is
automatically corrected.}

\item
\PDACCONFWDEF{atm\_SO2}{Concentration of sulphur dioxide in the atmosphere}{any real [0.0:1.0]}{0.0}{}

\item
\PDACCONFWDEF{troposphere\_z}{Upper limit of the troposphere}{}{11000 m}{}

\item
\PDACCONFWDEF{troposphere\_grad}{Temperature gradient in the troposphere}{}{-6.5D-3}{}

\item
\PDACCONFWDEF{tropopause\_z}{Upper limit of the tropopause}{}{20000 m}{}

\item
\PDACCONFWDEF{tropopause\_grad}{Temperature gradient in the tropopause}{}{0.0}{}

\item
\PDACCONFWDEF{lower\_stratosphere\_z}{Upper limit of the lower stratosphere}{}{32000 m}{}

\item
\PDACCONFWDEF{lower\_stratosphere\_grad}{Temperature gradient in the loser stratosphere}{}{1.D-3}{}

\item
\PDACCONFWDEF{upper\_stratosphere\_z}{Upper limit of the upper stratosphere}{}{47000 m}{}

\item
\PDACCONFWDEF{upper\_stratosphere\_grad}{Temperature gradient in the upper stratosphere}{}{1.0D-3}{}

\item
\PDACCONFWDEF{ozone\_layer\_z}{Upper limit of the ozone layer}{}{51000 m}{}

\item
\PDACCONFWDEF{ozone\_layer\_grad}{Temperature gradient in the ozone layer}{}{0.0}{}

\item
\PDACCONFWDEF{lower\_mesosphere\_z}{Upper limit of the lower mesosphere}{}{71000 m}{}

\item
\PDACCONFWDEF{lower\_mesosphere\_grad}{Temperature gradient in the lower mesosphere}{}{-2.8D-3}{}

\item
\PDACCONFWDEF{upper\_mesosphere\_z}{Upper limit of the upper mesosphere}{}{80000 m}{}

\item
\PDACCONFWDEF{upper\_mesosphere\_grad}{Temperature gradient in the upper mesosphere}{}{-2.0D-3}{}
\end{itemize}

\subsubsection{\OBJ{particles} namelist}
In this namelist the number of solid phases and the physical properties of the
particles forming each phase are specified. Note that the order in which 
particle classes are specified must be the order of the solid-phases array 
storage, in order to be consistent with the initial conditions assignement.

\begin{itemize}

\item
\PDACCONFWDEF{nsolid}{number of solid phases}{up to 10}{2}
{The number of solid phases considered in the multiphase equations.
The maximum number can be increased by modifying the max\_nsolid parameter
in the ``dimensions'' module.}

\item
\PDACCONFWDEF{diameter}{effective diameter particles (in microns)}{real}{100 microns}
{The model works well for particle diameters smaller than few millimeters}

\item
\PDACCONFWDEF{density}{microscopic particle density}{real}{2700 Kg/m$^3$}
{Depends on materials and porosity}

\item
\PDACCONFWDEF{sphericity}{particle sphericity}{0.5 to 1.0}{1.0}
{Partially accounts for particle shapes}

\item
\PDACCONFWDEF{viscosity}{Empirical viscosity coefficient}{0.5 to 2.0}{0.5[Pa s]}
{Largest values apply to coarse particles}

\item
\PDACCONFWDEF{specific\_heat}{solid specific heat}{real}{1.2D3 [J/(K Kg)]}
{Depends on materials}

\item
\PDACCONFWDEF{thermal\_conductivity}{solid thermal conductivity for Fourier law}{real}{2.D0}
{Depends on materials}

\end{itemize}

\subsubsection{\OBJ{numeric} namelist}
By modifying these parameters, you can modify the way \PDAC\ solves the 
model equations. We recommend non-expert users to avoid modifying the
default values.

\begin{itemize}

\item
\PDACCONFWDEF{rungekut}{order of Runge-Kutta explicit integration}{1,2,3}{1}
{The low-storage Runge-Kutta algorithm is used for explicit time integration.
The coefficients used in the RK integration are well suited only up to the
third-order. High-order temporal integration is recommended for 
convergence and stability when high-order spatial discretization schemes 
are used.}

\item
\PDACCONFWDEF{beta}{degree of upwinding}{0.0 to 1.0}{0.25}
{When High Order spatial discretization schemes are used and the MUSCL 
beta-scheme (both limited or unlimited) is selected, the degree of upwinding
can be chosen: \OBJ{beta}=0.0 corresponds to a completely centered schemes:
\OBJ{beta}=1.0 corresponds to a completely upwinded method. The accuracy depends
on the exact value: for \OBJ{beta}=0.25 and \OBJ{beta}=0.33 the formal third-order
is achieved.}

\item
\PDACCONFWDEF{muscl}{flag for MUSCL technique}{0,1}{0}
{The MUSCL technique extends the first-order discretization
scheme to high-orders by reconstructing the flux profile more accurately.
0 - uses first-order upwind; 1 - uses MUSCL technique.}

\item
\PDACCONFWDEF{mass\_limiter}{limiter type for mass equations}{0,1,2,3,4}{0}
{Select the type of MUSCL reconstruction and the limiter:\\
0 - beta scheme unlimited; 1 - Van Leer limiter; 2 - Minmod limiter;
3 - Superbee limiter; 4 - beta limited.}

\item
\PDACCONFWDEF{vel\_limiter}{limiter type for momentum equations}{0,1,2,3,4}{0}
{Select the type of MUSCL reconstruction and the limiter:\\
0 - beta scheme unlimited; 1 - Van Leer limiter; 2 - Minmod limiter;
3 - Superbee limiter; 4 - beta limited.}
\item
\PDACCONFWDEF{inmax}{maximum number of inner iterations}{integer (small)}{8}
{The exact value does not affect strongly the convergence but can slow down
the simulation. Optimal value is set by default.}

\item
\PDACCONFWDEF{maxout}{maximum number of Gauss-Siedel iterations for convergence}{integer (large)}{5000}
{Convergence is usually reached within less than 100 iterations. If maxout is reached the code CRASH!es}

\item
\PDACCONFWDEF{omega}{over/under-relaxation parameter}{0.0 to 2.0}{1.0}
{Values between 0.0 and 1.0 under-relax the iterative procedure, whereas
values above 1.0 overrelax.}

\item
\PDACCONFWDEF{optimization}{optimization degree}{1/2/3}{1}
{Level 1: not optimized. Level 2: Optimized memory access. Level 3: Optimized for 3D and 2 particle classes.}

\item
\PDACCONFWDEF{rlim}{Multiphase limit}{very small real value}{$10^{-8}$}
{It is related to minimum particle concentration for which multiphase flow
equations of momentum are solved. Below this limit one-phase equations are solved
(often critical for convergence on the cloud margins).}

\item
\PDACCONFWDEF{flim}{Multiphase limit}{very small real value}{$10^{-6}$}
{It is related to minimum particle concentration for which multiphase flow
equations of energy are solved. Below this limit one-phase equations are solved
(often critical for convergence on the cloud margins).}

\item
\PDACCONFWDEF{implicit\_fluxes}{flag for implicit computation of fluxes}{{\tt T/F}}{F}
{By selecting T, convective fluxes are computed (at the first or second-order, 
depending on flag \OBJ{muscl}) within the iterative solver. This modification
significantly slows down the computation but could relax the CFL constraint
so that larger time-steps can be used.}

\item
\PDACCONFWDEF{implicit\_enthalpy}{flag for implicit computation of enthalpies}
{{\tt T/F}}{F}
{Selecting the implicit solution of enthalpies enhances the level of coupling
between the momentum and enthalpy equations. Nevertheless the convective part 
of the enthalpy equations can be left out from the iterative solver
by opportunely selecting the flag \OBJ{implicit\_fluxes}. The performance loss
in any case is significant.}

\item
\PDACCONFWDEF{update\_eosg}{Update the equation of state}{{\tt T/F}}{T}
{Updating the equation of state after convergence is consistent with the old
version of PDAC but gives wrong solutions in test case (e.g. 1D and 2D shock waves)}
\end{itemize}

\subsection{\PDAC\ input cards}
\label{section:cards}

\subsubsection{\OBJ{'MESH'} card}

The \OBJ{'MESH'} card contains the two (in 2D) or three (in 3D) arrays of cell 
sizes specified for a non-uniform rectilinear mesh. 

\begin{itemize}
\item
\PDACCONF{dx(1:nx)}{Array of the cell sizes in x(r)-direction}{real}
{Array of the cell sizes in x(r)-direction.}

\item
\PDACCONF{dy(1:ny)}{Array of the cell sizes in y-direction}{real}
{Array of the cell sizes in y-direction.
Not present in 2D.}

\item
\PDACCONF{dz(1:nz)}{Array of the cell sizes in z-direction}{real}
{Array of the cell sizes in z-direction.}
\end{itemize}

Array elements can be separated by commas, tabs, blanks or lay on different
lines.  When uniform mesh is selected by {\tt iuni=1} parameter in the 
{\tt mesh} namelist, the card can be empty (but the card 
name must always be present in the input file).

\subsubsection{\OBJ{'FIXED\_FLOWS'} card}

The \OBJ{'FIXED\_FLOWS'} card is designed to specify boundary conditions (b.cs.).
Its name is due to the possibility to assign within this card any region with
specified flow conditions (e.g. for inlet flow). B.cs. are imposed in 
mesh region determined by rectangular blocks. The number of blocks is the
first parameter specified in the card

\begin{itemize}
\item
\PDACCONF{number\_of\_block}{number of blocks}{integer}
{The number of blocks used to specify b.c.}
\item
\PDACCONF{nblu}{Block tagged for the blunt body routines}{array of integers}
{If {\tt ibl=1} this array must contain a sequence of [0/1] indicating which
blocks must be tagged for the computation of drag and lift forces. If
{\tt ibl=0} it must not be present.}
\end{itemize}

Each block is characterized
by a flag that specifies the kind of b.c., and by two (in 2D) or three (in 3D)
couples of integer that specify the first and the last cell of the block in 
the two (three) directions. Each block is therefore identified by 5 (in 2D)
or 7 (in 3D) integers:\\

\begin{itemize}
\item
\PDACCONF{block\_type}{type of b.c.}{1 to 7}
{1 - fluid cells: all equations are solved; 2 - free-slip;
 3 - no-slip; 4 - free in/outflow; 5 - specified flow (inlet); 
 6 - extrapolation free in/outflow}

\item
\PDACCONF{block\_bounds}{limits of blocks}{integer}
{two or three couples of integers specifying 
$ i_{min}, i_{max}, (j_{min}, j_{max}), k_{min}, k_{max}$ }

\end{itemize}

Blocks are used also to specify ``blocking cells'' (e.g. 
the volcano topography). When the block type is equal to 5 (specified
flow), inlet conditions must be specified as follows:\\

First line:

\begin{itemize}
\item
\PDACCONF{fixed\_vgas\_x}{gas velocity component along x}{real}
{Inlet gas velocity along x(r)-direction. Note that CFL condition (see above for \OBJ{dt}) 
must be satisfied for every component.}

\item
\PDACCONF{fixed\_vgas\_y}{gas velocity component along y}{real}
{Inlet gas velocity along x(r)-direction (not present in 2D!).}

\item
\PDACCONF{fixed\_vgas\_z}{gas velocity component along z}{real}
{Inlet gas velocity along z-direction.}

\item
\PDACCONF{fixed\_pressure}{pressure at inlet}{real}
{The thermodynamic pressure of the gas phase.}

\item
\PDACCONF{fixed\_gaseps}{inlet gas volumetric fraction}{ $\le 1.0$}
{This value must be consistent with the solid phases volumetric fractions.
The closure relation imposes that the sum equals one.}

\item
\PDACCONF{fixed\_gastemp}{inlet gas temperature}{ a positive real }
{No constraint is imposed on gas temperature, but the range of validity
of the equation of state must be check.}
\end{itemize}

Further {\tt nsolid} lines:
\begin{itemize}
\item
\PDACCONF{fixed\_vpart\_x}{particle velocity along x}{real}
{Inlet particle velocity along x(r)-direction.
Note that CFL condition (see above for \OBJ{tt}) 
must be satisfied for every component.}

\item
\PDACCONF{fixed\_vpart\_y}{particle velocity along y}{real}
{Inlet particle velocity along y-direction(not present in 2D!).}

\item
\PDACCONF{fixed\_vpart\_z}{particle velocity along z}{real}
{Inlet particle velocity along z-direction.}

\item
\PDACCONF{fixed\_parteps}{particle volumetric fraction}{ $\le 1.0$}
{Must satisfy closure relation of total volumetric fraction}

\item
\PDACCONF{fixed\_parttemp}{particle temperature}{ a positive real}
{Particle temperature is not constrained by a thermal equation of state}

\end{itemize}

Last line:

\begin{itemize}

\item
\PDACCONF{fixed\_gasconc(1:7)}{concentration of each gas species at inlet}{$\le 1.0$}
{The mass fraction (concentration) of each of the seven gas species must be
specified. The closure relation for the total mass fraction must be satisfied.
If this constrain is not satisfied, the {\tt default\_gas} mass fraction is
automatically corrected.}

\end{itemize}

