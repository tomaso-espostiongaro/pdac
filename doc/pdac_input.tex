\section{Simulation Parameters}
\label{section:basic}

% INSERT GENERAL DESCRIPTION OF THE INPUT PARAMETERS.

\subsection{\PDAC\ input parameters}
\label{section:config_basic}

\subsubsection{{\tt control} namelist}

\begin{itemize}
\item
\PDACCONF{run\_name}{identification}{any string}
{This string identifies the simulation. Its length must not exceed 80 characters. It must be enclosed by quotes.}

\item
\PDACCONFWDEF{job\_type}{2d or 3d}{{\tt '2d','2D','3d','3D'}}{{\tt '2D'}}
{When 2D is selected, the model equations are simplified by assuming
some simmetry in the physical domain (translation in one cartesian 
direction or cylindrical symmetry)}

\item
\PDACCONFWDEF{restart\_mode}{flag for restart}{{\tt 'from\_scratch'}, {\tt 'restart'}}
{{\tt 'from\_scratch'}}
{If {\tt restart} is selected, initial conditions are set up by using the fields
dumped (in double precision) into the restart file {\tt pdac.res} (see above). 
Otherwise, initial conditions are set up from input data in {\tt pdac.dat}} 

\item
\PDACCONF{time}{initial time}{double real}
{Inital simulation time (in seconds)}

\item
\PDACCONF{tstop}{end time}{double real}
{Time (in seconds) at wich the simulation stops}

\item
\PDACCONF{dt}{time step}{double real}
{Time advancement step (in seconds). 
{\tt dt} is constrained by the CFL condition
$dt < C_{max}\frac{\Delta x}{V_{max}}$, where $\Delta x$ is
the size of the computational grid and $V_{max}$ is the maximum
velocity. The maximum CFL number $C_{max}\approx 0.2$ has been 
found empirically.}

\item
\PDACCONF{lpr}{level of verbosity}{1,2}
{increases the level of verbosity in warning and error messages 
in {\tt pdac.log}, {\tt pdac.err}, {\tt pdac.tst} files}

\item
\PDACCONF{tpr}{time interval for OUTPUT file}{}
{OUTPUT files are printed every {\tt tpr} seconds of simulated time}

\item
\PDACCONF{tdump}{time interval for restart file}{}
{restart file is overwritten every {\tt tdump} seconds of simulated time}

\item
\PDACCONFWDEF{max\_seconds}{maximum CPU time}{any real value}{20000.0}
{Maximum duration of a simulation (in seconds). If time exceeds this value
a restart file is written before the simulation is stopped (useful for
scheduling). Default time is set for 6 hours simulations}

\item
\PDACCONFWDEF{nfil}{number of first OUTPUT file}{any positive integer < 9999}{0}
{OUTPUT files are written with 4 digits extension ({\tt OUTPUT.XXXX}). 
The output time can be recovered as $({\tt XXXX - nfil})* {\tt tpr}$ }

\item
\PDACCONF{old\_restart}{}{}
{}

\item
\PDACCONF{formatted\_output}{flag for OUTPUT format}{{\tt T/F}}
{Determine the format of output files: T - formatted ascii, 
F - binary 4-bytes }

\end{itemize}

\subsubsection{{\tt model} namelist}

\begin{itemize}

\item
\PDACCONFWDEF{irex}{chemical reactions}{0,1}{0}
{The chemical reaction module has not yet been implemented}

\item
\PDACCONFWDEF{gas\_viscosity}{gas diffusive transport}{{\tt T/F}}{{\tt T}}
{T to solve the full set of model equations. F switches the diffusive
transport terms off (viscous and turbulent term in momentum equation, thermal
diffusivity in enthalpy equation). 
Gas viscosity is computed anyway to include the gas-particle drag and
energy exchange terms.}

\item
\PDACCONFWDEF{part\_viscosity}{particle diffusive transport}{{\tt T/F}}{{\tt T}}
{Switches the diffusive terms in the particle transport equations 
(viscous term in the momentum equation and thermal conductivity).}

\item
\PDACCONFWDEF{iss}{turbulence model for particles}{0,1}{0}
{1: computes a gas-analogous sub-grid-stress (turbulent viscosity)
for particles.}

\item
\PDACCONFWDEF{repulsive\_model}{flag for Coulombic repulsive model}{0,1}{1}
{The Coulombic repulsive model add a contribution to the diagonal part of
the solid viscous stress due to the repulsive interaction of particles 
at high concentrations. This term appears to be important for the solid
equation to be well posed.}

\item
\PDACCONFWDEF{iturb}{turbulence model for gas}{0,1,2}{1}
{0 - no gas turbulence model; 1 - Smagorinsky sub-grid stress model;
2 - Smagorinsky sgs model with roughness closure at walls.}

\item
\PDACCONFWDEF{modturbo}{Subgrid-scale model for gas turbulence}{1,2}{1}
{1 - Classical Smagrinsky model; 2 - Dynamic Samgorinsky model}

\item
\PDACCONFWDEF{cmut}{Smagorinsky constant}{usually between 0.1 and 0.4}{0.1}
{The exact value cannot be predicted {\it a priori}. The use of dynamic
Smagorinsky model (see above) makes the assignment of this constant
non-necessary}

\item
\PDACCONFWDEF{rlim}{Multiphase limit}{very small real value}{$10^{-8}$}
{Relied to particle concentrations: for small values the multiphase equations
are solved in an approximate way. Often critical for convergence on the cloud 
margins.}

\item
\PDACCONFWDEF{gravx}{acceleration along x}{any real}{0.0}
{Can be used to simulated a physical constrain, such as a wall}

\item
\PDACCONFWDEF{gravy}{acceleration along y}{any real}{0.0}
{Can be used to simulated a physical constrain, such as a wall}

\item
\PDACCONFWDEF{gravz}{acceleration along z}{any real}{-9.81}
{Usually the value of gravitational acceleration. Values different
from the default value could be not consistent with the atmospheric
stratification, leading to instabilities. A value of 0.0 suppresses
atmospheric stratification}.

\item
\PDACCONFWDEF{ngas}{number of gas components}{1 to 7}{2}
{Seven gas species are defined, specifically: \\
$ 1) O_2, 2) N_2, 3), CO_2, 4) H_2, 5) H_2O, 6) Air, 7) SO_2$. 
The number of gas species must be consistent with
the other input conditions, such as the ``inlet'' boundary condtions,
otherwise the program will stop.}

\item
\PDACCONFWDEF{default\_gas}{gas species taken as default}{1 to 7}{6}
{Since the sum of the mass fraction of the different gas species must
close to 1.0, the {\tt default\_gas} is defined as the difference 
between 1.0 and the sum of the other species. The default gas must
be present.}

\end{itemize}

\subsubsection{{\tt mesh} namelist}

\begin{itemize}
\item
\PDACCONF{nx}{Number of cells in x(r)-direction}{up to 512}
{When cylindrical coordinates are selected in 2D, ``x'' is used instead 
 of ``r''. Includes the boundary ghost cells. The maximum number can be 
 increased by modifying the max\_size parameter in the ``dimensions'' module}

\item
\PDACCONF{ny}{Number of cells in y-direction}{up to 512}
{Not used in 2D. Includes the boundary ghost cells. Idem.}

\item
\PDACCONF{nz}{Number of cells in z-direction}{up to 512}
{z is the second space coordinate in 2D. Includes the boundary ghost cells.
Idem. }

\item
\PDACCONFWDEF{itc}{flag for cylindrical coordinates}{0,1}{0}
{In 2D simulations, itc=1 sets the grid and the coordinates to cylindrical,
by modifying the discretized equations.}

\item
\PDACCONFWDEF{iuni}{flag for uniform mesh}{0,1}{0}
{0 - non uniform mesh; 1 - uniform mesh, takes {\tt dx0, dy0, dz0} as the
cells size in the three coordinate directions.}

\item
\PDACCONF{dx0}{cell sizes in x(r)-direction}{any real}
{}

\item
\PDACCONF{dy0}{cell sizes in y-direction}{any real}
{}

\item
\PDACCONF{dz0}{cell sizes in z-direction}{any real}
{}

\item
\PDACCONF{origin\_x}{origin of the cartesian space}{any real}
{}

\item
\PDACCONF{origin\_y}{origin of the cartesian space}{any real}
{}

\item
\PDACCONF{origin\_z}{origin of the cartesian space}{any real}
{}

\item
\PDACCONFWDEF{mesh\_partition}{domain decomposition criterion}{1,2,3}{1}
{}

\end{itemize}

\subsubsection{{\tt particles} namelist}

\begin{itemize}

\item
\PDACCONF{nsolid}{number of solid phases}{up to 10}
{Maximum number can be increased by modifying the max\_nsolid parameter
in the ``dimensions'' module.}

\item
\PDACCONF{diameter}{effective diameter particles (in microns)}{real}
{The model works well for particle diameters smaller than few millimeters}

\item
\PDACCONF{density}{microscopic particle density}{real}
{Depends on materials and porosity}

\item
\PDACCONF{sphericity}{particle sphericity}{0.5 to 1.0}
{Partially accounts for particle shapes}

\item
\PDACCONF{viscosity}{Empirical viscosity coefficient}{0.5 to 2.0}
{Largest values apply to coarse particles}

\item
\PDACCONF{specific\_heat}{solid specific heat}{ral}
{Depends on materials}

\item
\PDACCONF{thermal\_conductivity}{solid thermal conductivity for Fourier law}{real}
{Depends on materials}

\end{itemize}

\subsubsection{{\tt numeric} namelist}

\begin{itemize}

\item
\PDACCONFWDEF{rungekut}{order of Runge-Kutta explicit integration}{1,2,3}{1}
{The low-storage Runge-Kutta algorithm is used for explicit time integration.
The coefficients used in the RK integration are well suited only up to the
third order. High order temporal integration is recommended for 
convergence and stability when high order spatial discretization schemes 
are used}

\item
\PDACCONFWDEF{beta}{degree of upwinding}{0.0 to 1.0}{0.25}
{When High Order spatial discretization schemes are used, and the MUSCL 
beta-scheme (both limited or unlimited) is selected, the degree of upwinding
can be chosen: beta=0.0 corresponds to a completely centered schemes:
beta=1.0 corresponds to a completely upwinded method. The accuracy depends
on the exact value: for beta=0.25 and beta=0.33 the formal third order
is achieved.}

\item
\PDACCONFWDEF{muscl}{flag for MUSCL technique}{0,1}{0}
{The MUSCL technique extends the first order discretization
scheme to higher orders by reconstructing the flux profile more accurately.
0 - uses first order upwind; 1 - uses MUSCL technique.}

\item
\PDACCONFWDEF{lim\_type}{limiter type}{0,1,2,3,4}{0}
{Select the type of MUSCL reconstruction and the limiter:\\
0 - beta scheme unlimited; 1 - VanLeer limiter; 2 - minmod limiter;
3 - Superbee limiter; 4 - beta limited.}

\item
\PDACCONFWDEF{inmax}{maximum number of inner iterations}{integer (small)}{8}
{The exact value does not affect strongly the convergence but can slow
the simulation down. Optimal value is set by default.}

\item
\PDACCONFWDEF{maxout}{maximum number of Gauss-Siedel iterations for convergence}{integer (large)}{5000}
{Convergence is usually reached within less than 100 iterations. If maxout is reached the code CRASHES}

\item
\PDACCONFWDEF{omega}{over/under-relaxation parameter}{0.0 to 2.0}{1.0}
{Values between 0.0 and 1.0 underrelax the iterative procedure, whereas
values above 1.0 overrelax.}

\end{itemize}

\subsubsection{{\tt ROUGHNESS} card}

The {\tt ROUGHNESS} card is now obsolete in 3D, but is mantained for 2D 
simulations (in 3D it will be replaced by the roughness matrix). In this
card three elements must be specified:
\begin{itemize}
\item
\PDACCONF{ir}{number of rough zones}{1,2}
{The code allows the specification of one or two rough regions, characterized
by different roughness lengths.}

\item
\PDACCONF{zrough(:)}{roughness length (in metres)}{real}
{Specify one roughness length for each rough region. Typical values for ground
ranges from few centimeters to some metres.}

\item
\PDACCONF{roucha}{distance of roughness change}{real}
{The distance from the left boundary at which the roughness changes.}

\end{itemize}

\subsubsection{{\tt MESH} card}

The {\tt MESH} card contains the two (in 2D) or three (in 3D) arrays of cell 
sizes specified for a non-uniform rectilinear mesh. 

\begin{itemize}
\item
\PDACCONF{dx(1:nx)}{cell sizes in x(r)-direction}{real}
{}

\item
\PDACCONF{dy(1:ny)}{cell sizes in y-direction}{real}
{Not present in 2D.}

\item
\PDACCONF{dz(1:nz)}{cell sizes in z-direction}{real}
{}
\end{itemize}

Array elements can be separated by commas, tabs, blanks or lay on different
lines.  When uniform mesh is selected by {\tt iuni=1} parameter in the 
{\tt mesh} namelist, the card can be empty (but the card 
name must always be present in the input file).

\subsubsection{{\tt FIXED\_FLOWS} card}

The {\tt FIXED\_FLOWS} card is designed to specify boundary conditions. Its
name is due to the possibility to assign within this card any region of
specified flow conditions (e.g. for inlet flow). B.c. are imposed in 
mesh region determined by rectangular blocks. The number of blocks is the
first parameter specified in the card

\begin{itemize}
\item
\PDACCONF{number\_of\_block}{number of blocks}{integer}
{The number of blocks used to specify b.c.}
\end{itemize}

Each block is characterized
by a flag that specifies the kind of b.c., and by two (in 2D) or three (in 3D)
couples of integer that specify the first and the last cell of the block in 
the two (three) directions. Each block is therefore identified by 5 (in 2D)
or 7 (in 3D) integers:\\

\begin{itemize}
\item
\PDACCONF{block\_type}{type of b.c.}{1 to 7}
{1 - fluid cells: all equations are solved; 2 - free-slip;
 3 - no-slip; 4 - free in/outflow; 5 - specified flow (inlet); 
 6 - specified pressure; 7 - specified input profile (only 2D).}

\item
\PDACCONF{block\_bounds}{limits of blocks}{integer}
{two or three couples of integers specifying 
$ i_{min}, i_{max}, (j_{min}, j_{max}), k_{min}, k_{max}$ }

\end{itemize}

Blocks are used also to specify ``blocking cells'' (e.g. to 
represent the topography). When the block type is equal to 5 (specified
flow), inlet conditions must be specified as follows:\\

First line:

\begin{itemize}
\item
\PDACCONF{fixed\_vgas\_x}{gas velocity component along x}{real}
{Inlet gas velocity. Note that CFL condition (see above for {\tt dt}) 
must be satisfied for every component.}

\item
\PDACCONF{fixed\_vgas\_y}{gas velocity component along y}{real}
{Inlet gas velocity (not present in 2D!).
Note that CFL condition (see above for {\tt dt}) 
must be satisfied for every component.}

\item
\PDACCONF{fixed\_vgas\_z}{gas velocity component along z}{real}
{Inlet gas velocity. Note that CFL condition (see above for {\tt dt}) 
must be satisfied for every component.}

\item
\PDACCONF{fixed\_pressure}{pressure at inlet}{real}
{The thermodynamic pressure}

\item
\PDACCONF{fixed\_gaseps}{inlet gas volumetric fraction}{ $\le 1.0$}
{This value must be consistent with the solid phases volumetric fractions.
The closure relation imposes that the sum equals one.}

\item
\PDACCONF{fixed\_gastemp}{inlet gas temperature}{ a positive real }
{No constraint is imposed on gas temperature, but the range of validity
of the equation of state must be check.}
\end{itemize}

Further {\tt nsolid} lines:
\begin{itemize}
\item
\PDACCONF{fixed\_vpart\_x}{particle velocity along x}{real}
{Inlet particle velocity.
Note that CFL condition (see above for {\tt dt}) 
must be satisfied for every component.}

\item
\PDACCONF{fixed\_vpart\_y}{particle velocity along y}{real}
{Inlet particle velocity (not present in 2D!).
Note that CFL condition (see above for {\tt dt}) 
must be satisfied for every component.}

\item
\PDACCONF{fixed\_vpart\_z}{particle velocity along z}{real}
{Inlet particle velocity.
Note that CFL condition (see above for {\tt dt}) 
must be satisfied for every component.}

\item
\PDACCONF{fixed\_parteps}{particle volumetric fraction}{ $\le 1.0$}
{Must satisfy closure relation of total volumetric fraction}

\item
\PDACCONF{fixed\_parttemp}{particle temperature}{ a positive real}
{Particle temperature is not constrained by a thermal equation of state}

\end{itemize}

Last line:

\begin{itemize}

\item
\PDACCONF{fixed\_gasconc(1:7)}{concentration of each gas species at inlet}{$\le 1.0$}
{The mass fraction (concentration) of each one of the seven gas species must be
specified. The closure relation for the total mass fraction must be satisfied.
If this constrain is not satisfied, the {\tt default\_gas} mass fraction is
automatically corrected.}

\end{itemize}

\subsubsection{{\tt INITIAL\_CONDITIONS} card}

The {\tt INITIAL\_CONDITIONS} card specifies the ambient conditions at 
the beginning of the simulation in all the computational cells of
type 1 (fluid cells).

First line:

\begin{itemize}
\item
\PDACCONF{initial\_vgas\_x}{gas velocity component along x}{real}
{Initial gas velocity. Can be used for cross-winds Note that CFL condition (see above for {\tt dt}) 
must be satisfied for every component.}

\item
\PDACCONF{initial\_vgas\_y}{gas velocity component along y}{real}
{Inlet gas velocity. Can be used for cross-winds (not present in 2D!).
Note that CFL condition (see above for {\tt dt}) 
must be satisfied for every component.}

\item
\PDACCONF{initial\_vgas\_z}{gas velocity component along z}{real}
{Inlet gas velocity. Can be used for cross-winds.
Note that CFL condition (see above for {\tt dt}) 
must be satisfied for every component.}

\item
\PDACCONF{initial\_pressure}{pressure at inlet}{real}
{The thermodynamic pressure in the domain (if gravity is switched off) or
at the ground (if atmosphere is stratified, values different from 1 atmosphere
lead to a WARNING message).}

\item
\PDACCONF{initial\_void\_fraction}{atmospheric gas volumetric fraction}
{ $\le 1.0$}
{Solid phases volumetric fractions are computed to satisfy the closure relation}

\item
\PDACCONF{initial\_temperature}{inlet gas temperature}{ a positive real }
{No constraint is imposed on gas temperature, but the range of validity
of the equation of state must be check. Particles are supposed to be in thermal
equilibrium with gas.}
\end{itemize}

Further {\tt nsolid} lines:
\begin{itemize}
\item
\PDACCONF{initial\_vpart\_x}{particle velocity along x}{real}
{Initial particle velocity.
Note that CFL condition (see above for {\tt dt}) 
must be satisfied for every component.}

\item
\PDACCONF{initial\_vpart\_y}{particle velocity along y}{real}
{Initial particle velocity (not present in 2D!).
Note that CFL condition (see above for {\tt dt}) 
must be satisfied for every component.}

\item
\PDACCONF{initial\_vpart\_z}{particle velocity along z}{real}
{Initial particle velocity.
Note that CFL condition (see above for {\tt dt}) 
must be satisfied for every component.}

\end{itemize}

Last line:

\begin{itemize}

\item
\PDACCONF{initial\_gasconc(1:7)}{concentration of each gas species in atmosphere}{$\le 1.0$}
{The mass fraction (concentration) of each one of the seven gas species must be
specified. The closure relation for the total mass fraction must be satisfied.
If this constrain is not satisfied, the {\tt default\_gas} mass fraction is
automatically corrected.}

\end{itemize}
